\documentclass[oneside]{article}
\usepackage{xcolor}
\usepackage{siunitx}
\usepackage{enumerate}
\usepackage{fancyhdr}
\usepackage{minted}
\usepackage{lastpage}
\usepackage{xcolor}
\usepackage{tcolorbox}
\usepackage{amsmath}
\usepackage[colorlinks=true]{hyperref}
\usepackage{setspace}
\usepackage[absolute]{textpos}
\usepackage{xepersian} % Always last package to load
\settextfont{B Nazanin}
\setlatintextfont{Times New Roman}
%\setlatinmonofont[Scale=0.8]{Monaco}
\defpersianfont\nastaliqfont{IranNastaliq}
\setlength{\TPHorizModule}{1cm}
\setlength{\TPVertModule}{1cm}
\linespread{1.5}
\pagestyle{fancy}
\renewcommand{\headrulewidth}{0pt}
\newcommand*{\fancypagenumber}{%
\fancyfoot[C]{صفحه
\thepage
از
\pageref{LastPage}}
}
\fancypagenumber
\fancypagestyle{plain}{\fancypagenumber}
% insert syntax highlighted code from a file
\newcommand{\inputcode}[2]{\inputminted[mathescape,%
                                                 linenos=true,%
                                                 formatcom=\small\setstretch{1}]{#1}{#2}}%
%\renewcommand{\theFancyVerbLine}{\sffamily\scriptsize
%\textcolor[rgb]{0.5,0.5,1.0}{\oldstylenums{\arabic{FancyVerbLine}}}}
\title{تمرین های سری دوم درس
\lr{Service Based Architecture}
}
%\author{دکتر شفیعی}
\date{\vspace{-5ex}}
\begin{document}
\maketitle

\begin{textblock}{5}(6.5,2)\nastaliqfont
\noindent\Large
بسم الله الرحمن الرحیم
\end{textblock}
%\begin{tcolorbox}
%\begin{center}
%این اولین سری از تمارین مربوطه به پایتون می باشد. در صورت داشتن سوال، در جلسه ی رفع اشکال پاسخ داده می شود 
%\end{center}
%\end{tcolorbox}

\begin{enumerate}
\item
برنامه ای بنویسید که فایلی از نوع 
\lr{text}
به نام 
\lr{myFile.txt}
ایجاد کرده و
عدد رندمی (بین یک تا ۱۰) را توسط سیستم ایجاد کرده و به اندازه ی آن عدد رندم 
 درون فایل مورد نظر با حلقه ای 
\lr{string}های 
 زیر را بنویسید 
\begin{latin}\inputcode{Text}{q1.txt}\end{latin}
که در اینجا
\lr{n}
عدد رندمی است.
برای بدست آوردن عدد رندم از دستور 
\lr{(numpy.random.randint)}
می توان استفاده کرد و همچنین
برای ایجاد فایل از دستور
\lr{open("myFile.txt","w+")}
استفاده می توان کرد.
\item 
\begin{enumerate}
\item
برنامه ای بنویسید که 
نام فایل درون فولدر 
\lr{question}
را دریافت کند، سپس با توجه به نام فایل، آن را باز کرده و خوانده و محتوای آن را درون لیستی قرار دهید.
\item 
در این فایل 
\lr{Username}
و 
\lr{Password}
را 
پیدا کرده
(\lr{User: , Pass:})
 و آنها را
چاپ کنید.
\item
\lr{string}
درون 
براکت را نیز چاپ نماید.

\textcolor{red}{
(از کتابخانه ی
\lr{re}
و دستور زیر 
می توان استفاده کرد.)
\begin{latin}\inputcode{Text}{q2.txt}\end{latin}
}
\end{enumerate}
\item 
برنامه ای بنویسید که
فایل
\lr{std.xml}
داده شده را بخواند.
این فایل شامل مشخصات دانشجویان می باشد. 
\begin{enumerate}
\item
 تعداد دانشجوها را اعلام نماید و 
 اسامی و مشخصات دانشجویان را در خروچی چاپ نمایید.
 \item
 \lr{tag}
 و
 \lr{attribute}
 زیرمجموعه های 
 \lr{root}
 که در اینجا
 \lr{college}
 است را بدست آورید.
 \item
 فایل 
 \lr{Out.xml}
 جدیدی ایجاد کرده و دانشجویانی که معدل آنها بیشتر از 
 $18$
 است را در آن قرار دهید. 
\end{enumerate}
\item 
برنامه ای بنویسید که
برروی
\lr{localhost:8000}
شماره ی دانشجویی را به سرور درخواست بدهد
و سرور اسم دانشجو را 
از فایل 
\lr{std.xml}
پیدا کرده و آن را در کنسول چاپ نموده و
به کاربر ارسال نماید.
(این سوال به طور کامل در جلسه ی حل تمرین توضیح داده می شود).


\end{enumerate}

%\begin{tcolorbox}
%جهت تحویل تمارین هر تمرین را داخل یک فلدر بریزید که با شماره تمرین نام گذاری شده است و 
% گزارش کار را به فرمت یک فایل
%\lr{PDF}
%در فلدراصلی قرار دهید. فلدر اصلی را بعد از فشرده سازی به صورت 
%\lr{HW1-\rl{شماره دانشجویی}.\textbf{zip}}
% نام گزاری، و ارسال کنید.
%\begin{center}
%\textbf{
%مهلت تحویل: تا ساعت 12 ظهر دوشنبه سیزدهم اردیبهشت ماه 1400.
%}
%\end{center}
%\end{tcolorbox}
\end{document}
