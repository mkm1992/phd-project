\documentclass[oneside]{article}

\usepackage{siunitx}
\usepackage{enumerate}
\usepackage{fancyhdr}
\usepackage{minted}
\usepackage{lastpage}
\usepackage{xcolor}
\usepackage{tcolorbox}
\usepackage{amsmath}
\usepackage[colorlinks=true]{hyperref}
\usepackage{setspace}
\usepackage[absolute]{textpos}
\usepackage{xepersian} % Always last package to load
\settextfont{B Nazanin}
\setlatintextfont{Times New Roman}
%\setlatinmonofont[Scale=0.8]{Monaco}
\defpersianfont\nastaliqfont{IranNastaliq}
\setlength{\TPHorizModule}{1cm}
\setlength{\TPVertModule}{1cm}
\linespread{1.5}
\pagestyle{fancy}
\renewcommand{\headrulewidth}{0pt}
\newcommand*{\fancypagenumber}{%
\fancyfoot[C]{صفحه
\thepage
از
\pageref{LastPage}}
}
\fancypagenumber
\fancypagestyle{plain}{\fancypagenumber}
% insert syntax highlighted code from a file
\newcommand{\inputcode}[2]{\inputminted[mathescape,%
                                                 linenos=true,%
                                                 formatcom=\small\setstretch{1}]{#1}{#2}}%
%\renewcommand{\theFancyVerbLine}{\sffamily\scriptsize
%\textcolor[rgb]{0.5,0.5,1.0}{\oldstylenums{\arabic{FancyVerbLine}}}}
\title{تمرین های سری اول درس
\lr{Service Based Architecture}
}
%\author{دکتر شفیعی}
\date{\vspace{-5ex}}
\begin{document}
\maketitle

\begin{textblock}{5}(6.5,2)\nastaliqfont
\noindent\Large
بسم الله الرحمن الرحیم
\end{textblock}
%\begin{tcolorbox}
%\begin{center}
%این اولین سری از تمارین مربوطه به پایتون می باشد. در صورت داشتن سوال، در جلسه ی رفع اشکال پاسخ داده می شود 
%\end{center}
%\end{tcolorbox}

\begin{enumerate}
\item
برنامه ای بنویسید که سه نمره ی دانشجو را از کاربر بگیرد.
\\
اعداد صحیح یا اعشاری می توانند باشند .حاصل جمع، حاصل ضرب و میانگین اعداد را چاپ کند. به این صورت:
\\
\begin{latin}
\texttt{Please enter 3 integers: 10 13 12.5} \\
\texttt{Sum is: 35.5 }\\
\texttt{Product is: 16.25 }\\
\texttt{Average is: 11.833}
\end{latin}
همچنین در صورتی که میانگین نمره ها از $18$ بیشتر باشد، در خروجی مقدار \lr{A}، میانگین بین $16$ تا 
$18$
بود مقدار \lr{B}
و در غیر اینصورت مقدار
\lr{C}
را در خروجی چاپ نماید.
\item
\begin{enumerate}
\item 
یک اسکریپت پایتون با دیکشنری بنویسید که کلیدها اعداد بین 1 تا 15 باشند و مقادیر مربع کلیدها هستند
$(1:1, 2:4, 3:9, ...)$
.
\item
از کاربر نام و معدل 10 دانشجو را بگیرید و در دیکشنری (کلید : نام، مقدار : معدل) قرار دهید.
\end{enumerate}
\item 
برنامه ای بنویسید که یک کلمه یا جمله
از کاربر بگیرد و در صورتی که این
کلمه با مقدار عکس خود یکی باشد، مقدار 
\lr{true}
و در غیر این صورت مقدار
\lr{false}
را چاپ نماید
(
برای مثال 
\lr{hi ih}
مقدار 
\lr{true}
و 
\lr{tree}
مقدار
\lr{false}
است.
)

\item
	برنامه ای بنویسید که از کاربر عدد بگیرد تا زمانی که کاربر عدد صفر را وارد کند. بعد به روش
\lr{bubble sort}
  این اعداد را مرتب کنید و در خروجی چاپ کنید. 
  \\
راهنمایی:
\\
روش
\lr{bubble sort}
یک الگوریتم مرتب‌سازی ساده‌است که فهرست را پشت سرهم پیمایش می‌کند تا هر بار عناصر کنارهم را با هم سنجیده و اگر در جای نادرست بودند جابه‌جایشان کند. دراین الگوریتم این کار باید تا زمانی که هیچ جابه‌جایی در فهرست رخ ندهد، ادامه یابد و در آن زمان فهرست مرتب شده‌است.

\item
در شاخه ای که فایل پایتون شما وجود دارد یک فلدر به نام
\lr{root}
بسازید. هدف این است که داخل این فلدر تعدادی فایل و فلدر دیگر بسازیم به این صورت:
\begin{latin}\inputcode{Text}{random_folder_structure.txt}\end{latin}
این فلدر تا 5 مرحله شاخه بندی شده است. داخل هر فلدر تعدادی فایل متنی وجود دارد که با 
\lr{\texttt{.txt}}
داخل شکل نشان داده شده اند. اسم هر کدام از این فایل ها یک عدد تصادفی بین 0 تا 100 می باشد. توجه کنید که تعداد این فایل ها نیز تصادفی می باشد. 
\begin{enumerate}
\item
برنامه مورد نظر را طوری بنویسید که با هر بار اجرا کردن فلدر ها و فایل های قبلی پاک شوند (در صورت وجود) و دوباره ایجاد شوند.
\item
در قسمت دوم همین برنامه با استفاده از تابع
\lr{\texttt{os.walk}}
ابتدا تعداد کل فایل های ایجاد شده را بشمارید و سپس  تمامی  فایل هایی که داخل  نام  فایل عدد صفر را دارند به جای عدد صفر از کاراکتر
\lr{\texttt{'\_'}}
استفاده کنید 
(\lr{rename}).
\end{enumerate}
راهنمایی: در این تمرین نیاز به استفاده از تعدادی از توابع داخل کتابخانه 
\lr{\texttt{os}}
از جمله
\lr{\texttt{os.path.exists}}
،
\lr{\texttt{mkdir}}
و 
\lr{\texttt{rename}}
دارید. همچنین از توابع کتابخانه
\lr{\texttt{shutil}}
مثل
\lr{\texttt{rmtree}}
نیز می توانید استفاده کنید. با توابع رشته ای هم می توانید جهت عوض کردن اسم فایل کار کنید.
\item
\begin{enumerate}
\item
 \textcolor{red}{امتیازی-}
یک فایل متنی درست کنید به این ترتیب که متشکل از 
\num{1000000}
 خط باشد که از شماره خط صفر شروع می شود. هر خط این فایل به این شیوه نوشته شده باشد:
\begin{latin}\inputcode{Text}{random.txt}\end{latin}
برای تولید عدد تصادفی بین صفر و یک از تابع 
\lr{\texttt{random}}
داخل کتابخانه
\lr{\texttt{numpy.random}}
استفاده کنید و با 6 رقم بعد از ممیز عدد را نشان دهید. توجه کنید که در این فایل بعد از کلمه
\lr{Line}
از فاصله استفاده شده و بعد از شماره خط از 
\lr{TAB}
استفاده می شود. زمان اجرای کار را ذخیره کنید و نشان دهید. 
\item
\textcolor{red}{امتیازی-}
سپس همین کار را دقیقا توسط کتابخانه 
\lr{\texttt{csv}}
انجام دهید و زمان را اندازه بگیرید و درصد کاهش و یا افزایش را بیان کنید. به نظر شما علت این کاهش یا افزایش چه چیزی می تواند باشد؟ (همه جا از فاصله عادی یا
\lr{TAB})
استفاده کنید.
\item
\textcolor{red}{امتیازی-}
در قسمت سوم این سوال اگر کل داده ها را داخل یک لیست در ابتدا ذخیره کنید و سپس داخل فایل بریزید آیا سریع تر خواهد بود؟ در صورت مثبت و منفی بودن جواب علت را توضیح دهید؟
\end{enumerate}
در کل یک فایل پایتون خواهیم داشت که از سه قسمت تشکیل شده است و در هر قسمت زمان اجرا ذخیره می شود و نشان داده می شود.

راهنمایی: برای قسمت دوم سوال بایستی از تابع 
\lr{\texttt{csv.writer}}
استفاده کنید که ورودی آن 
\lr{File Handle}
می باشد و خروجی آن 
\lr{CSV Handle}
خواهد بود. متغیر اخیر یک متد به نام 
\texttt{writerow}
دارد که یک لیست را ورودی می گیرد و در خروجی چاپ می کند. در قسمت سوم سوال از
\lr{List Comprehension}
استفاده کنید. کل عملیات باز کردن فایل و ریختن داده ها در دو خط کد قابل انجام خواهد بود. از تابع 
\lr{\texttt{writelines}}
استفاده بفرمایید.
\end{enumerate}

%\begin{tcolorbox}
%جهت تحویل تمارین هر تمرین را داخل یک فلدر بریزید که با شماره تمرین نام گذاری شده است و 
% گزارش کار را به فرمت یک فایل
%\lr{PDF}
%در فلدراصلی قرار دهید. فلدر اصلی را بعد از فشرده سازی به صورت 
%\lr{HW1-\rl{شماره دانشجویی}.\textbf{zip}}
% نام گزاری، و ارسال کنید.
%\begin{center}
%\textbf{
%مهلت تحویل: تا ساعت 12 ظهر دوشنبه سیزدهم اردیبهشت ماه 1400.
%}
%\end{center}
%\end{tcolorbox}
\end{document}
