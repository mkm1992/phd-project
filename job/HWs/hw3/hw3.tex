\documentclass[oneside]{article}
\usepackage{xcolor}
\usepackage{siunitx}
\usepackage{enumerate}
\usepackage{fancyhdr}
\usepackage{minted}
\usepackage{lastpage}
\usepackage{xcolor}
\usepackage{tcolorbox}
\usepackage{amsmath}
\usepackage[colorlinks=true]{hyperref}
\usepackage{setspace}
\usepackage[absolute]{textpos}
\usepackage{xepersian} % Always last package to load
\settextfont{B Nazanin}
\setlatintextfont{Times New Roman}
%\setlatinmonofont[Scale=0.8]{Monaco}
\defpersianfont\nastaliqfont{IranNastaliq}
\setlength{\TPHorizModule}{1cm}
\setlength{\TPVertModule}{1cm}
\linespread{1.5}
\pagestyle{fancy}
\renewcommand{\headrulewidth}{0pt}
\newcommand*{\fancypagenumber}{%
\fancyfoot[C]{صفحه
\thepage
از
\pageref{LastPage}}
}
\fancypagenumber
\fancypagestyle{plain}{\fancypagenumber}
% insert syntax highlighted code from a file
\newcommand{\inputcode}[2]{\inputminted[mathescape,%
                                                 linenos=true,%
                                                 formatcom=\small\setstretch{1}]{#1}{#2}}%
%\renewcommand{\theFancyVerbLine}{\sffamily\scriptsize
%\textcolor[rgb]{0.5,0.5,1.0}{\oldstylenums{\arabic{FancyVerbLine}}}}
\title{تمرین های سری سوم درس
\lr{Service Based Architecture}
}
%\author{دکتر شفیعی}
\date{\vspace{-5ex}}
\begin{document}
\maketitle

\begin{textblock}{5}(6.5,2)\nastaliqfont
\noindent\Large
بسم الله الرحمن الرحیم
\end{textblock}
%\begin{tcolorbox}
%\begin{center}
%این اولین سری از تمارین مربوطه به پایتون می باشد. در صورت داشتن سوال، در جلسه ی رفع اشکال پاسخ داده می شود 
%\end{center}
%\end{tcolorbox}

\begin{enumerate}
\item 
برنامه ای بنویسید که
با استفاده از کتابخانه ی
\lr{SOAPpy}
یک سرور
 \lr{SOAP}
برروی 
\lr{localhost:8080}
قرار دهید.
این برنامه می بایست شامل دو فایل 
\lr{server}
و 
\lr{client}
باشد.
در این برنامه فایل 
\lr{std.xml}
که در تمرین قبلی نیز بود قرار گرفته است.
\begin{enumerate}
\item 
در این برنامه بر روی سرور تابعی بنویسید که استاد درس ریاضی (با وارد کردن کلمه ی 
\lr{"teacher"}
) بتواند نمره ی درس ریاضی را با ورود شماره دانشجویی و نمره ی فرد، در فایل 
\lr{xml}
قرار دهد.
\item
در بخش دوم تابعی بنویسید که دانشجو(با وارد کردن کلمه 
\lr{"student"}
) با ورود شماره دانشجویی خود، بتواند نمره ی خود را برروی کنسول ببیند.
\item
در بخش سوم تابعی بنویسید که مدیرگروه (با وارد کردن کلمه
\lr{"admin"}
)
بتواند لیست نمره ی همه ی دانشجوها و معدل نمرات کلاس را ببیند.
\end{enumerate}
برای نصب این کتابخانه بر روی پایتون ۲و پایتون
3
به ترتیب از دستور های داده شده استفاده کنید و ترجیحا برروی پایتون ۲ نصب نمایید.
\begin{latin}\inputcode{Text}{m.txt}\end{latin}
\item
برنامه ای بنویسید که 
برروی 
\lr{localhost:8080}
سروری را قرار دهد و از متدهای 
\lr{get}
و
\lr{post}
استفاده گردد.
فایل 
\lr{city.json}
را خوانده و
در متد
\lr{get}
با وارد کردن نام شهر،
دمای هوا را بدهد و در متد 
\lr{post}
با وارد کردن نام شهر، 
در خروجی نام شهر و دمای هوا را به صورت  
\lr{json}
بدهد.



\end{enumerate}

%\begin{tcolorbox}
%جهت تحویل تمارین هر تمرین را داخل یک فلدر بریزید که با شماره تمرین نام گذاری شده است و 
% گزارش کار را به فرمت یک فایل
%\lr{PDF}
%در فلدراصلی قرار دهید. فلدر اصلی را بعد از فشرده سازی به صورت 
%\lr{HW1-\rl{شماره دانشجویی}.\textbf{zip}}
% نام گزاری، و ارسال کنید.
%\begin{center}
%\textbf{
%مهلت تحویل: تا ساعت 12 ظهر دوشنبه سیزدهم اردیبهشت ماه 1400.
%}
%\end{center}
%\end{tcolorbox}
\end{document}
