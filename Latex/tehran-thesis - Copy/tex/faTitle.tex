% !TeX root=../main.tex
% در این فایل، عنوان پایان‌نامه، مشخصات خود، متن تقدیمی‌، ستایش، سپاس‌گزاری و چکیده پایان‌نامه را به فارسی، وارد کنید.
% توجه داشته باشید که جدول حاوی مشخصات پروژه/پایان‌نامه/رساله و همچنین، مشخصات داخل آن، به طور خودکار، درج می‌شود.
%%%%%%%%%%%%%%%%%%%%%%%%%%%%%%%%%%%%
% دانشگاه خود را وارد کنید
\university{دانشگاه تهران}
% پردیس دانشگاهی خود را اگر نیاز است وارد کنید (مثال: فنی، علوم پایه، علوم انسانی و ...)
\college{پردیس دانشکده‌های فنی}
% دانشکده، آموزشکده و یا پژوهشکده  خود را وارد کنید
\faculty{دانشکدهٔ مهندسی برق و کامپیوتر}
% گروه آموزشی خود را وارد کنید (در صورت نیاز)
\department{گروه شبکه}
% رشته تحصیلی خود را وارد کنید
\subject{مهندسی برق}
% گرایش خود را وارد کنید
\field{مخابرات سیستم}
% عنوان پایان‌نامه را وارد کنید
\title{تخصیص منابع در شبکه‌های دسترسی رادیویی باز با برش‌دهی شبکه }
% نام استاد(ان) راهنما را وارد کنید
\firstsupervisor{دکتر شاه منصوری}
\firstsupervisorrank{دانشیار}
%\secondsupervisor{دکتر راهنمای دوم}
%\secondsupervisorrank{استادیار}
% نام استاد(دان) مشاور را وارد کنید. چنانچه استاد مشاور ندارید، دستورات پایین را غیرفعال کنید.
%\firstadvisor{دکتر مشاور اول}
%\firstadvisorrank{استادیار}
%\secondadvisor{دکتر مشاور دوم}
% نام داوران داخلی و خارجی خود را وارد نمایید.
\internaljudge{دکتر داور داخلی}
\internaljudgerank{دانشیار}
\externaljudge{دکتر داور خارجی}
\externaljudgerank{دانشیار}
\externaljudgeuniversity{دانشگاه داور خارجی}
% نام نماینده کمیته تحصیلات تکمیلی در دانشکده \ گروه
\graduatedeputy{دکتر نماینده}
\graduatedeputyrank{دانشیار}
% نام دانشجو را وارد کنید
\name{مژده}
% نام خانوادگی دانشجو را وارد کنید
\surname{کربلایی مطلب}
% شماره دانشجویی دانشجو را وارد کنید
\studentID{810196074}
% تاریخ پایان‌نامه را وارد کنید
\thesisdate{مهر ۱۳۹۹}
% به صورت پیش‌فرض برای پایان‌نامه‌های کارشناسی تا دکترا به ترتیب از عبارات «پروژه»، «پایان‌نامه» و «رساله» استفاده می‌شود؛ اگر  نمی‌پسندید هر عنوانی را که مایلید در دستور زیر قرار داده و آنرا از حالت توضیح خارج کنید.
%\projectLabel{پایان‌نامه}

% به صورت پیش‌فرض برای عناوین مقاطع تحصیلی کارشناسی تا دکترا به ترتیب از عبارت «کارشناسی»، «کارشناسی ارشد» و «دکتری» استفاده می‌شود؛ اگر نمی‌پسندید هر عنوانی را که مایلید در دستور زیر قرار داده و آنرا از حالت توضیح خارج کنید.
%\degree{}
%%%%%%%%%%%%%%%%%%%%%%%%%%%%%%%%%%%%%%%%%%%%%%%%%%%%
%% پایان‌نامه خود را تقدیم کنید! %%
\dedication
{
{\Large تقدیم به:}\\
\begin{flushleft}{
	\huge
	پدر و مادرم\\
%	\vspace{7mm}
%	و\\
%	\vspace{7mm}
%	پدر و مادرم
}
\end{flushleft}
}
%% متن قدردانی %%
%% ترجیحا با توجه به ذوق و سلیقه خود متن قدردانی را تغییر دهید.
\acknowledgement{
سپاس خداوندگار حکیم را که با لطف بی‌کران خود، آدمی را به زیور عقل آراست.

در آغاز وظیفه‌  خود  می‌دانم از زحمات بی‌دریغ اساتید  راهنمای خود،  جناب آقای دکتر ... و ...، صمیمانه تشکر و  قدردانی کنم که در طول انجام این پایان‌نامه با نهایت صبوری همواره راهنما و مشوق من بودند و قطعاً بدون راهنمایی‌های ارزنده‌ ایشان، این مجموعه به انجام نمی‌رسید.

از جناب آقای دکتر ... که  زحمت مشاوره‌، بازبینی و تصحیح این پایان‌نامه را تقبل فرمودند کمال امتنان را دارم.

%از همکاری و مساعدت‌های دکتر ... مسئول تحصیلات تکمیلی و سایر کارکنان دانشکده بویژه سرکار خانم ... کمال تشکر را دارم.

با سپاس بی‌دریغ خدمت دوستان گران‌مایه‌ام، خانم‌ها ... و آقایان ... در آزمایشگاه ...، که با همفکری مرا صمیمانه و مشفقانه یاری داده‌اند.

و در پایان، بوسه می‌زنم بر دستان خداوندگاران مهر و مهربانی، پدر و مادر عزیزم و بعد از خدا، ستایش می‌کنم وجود مقدس‌شان را و تشکر می‌کنم از خانواده عزیزم به پاس عاطفه سرشار و گرمای امیدبخش وجودشان، که بهترین پشتیبان من بودند.
}
%%%%%%%%%%%%%%%%%%%%%%%%%%%%%%%%%%%%
%چکیده پایان‌نامه را وارد کنید
\fa-abstract{
در این پروپزال، ساختار رادیویی دسترسی باز (ORAN) در نسل پنجم معرفی می‌شود و تخصیص منابع در آن در نظر گرفته می‌شود.
شبکه دسترسی رادیویی باز از ترکیب C-RAN و xRAN بدست آمده‌است. معماری \lr{ORAN} برای ایجاد زیرساختهای \lr{RAN} نسل بعدی طراحی شده است.
معماری \lr{ORAN} با تکیه بر اصول هوشمندی و باز بودن، پایه و اساس ساخت \lr{RAN} مجازی بر روی سخت افزار آزاد، با کنترل رادیویی ایجاد شده توسط هوش مصنوعی است که توسط اپراتورهای سراسر جهان پیش بینی شده است.
\lr{ORAN}،
المانهای شبکه ی دسترسی رادیویی را مجازی می‌کند، آنها را جدا کرده و رابطهای باز مناسب را 
برای اتصال این عناصر
تعیین می‌کند. همچنین، 
\lr{ORAN}
از روشهای یادگیری ماشین برای هوشمندسازی لایه‌های 
\lr{RAN}
استفاده می‌نماید.

در اینجا، مسئله‌ی برش شبکه در بخش رادیویی و قرارگیری توابع مجازی شبکه برروی مراکز داده باهم در شبکه‌ی دسترسی رادیویی باز مورد بررسی قرار گرفته است. 
بخش رادیویی
 مدل سازی شده و تاخیر و نرخ و پارامترهای دیگر بدست می آید. 
در این شبکه سرویسهای مختلف در نظر گرفته شده که شامل تعدادی کاربر است که تقاضای استفاده از آن سرویس را دارد. همچنین تعدادی برش شبکه فرض شده است که شامل منابع فیزیکی، واجد رادیویی و واحد توزیع شده و مرکزی می‌باشد. واحد توزیع شده و مرکزی نیز شامل توابع شبکه‌ی مجازی هستند که پردازشها را انجام می‌دهند.
فرض براین است که کاربران بر اساس سرویس مورد نیاز، دسته بندی می شوند و هدف تخصیص برشهای شبکه به سرویسهاست و سپس تخصیص منابع فیزیکی محاسباتی به این برشهای اختصاص یافته به سرویسها می باشد.
 
برای حل این مسئله، ابتدا مسئله را به دو مسئله‌ی کوچکتر مختلف شکسته که در بخش اول، تخصیص برش شبکه به کاربران سرویسها و تخصیص توان در ساختار رادیویی باز حل شده و پس از آن، برشهایی از شبکه که به سرویس اختصاص داده شده را به مراکز داده نگاشت می‌دهیم.
در این مسئله، تاخیر و نرخ هر کاربر در سرویس مورد بررسی قرار گرفته شده و چالش تخصیص منابع که شامل برش بخش رادیویی به هر سرویس است و جاگیری توابع شبکه حل می‌شود.
جواب بهینه باداستفاده از نرم‌افزار MOSEK و CVX در متلب بدست می‌آید. همچنین روش ابتکاری، برای حالت متمرکز، در نظر گرفته شده است که مسئله‌ی اول شامل مسئله‌ی بسته‌بندی جعبه و تخصیص توان است که چون یک مسئله‌ی NP-Hard می‌باشد در دو بخش به صورت تکراری برای تخصیص سرویس به برش و بدست آوردن توان حل می‌شود که بخش تخصیص توان به یک مسئله‌ی محدب تبدیل می‌شود و مسئله‌ی دوم نیز یک مسئله‌ی سه بعدی بسته بندی جعبه است که حل دو بخش بسته بندی جعبه، بر اساس مرتب کردن بسته‌ها به ترتیب با استفاده از اندازه‌گیری بر مبنای پارامترهای آن \cite{3dbin}، بدست می‌آید.
سپس مسئله به صورت ساده‌تر به دو مسئله‌ی بسته‌بندی جعبه و کوله‌پشتی نوشته شده و برای حالت دینامیکی متغیر با زمان با روش یادگیری تقویتی حل می‌شود.
}
% کلمات کلیدی پایان‌نامه را وارد کنید
\keywords{ تخصیص برش شبکه ،شبکه‌ی دسترسی رادیویی باز، توابع مجازی شبکه  }
% انتهای وارد کردن فیلد‌ها
%%%%%%%%%%%%%%%%%%%%%%%%%%%%%%%%%%%%%%%%%%%%%%%%%%%%%%
