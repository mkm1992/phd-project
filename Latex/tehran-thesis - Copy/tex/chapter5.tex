\chapter{پیشنهادات و کارهای آتی}
\section{مقدمه}
در فصل اول، مقدمه‌ای بر مفاهیم مورد استفاده را بیان کردیم و در مورد نسل پنجم مخابرات و مفاهیم آن صحبت نمودیم. سپس در فصل دوم مروری بر کارهای انجام شده کردیم و مقالات مرتبط با برش شبکه و شبکه‌های دسترسی باز و قرارگیری توابع مجازی شبکه را بیان نمودیم تا مروری بر چالشهای مطرح شده نسل پنجم مخابرات کرده و حل این چالشها را مورد بررسی قرار دادیم . در فصل سوم صورت مسئله‌ای در زمینه‌ی برش شبکه در شبکه‌های دسترسی باز، معرفی کرده و با روش ابتکاری، آن را حل نمودیم و نتایج را با مقدار بهینه مقایسه کردیم.
در  فصل چهارم، دو مسئله‌ی بیان شده در فصل سوم را به صورت کاملا ساده با روش یادگیری عمیق تقویتی به صورت دینامیکی و در هر بازه‌ی زمان حل نمودیم. این دو مسئله، MDP \LTRfootnote{Markov Decision Processs}
بوده و قابل حل با این روش هستند. 
حال در این فصل در مورد مزایا و معایب کارهای انجام شده در فصل سوم و چهارم صحبت کرده و کارهای آتی و پیشنهادات را بیان می‌کنیم.
\section{نتیجه‌گیری}
در اینجا، مسئله‌ی برش شبکه در بخش رادیویی و قرارگیری توابع مجازی شبکه برروی مراکز داده باهم مورد بررسی قرار گرفته شد.
برای حل این مسئله، ابتدا مسئله به دو بخش مختلف شکسته شد که در بخش اول، تخصیص برش شبکه به کاربران سرویسها و تخصیص توان حل شده و پس از آن، برشهایی از شبکه که به سرویس اختصاص داده شده را به مراکز داده نگاشت می‌دهیم.
در این مسئله، تاخیر و نرخ هر کاربر در سرویس مورد بررسی قرار گرفته شده و چالش تخصیص منابع که شامل برش بخش رادیویی به هر سرویس است و جاگیری توابع شبکه حل می‌شود.
 الگوریتم ارائه شده سرعت بسیار بیشتری از الگوریتم بهینه که با MOSEK و CVX بدست می‌آید، دارد.
 سپس مسئله به صورت ساده‌تر برای حالت دینامیکی با روش یادگیری تقویتی حل گردیده است. 
 \subsection{مزایای این چالش و حل آن}
 در مسئله‌ی بیان شده‌ی فصل سوم، مدل سیستم به صورت دقیق بیان شده و نرخ کاربر، ظرفیت لینک fronthaul و تاخیر به طور دقیق مورد بررسی قرار گرفته شده است. همچنین مسئله به واقعیت نزدیکی زیادی دارد. همچنین الگوریتم ابتکاری تعریف شده در فصل سوم برای حالتی که تداخل به نسبت کم باشد به حالت بهینه بسیار نزدیک است. در فصل چهارم همین مسئله با فرض اینکه سرویسها نیازمند تاخیر کم یا نرخ بالا هستند به صورت پارامتریک در هر لحظه از زمان حل می‌گردند. 
 در بخش بعدی چالشهای قرارگیری توابع مجازی برروی مراکز داده به طور دقیق بررسی شده و ‌در فصل چهارم این مسئله به صورت دینامیکی در هر لحظه حل گردیده است. در حل مسئله در حالت دینامیکی سعی براین است که مراکز داده کمترنی انرژی را مصرف نموده و از هدر رفت انرژی بپرهیزیم.
 \subsection{معایب  پروژه انجام شده}
 در فصل سوم از الگوریتم ابتکاری در این کار استفاده شده است. زمانی که تعداد بلوکهای منابع فیزیکی به نسبت کاربران بسیار کم باشد و تداخل به شدت زیاد گردد، الگوریتم مسئله‌ی اول به خوبی قادر به پاسخ‌گویی نیست و از حالت بهینه فاصله ‌می‌گردد. در مسئله‌ی دوم، زمانی که تعداد مراکز داده زیاد گردد فاصله‌ی حالت بهینه از الگوریتم ابتکاری زیاد شده است. 
همچنین در فصل چهارم صورت مسئله بسیار ساده‌تر از واقعیت است و مسئله در حالت دینامیکی برای تعداد درخواست کم در این حالت حل گردیده است.
 \subsection{نوآوری‌های این پروژه}
 در این پروژه، تخصیص توان و برش شبکه در شبکه‌های دسترسی باز مورد بررسی قرار گرفته است.
 ما مسئله‌ی اختصاص UE به خدمات، خدمات به برش‌ها و منابع فیزیکی بی سیم و همچنین مرکز داده به برش‌ها را به عنوان یک مشکل بهینه سازی فرمول‌بندی کرده‌ایم. سپس با ارائه‌ی روشهای ابتکاری، به حل آنها پرداختیم. در نهایت مسئله‌ی ساده شده را در حالت دینامیکی و متغیر با زمان حل کردیم.
 \section{پیشنهادات}
 
 در این بخش، پیشنهادات و کارهای آتی را بیان خواهیم کرد.
 \begin{itemize}
\item
 یکی از کارهای آتی، مدل کردن برش شبکه در ساختار شبکه‌ی دسترسی رادیویی باز و حل آن بوسیله‌ی روش یادگیری تقویتی عمیق می‌باشد. در فصل چهارم از این روش برای سیستم ساده شده استفاده گردیده و به دلیل کم بودن تعداد حالات با استفاده از روش یادگیری تقویتی حل شده و در فصل سوم نیز مدل سیستم بیان شده، یکی از کارهای بعدی این است که سیستم مدل فصل سوم را به سیستمهای رادیویی باز نزدیکتر کرده و
با روش یادگیری تقویتی عمیق حل نماییم. که در اینجا، بدست آوردن توان و ارتباط برش با سرویس از این روش بدست خواهد آمد. همچنین مقایسه‌ی روش یادگیری تقویتی عمیق و یادگیری تقویتی در اینجا نیز مورد توجه قرار خواهد گرفت.
\item 
یکی دیگر از کارهای آتی، بدست آوردن پارامترهای کیفیت سرویس QoS\LTRfootnote{Quality of Service}
در شبکه‌های دسترسی باز می‌باشد که شامل تاخیر انتها به انتها، میزان از دست دادن بسته ها\LTRfootnote{Packet Loss}،
قابلیت اطمینان و ... می‌باشد.
در اینجا می‌توان تاخیر را هم در بخش رادیویی هم در بخش هسته‌ی شبکه بدست آورد. 
همچنین،
به منظور نشان دادن نقش هوش در ORAN طرح مدیریت هوشمند منابع رادیویی را برای کنترل تراکم ترافیک و نشان دادن کارایی آن در یک مجموعه داده واقعی از یک اپراتور بزرگ بدست می‌آوریم.
\item 
شبکه تعریف شده توسط نرم افزار (SDN) و مجازی سازی عملکرد شبکه (NFV) فناوری های کلیدی امکان پذیر در شبکه های ارتباطی نسل پنجم (5G) برای قرارگیری برش های شبکه سفارشی در سطح سرویس در زیرساخت شبکه، بر اساس خواسته های منابع آماری برای تأمین کیفیت طولانی مدت خدمات (QoS) مورد نیاز می‌باشد. با این حال، بارهای ترافیکی در برش های مختلف با گذشت زمان تحت تغییر قرار می‌گیرند ، در نتیجه چالش هایی برای تأمین کیفیت کیفیت مداوم ایجاد می‌شود.
در کارهای آتی یک مشکل انتقال جریان پویا برای سرویس های متصل شده به برش شبکه، برای پاسخگویی به نیازهای تأخیر پایان انتها به انتها (E2E) با ترافیک متغیر، مورد بررسی قرار خواهد گرفت.
\item
یکی دیگر از کارهای آتی، تخصیص منابع به روش توزیع شده برای برش شبکه از منابع محاسباتی و منابع دیگر همانند پهنای باند می‌باشد.
همچنین از روش توزیع شده در لینک فراسو  
\LTRfootnote{Uplink}
برای تخصیص توان کاربران، تخصیص پهنای باند و ... استفاده می‌گردد. یکی از روشها، استفاده از 
\lr{Distributed ADMM}
می‌باشد که در این روش تعدادی عامل به صورت همکارانه سعی در حل یک معادله‌ی بهینه‌سازی مشترک دارند که تابع هدف مجموعی از مقدارهای خصوصی هر عامل ‌می‌باشد. 
 \end{itemize}
 
  