%%%%%%%%%%%%%%%%%%%%%%%%%%%%%%%%%%%%%%%%%
% Imperial College London
% UKRI Centre for Doctoral Training in AI for Healthcare
% https://ai4health.io
%
% LaTeX Template
% Version 1.0 (22/10/19) Aldo Faisal
%%%%%%%%%%%%%%%%%%%%%%%%%%%%%%%%%%%%%%%%%
%----------------------------------------------------------------------------------------
%	PACKAGES AND OTHER DOCUMENT CONFIGURATIONS
%----------------------------------------------------------------------------------------
\input{ai4health_header}
\begin{document}
\begin{titlepage}

\newcommand{\HRule}{\rule{\linewidth}{0.5mm}} % Defines a new command for the horizontal lines, change thickness here
\setlength{\topmargin}{0in}
\center % Center everything on the page
 
 
 \begin{minipage}{0.4\textwidth}
\begin{flushleft} \large
\hspace*{-0.5cm}
%\includegraphics[scale=0.14]{imperial.png}\\
\end{flushleft}
\end{minipage}
~
\begin{minipage}{0.5\textwidth}
\begin{flushright} \large
\hspace*{2cm}
%\includegraphics[scale=0.20]{logo.png}\\
\end{flushright}
\end{minipage}\\[1cm]
%----------------------------------------------------------------------------------------
%	HEADING SECTIONS
%----------------------------------------------------------------------------------------

\textsc{\Large resource allocation in open ran system using network slicing}\\[0.5cm] % Major
\LARGE \textsc{University of Tehran} \\[1.5cm] 
% Name of your heading such as course name
\textsc{\large }\\[0.5cm] % Minor heading such as course title

%----------------------------------------------------------------------------------------
%	TITLE SECTION
%----------------------------------------------------------------------------------------

\HRule \\[0.4cm]
{ \huge \bfseries Research Planning Report}\\[0.4cm] % Title of your document
\HRule \\[1cm]
 
%----------------------------------------------------------------------------------------
%	AUTHOR SECTION
%----------------------------------------------------------------------------------------

\begin{minipage}{0.4\textwidth}
\begin{flushleft} \large
\emph{Student:}\\
Mojdeh Karbalaee Motalleb \\ % Your name, if you are medically qualified please use "Last Name, M.D."
\emph{Email Address:}\\
mojdeh.karbalaee@ut.ac.ir
\end{flushleft}
\end{minipage}
~
\begin{minipage}{0.5\textwidth}
\begin{flushright} \large
\emph{Supervisor:} \\
Dr. Onel Lopez \\% Supervisor's Name
%Supervisor Department(s)\\
%[0.5cm] \emph{Healthcare Supervisor} \\
%Fist name \textsc{Last name} % Supervisor's Name
%Supervisor Department(s)  or NHS Trust+section\\
\emph{Department: } \\
Electrical Engineering
% \textsc{Add name here} % Supervisor's Name
% Supervisor Department(s) or NHS Trust+section
\end{flushright}
\end{minipage}\\[1cm]

%----------------------------------------------------------------------------------------
%	DATE SECTION
%----------------------------------------------------------------------------------------

{\large \today}\\[0.5cm] % Date, change the \today to a set date if you want to be precise

\vfill % Fill the rest of the page with whitespace

\end{titlepage}

%\renewcommand{\abstractname}{Abstract}
\begin{abstract}
%% 100 words
\emph{In this project }
\end{abstract}

\section{Introduction \& Literature review}
\emph{Give background and motivation for your work. Explain  knowledge needed. Show its scientific and healthcare need.
Define a set of clear questions and approaches to answering them. Highlight their originality and/or significance. Explain how they adds to, develop (or challenges) existing literature in the field. Note, this comprises both AI questions and Healthcare questions.}
\subsection{AI literature}
\subsection{Healthcare literature}
\section{PhD goal}
\emph{Explain the main question you want to answer in terms of AI technology developed and healthcare challenge addressed. Connect it with current state of research and show its importance in this context.
Show what you want to achieve. Explain your contribution to the subject.}

\section{Methodology}
\emph{Specify methodology and techniques which will be used for your research. Be specific this will allow the reviewers to understand how well thought out and achievable your plan is.}
\section{Summarise the work done since the start of the PhD.}
\emph{Show interim results, if any.}
\section{Program of work}
\emph{This is your list, description and schedule of activities to achieve in PhD goal . Include Publication \& conference deadlines as expected, i.e. Give an overview of subjects or steps of your work you intend to publish,  suggest some meetings or conferences you plan to attend in order to present your research.}

\section{Data, Infrastructure or Software needs}
\emph{Make a list of facilities or infrastructure, including data and software you will need to complete your research. Explain how it is or will be made available. Assess the \textbf{Data Readiness Level} of all needed data sources.}

\section{Planned Impact}
\emph{Assuming that you achieve your PhD goals, what is the impact that your proejct will achieve. Assess your project's planned impact in terms of the PICO (population, intervention, comparator and outcome) framework.}

\subsection*{Example: Diagnosis}
\paragraph{Population}: To which populations of patients would the test be applicable? How can they be best described? Are there subgroups that need to be considered?

\paragraph{Intervention} (index test[s]): The test or test strategy being evaluated.

\paragraph{Comparator} The test with which the index test(s) is/are being compared, usually the reference standard (the test that is considered to be the best available method to establish the presence or absence of the condition of interest - this may not be the one that is routinely used in practice).

Target condition: The disease, disease stage or subtype of disease that the index test(s) and the reference standard are being used to establish.

\paragraph{Outcome} The diagnostic accuracy of the test or test strategy for detecting the target condition. This is usually reported as test parameters, such as sensitivity, specificity, predictive values, likelihood ratios, or - where multiple cut-off values are used - a receiver operating characteristic (ROC) curve.

\subsection*{Example: Intervention}

\paragraph{Population} Which populations of patients are we interested in? How can they be best described? Are there subgroups that need to be considered?

\paragraph{Intervention} Which intervention, treatment or approach should be used?

\paragraph{Comparators} What is/are the main alternative(s) to compare with the intervention being considered?

\paragraph{Outcome} What is really important for the patient? Which outcomes should be considered? Examples include intermediate or short-term outcomes; mortality; morbidity and quality of life; treatment complications; adverse effects; rates of relapse; late morbidity and re-admission; return to work, physical and social functioning; resource use.


%**********************************************%
\newpage
\addcontentsline{toc}{section}{References}
\begin{thebibliography}{9}
\bibitem{latexcompanion} 
Michel Goossens, Frank Mittelbach, and Alexander Samarin. 
\textit{The \LaTeX\ Companion}. 
Addison-Wesley, Reading, Massachusetts, 1993.
 
\bibitem{einstein} 
Albert Einstein. 
\textit{Zur Elektrodynamik bewegter K{\"o}rper}. (German) 
[\textit{On the electrodynamics of moving bodies}]. 
Annalen der Physik, 322(10):891-921, 1905.
 
\bibitem{knuthwebsite} 
Knuth: Computers and Typesetting,
\\\texttt{http://www-cs-faculty.stanford.edu/\~{}uno/abcde.html}

\end{thebibliography}

\end{document}