 %\pdfoutput=1
\documentclass[conference]{IEEEtran}
\IEEEoverridecommandlockouts
% The preceding line is only needed to identify funding in the first footnote. If that is unneeded, please comment it out.
\usepackage[T1]{fontenc}
\usepackage{cite}
\usepackage{mathtools}
\usepackage{stackengine}
\def\delequal{\mathrel{\ensurestackMath{\stackon[1pt]{=}{\scriptstyle\Delta}}}}
\usepackage{amsmath,amssymb,amsfonts}
\usepackage{amsmath,epsfig,cite,amsfonts,amssymb,psfrag,subfig}
\usepackage{graphicx}
\usepackage{textcomp}
\usepackage{xcolor}
\usepackage{algorithm}
\usepackage[noend]{algpseudocode}
\usepackage{amsthm}
\def\BibTeX{{\rm B\kern-.05em{\sc i\kern-.025em b}\kern-.08em
    T\kern-.1667em\lower.7ex\hbox{E}\kern-.125emX}}
\allowdisplaybreaks
\newtheorem{remark}{Remark}
\newtheorem{theorem}{Theorem}
\newtheorem{lemma}{Lemma}
\newtheorem{proposition}{Proposition}
\newtheorem{corollary}{Corollary}
\newcommand{\diag}{\mathop{\mathrm{diag}}}
\DeclareMathOperator{\E}{\mathbb{E}}
\usepackage[margin=0.7in]{geometry}
\setlength{\columnsep}{11mm}
\begin{document}

\title{Network Slicing and Resource Allocation in an Open RAN System \vspace{-.1cm}
}
%
%\author{\IEEEauthorblockN{1\textsuperscript{st} Mojdeh Karbalaee Motalleb}
%\IEEEauthorblockA{\textit{Electrical and Computer Engineering} \\
%\textit{Tehran University}\\
%Tehran, Iran \\
%mojdeh.karbalaee@ut.ac.ir}
%\and
%\IEEEauthorblockN{2\textsuperscript{nd} Vahid Shah-Mansouri}
%\IEEEauthorblockA{\textit{Electrical and Computer Engineering} \\
%\textit{Tehran University}\\
%Tehran, Iran \\
%vmansouri@ut.ac.ir}
%\and
%\IEEEauthorblockN{3\textsuperscript{rd} Salar Nouri Naghadeh}
%\IEEEauthorblockA{\textit{Electrical and Computer Engineering} \\
%\textit{Tehran University}\\
%Tehran, Iran \\
%salar.nouri@ut.ac.ir}
%}
  \author{
    \IEEEauthorblockN{Mojdeh Karbalaee Motalleb}
    \IEEEauthorblockA{School of ECE, College of Engineering, University of Tehran, Iran \\
    Email: \{mojdeh.karbalaee\}@ut.ac.ir,
    \vspace{-.2cm}
  }
  }

\maketitle

\begin{abstract}

\end{abstract}

\begin{IEEEkeywords}

\end{IEEEkeywords}

\section{Introduction}


\section{System Model and Problem Formulation}\label{systemmodel}

In this section, first, we  present the system model. Then, we obtain achievable data rates and delays for the downlink (DL) of the ORAN system. Afterward, we discuss about assignment of physical data center resources.
Finally, the main problem is expressed.
\subsection{System Model}
Suppose we have two service types includes eMBB and URLLC. Assume we have $V_1$ and $V_2$ different applications for the first and second service type, respectively ($V = V_1 + V_2$).
Assume we have $S$ preallocated slices serving $V$ services; There are $S_1$ slices for the first service type (eMBB) and $S_2$ slices for the second service type (URLLC) ($S = S_1 + S_2$).
Each Service $v_j\in \{1,2,...,V_j \} $ consists of $U_{v_j}$ request from the 
single-antenna UEs which require certain QoS to be able to use the requested program($j \in \{1,2\}$ indicate service type).
There are different application request which fall into one of these service categories. Each application request requires specific QoS. Based on the request for the application and QoS, UE may be admitted and allocated to slice.
Each slice $s_j \in \{1,2,...,S_j \}$, $j \in \{1,2\}$ consists of $K_{s_j}$, $j \in \{1,2\}$ preallocated virtual resource blocks that are mapped to Physical Resource Blocks (PRBs), $M_{s,1}$ VNFs for the processing of O-DU,  $M_{s,2}$ VNFs for the processing of O-CU-UP and $M_{u}$ VNFs for the processing of UPF.

Also, each VNF instance is running on the virtual machine (VM) that are using resources from the data centers. Each VM, requires enough resources of CPU, memory, storage and network bandwidth.

In addition, there are $R$ multi-antenna RU that are shared between slices. Each RU $r \in \{1,2,...,R \}$
has $J$ antenna for transmitting and receiving data. Moreover, all RUs, have access to PRBs.
\subsection{The Achievable Rate}
The SNR of $i^{th}$ UE in $v^{th}$ service experienced at slice $s$ on PRB $k$ which is obtained from
 $\rho_{u(v,i)}^{k,s}$ is the SNR 
\begin{equation}\label{eq2}
\rho_{u(v,i)}^{k,s} =  \frac{p_{u(v,i)}^{k,s}\sum_{r=1}^{R}|{\bold{h}_{r,u(v,i)}^{H \: k,s}} \bold{w}_{r,u(v,i)}^{k,s} g_{r,u(v,i)}|^2}{BN_0 + I_{u(v,i)}^{k,s}},
\end{equation} 
where $p_{u(v,i)}^{k,s}$ represents the transmission power allocated to $i^{th}$ UE in $v^{th}$ service, served at slice $s$ on PRB $k$. 
${\bold{h}_{r,u(v,i)}^{k,s}} \in \mathbb{C}^{J}$ is the vector of channel gain of a wireless link from 
$r^{th}$ RU to the $i^{th}$ UE in $v^{th}$ service. In addition, $\bold{w}_{r,u(v,i)}^{k,s} \in \mathbb{C}^{J}$ depicts the  transmit beamforming vector from $r^{th}$ RU to the $i^{th}$ UE in $v^{th}$ service that is the zero forcing beamforming vector to minimize the interference which is indicated as below
\begin{equation}
\bold{w}_{r,u(v,i)}^{k,s} = {\bold{h}_{r,u(v,i)}^{k,s}}({\bold{h}_{r,u(v,i)}^{H \: k,s1}} {\bold{h}_{r,u(v,i)}^{k,s}})^{-1}
\end{equation}
Moreover, $g_{r,u(v,i)} \in \{0,1\}$ is a binary variable that illustrates whether RU $r$ is mapped to the $i^{th}$ UE in $v_{1}^{th}$ service or not. 
Also, $BN_0$ denotes the power of Gaussian additive noise, and $I_{u(v,i)}^{k,s1}$ is the power of interfering signals represented as follow
\begin{equation}
\begin{split}
I_{u(v,i)}^{k,s} &=
 \underbrace{\sum_{\substack{l=1 \\ l\neq i}}^{{U}_{v}} \gamma_{1}  p_{u(v,i)}^{k,s}\sum_{r=1}^{R}|{\bold{h}_{r,u(v,i)}^{H \: k,s}} \bold{w}_{r,u(v,i)}^{k,s1} g_{r,u(v,i)}|^2}_{\text{(intra-service interference)}}\\
&+ \underbrace{\sum_{\substack{y=1 \\ l\neq v}}^{V}\sum_{n=1}^{S}\sum_{l=1}^{{U}_y} \gamma_{2}  p_{u(y,l)}^{k,n}\sum_{r=1}^{R}|{\bold{h}_{r,u(v,i)}^{H \: k,s}} \bold{w}_{r,u(y,l)}^{k,n} g_{r,u(y,l)}|^2}_{\text{(inter-service interference)}}\\
\end{split}
\end{equation}
where $\gamma_{1} = e^{s}_{k,u(v,i)}e^{s}_{k,u(v,l)}a_{u(v,i)}a_{u(v,l)} b_{v, s} $
and $\gamma_{2} = e^{s}_{k,u(v,i)}e^{n}_{k,u(y,l)}a_{u(v,i)}a_{u(y,l)} b_{v, s} b_{y,n} $.
Where $a_{u(v,i)} \in \{0,1\}$ is a binary variable to depict user admission.
$b_{v, s}$ is a binary variable that illustrates whether slice $s$ is allocated to service $v$ or not. 
$e^{s}_{k,u(v,i)}$ is the binary variable to show whether the UE $i$ in service $v$ using slice $s$, is assigned to $k^{th}$ PRB or not.

The achievable data rate for the $i^{th}$ UE request in the $v_{1}^{th}$ application of service type 1 (eMBB) can be written as
\begin{equation}\label{eq1}
\mathcal{R}_{u(v_1,i)}^{e} = \sum_{s_1=1}^{S_1}\sum_{k=1}^{K_s} B \log_2({1+ \rho_{u(v_1,i)}^{k,s_1}})a_{u(v_1,i)} b_{v_1, s_1} e^{s_1}_{k,u(v_1,i)},
\end{equation}
where $B$ is the bandwidth of system. 

Since the blocklength in URLLC is finite, the achievable data rate for the $i^{th}$ UE request in the $v_{2}^{th}$ application of service type 2 (URLLC) is not achieved from Shannon Capacity formula. So, for the short packet transmission the achievable data rate is approximated from follow
\begin{equation}\label{eq1}
\mathcal{R}_{u(v_2,i)}^{u} = \sum_{s_2=1}^{S_2}\sum_{k=1}^{K_s} B (\log_2({1+ \rho_{u(v_2,i)}^{k,s_2}})- \zeta_{u(v_2,i)}^{k,s_2}){\beta}_{u(v_2,i)}^{k,s_2}
\end{equation}
Where ${\beta}_{u(v_2,i)}^{k,s_2}=a_{u(v_2,i)} b_{v_2, s_1} e^{s_2}_{k,u(v_2,i)}$
and $\zeta_{u(v_2,i)}^{k,s_2} = log_2({e})Q^{-1}(\epsilon) \sqrt{\frac{C_{u(v_2,i)}^{k,s_2}}{N_{u(v_2,i)}^{k,s_2}}})$




\newpage
\subsection{Mean Delay}
Assume the packet arrival of UEs follows a Poisson process with arrival rate $\lambda_{u(v,i)}$ for the $i^{th}$ UE of the $v^{th}$ service.
Therefore, the mean arrival data rate of UEs mapped to the $s^{th}$ slice in the CU layer is $\alpha_{s_j} = \sum_{v=1}^{V}\sum_{u=2}^{U_v}a_{v,s}\lambda_{u(v,i)}$, where $a_{v,s}$ is a binary variable which indicates whether the $v^{th}$ service is mapped to the $s^{th}$ slice or not.
Furthermore, the mean arrival data rate of the DU layer is approximately equal to the mean arrival data rate of the first layer $\alpha_{s} =\alpha_{s_j} \approx \alpha_{s_2}$ since, by using Burke’s Theorem, the mean arrival data rate of the second layer which is processed in the first layer is still Poisson with rate $\alpha_{s}$.
It is assumed that there are load balancers in each layer for each slice to divide the incoming traffic to VNFs equally \cite{frdl,luong2018novel,luong2018novel1}.
Suppose the baseband processing of each VNF is depicted as an M/M/1 processing queue.
Each packet is processed by one of the VNFs of a slice. So, the mean delay of the $s^{th}$ slice in the first and the second layer, modeled as M/M/1 queue, is formulated as follow, respectively
\begin{equation}
\begin{split}
d_{s_j} &= \frac{1}{\mu_1 - \alpha_{s}/{M_{s,1}}},\\
d_{s_2} &= \frac{1}{\mu_2 - \alpha_{s}/{M_{s,2}}}.
\end{split}
\end{equation}
where $1/\mu_1$ and $1/\mu_2$ are the mean service time of the first and the second layers respectively.
Besides, $\alpha_{s}$ is the  arrival rate which is divided
by load balancer before arriving to the VNFs. The  arrival rate of each VNF in each layer of the slice $s$ is $\alpha_{s}/{M_{s,i}}$ $ i \in \{1,2\}$.
In addition, $d_{s_{tr}}$ is the transmission delay for $s^{th}$ slice on the  wireless link. The arrival data rate of wireless link
 is equal to the arrival data rate of load balancers for each slice \cite{frdl}.
Moreover, it is assumed that the service time of transmission queue for each slice $s$ has
 an exponential distribution with mean $1/(R_{{tot}_s})$ and can be modeled as a M/M/1 queue \cite{frdl,luong2018novel,luong2018novel1,guo2016exploiting}. Therefore,
the mean delay of the transmission layer is
\begin{equation}
 d_{s_{tr}} = \frac{1}{R_{{tot}_s} - \alpha_{s}};
\end{equation}
where, $R_{{tot}_s} =  \sum_{v=1}^{V}\sum_{u=2}^{U_v}a_{v,s}R_{u(v,i)}$ is the total achievable rate of each slice that is mapped to specific service.
Mean delay of each slice is obtained as below.
\begin{equation}
D_{s} = d_{s_j} + d_{s_2} + d_{s_{tr}} \forall s.
\end{equation}
\subsection{Physical Data Center Resource}
Each VNF requires
physical resources that contain memory, storage and CPU.
Let the required resources for VNF $f$ in slice $s$ is represented by a tuple as
\begin{equation}
\bar{\Omega}_{s}^f = \{\Omega_{M,{s}}^f, \Omega_{S,{s}}^f, \Omega_{C,{s}}^f \},
\end{equation}
where $\bar{\Omega}_{s}^f\in \mathbb{C}^{3}$ and $\Omega_{M,{s}}^f, \Omega_{S,{s}}^f, \Omega_{C,{s}}^f$ indicate the amount of required memory, storage, and CPU, respectively.
Moreover, the total amount of required memory, storage and CPU of all VNFs of a slice is defined as
\begin{equation}
\textstyle \bar{\Omega}_{\mathfrak{z},s}^{tot} = \sum_{f=1}^{F_s}\bar{\Omega}_{\mathfrak{z},s}^f \;\; \mathfrak{z} \in \{M, S, C\}.
\end{equation}
Where, $F_s = M_{s_j} + M_{s_2}$
Also, there are $D_c$ data centers (DC), serving the VNFs. Each DC contains several servers that supply VNF requirements.
The amount of memory, storage and CPU is denoted by $\tau_{M_{j}}, \tau_{S_{j}}$and $\tau_{C_{j}} $ for the $j^{th}$ DC, respectively
\begin{equation*}
\tau_j = \{\tau_{M_{j}}, \tau_{S_{j}}, \tau_{C_{j}} \},
\end{equation*}
In this system model, the assignment of physical DC resources to VNFs is considered. Let $y_{s,d}$ be a binary variable indicating whether the $d^{th}$ DC is connected to the VNFs of $s^{th}$ slice or not.
\subsection{Problem Statement}
An important criterion to measure the optimality of a system is energy efficiency represented as the sum-rate to sum-power
\begin{equation}
\textstyle \eta(\boldsymbol{P},\boldsymbol{A}) := \frac{\sum\limits_{v=1}^{V} \sum\limits_{k=1}^{{U}_v}\mathcal{R}_{u_{(v,k)}} }{\sum\limits_{s=1}^{S} \sum\limits_{i=1}^{{R}_s}\bar{p}_{r_{(s,i)}}} = \frac{\mathfrak{R}_{tot}(\boldsymbol{P},\boldsymbol{A})}{P_r^{{tot}}(\boldsymbol{P},\boldsymbol{A})},
\end{equation}
where, $P_r^{tot}(\boldsymbol{P},\boldsymbol{A}) = \sum\limits_{s=1}^{S}\sum\limits_{i=1}^{{R}_s}\bar{p}_{r_{(s,i)}}$ is the total power consumption of all RUs in all slices. Also, $\mathfrak{R}_{tot}(\boldsymbol{P},\boldsymbol{A}) = \sum\limits_{v=1}^{V} \sum\limits_{k=1}^{{U}_v}\mathcal{R}_{u_{(v,k)}} $ is the total rates of all UEs applied for all types of services.
Assume the power consumption of baseband processing at each DC $d$ that is connected to VNFs of a slice $s$ is depicted as
$\phi_{s,d}$. So the total power of the system for all active DCs that are connected to slices can be represented as
\begin{equation*}
\textstyle \phi_{tot} = \sum_{s=1}^{S}\sum_{d=1}^{D_c}y_{s,d}\phi_{s,d}.
\end{equation*}
Also, a cost function for the placement of VNFs into DCs is defined as
\begin{equation}\label{eqpsi}
\textstyle  \psi_{tot} = \phi_{tot} - \nu \sum_{d=1}^{D_c}\sum_{v=1}^{V}y_{s,d}a_{v,s}
\end{equation}
where, $\nu$ is a design variable to value between the first term of \eqref{eqpsi} which is the total power consumption of physical resources and the second term that is shown the amount of admitted slices to have physical resources.
Our goal is to maximize sum-rate and minimize sum-power (the total power of all RUs and the total power consumption of baseband processing at all DCs) simultaneously, with the presence of constraints which is written as follow,
\begin{subequations}
\begin{alignat}{4}
\max\limits_{\boldsymbol{P}, \boldsymbol{A}, \boldsymbol{Y} }   \quad &  \eta(\boldsymbol{P},\boldsymbol{A})+ \varphi \frac{1}{\psi_{tot}(\boldsymbol{Y})} \\
\text{subject to} \quad  & \bar{p}_{r_{(s,i)}} \leq P_{max} \quad \forall s, \forall i,
 \label{c11} \\
&p_{u_{(v,k)}}  \geq 0  \quad \forall v, \forall k,\label{c12} \\
&\mathcal{R}_{u_{(v,k)}} \geq  \mathcal{R}_{u_{(v,k)}}^{min} \quad \forall v, \forall k,\label{c13} \\
&C_{r_{(s,i)}} \leq C_{r_{(s,i)}}^{max} \quad \forall s, \forall i, \label{c14}\\
&D_{s} \leq D_{s}^{max} \quad \forall s,\label{c15} \\
& \textstyle  \sum_{s=1}^{S}a_{v,s} \geq 1 \quad \forall s, \label{c21} \\
& \textstyle  \sum_{d=1}^{D_c}\sum_{v=1}^{V}y_{s,d}a_{v,s} \geq 1\times\sum_{v=1}^{V}a_{v,s} \forall s,\label{c23} \\
 &\textstyle \sum_{s=1}^{S} y_{s,d} \bar{\Omega}_{\mathfrak{z},s}^{tot}  \leq   \tau_{\mathfrak{z}_d}  \forall d, \forall \mathfrak{z}\in \mathcal{E}; \label{c22}
\end{alignat}
\label{constraints}
\end{subequations}
where $\boldsymbol{P} =[p_{u(v,k)}] \:\: \forall v , \forall k $, is the matrix of power for UEs, $\boldsymbol{A} =[a_{v,s}] \:\: \forall v , \forall s $ denotes the binary variable for connecting slices to services and $\boldsymbol{Y} =[y_{s,d}]  \:\: \forall s ,  \forall d $ is a binary variable shown whether
the physical DC is mapped to a VNFs of a slice or not.
Also, $\varphi$ is weighted variable to value between first and second term of objective function.
\eqref{c11}, and \eqref{c12}, indicate that the power of each RU do not exceed the maximum power, and the power of each UE is a positive integer value, respectively. Also \eqref{c13} shows that the rate of each UE is more than a threshold. \eqref{c14} and \eqref{c15} expressed the limited capacity of the fronthaul link, and the limited delay of receiving signal, respectively.
Furthermore, \eqref{c21}
ensures that each service is mapped to at least one slice.
Also, \eqref{c23}, guarantees that each slice (VNFs in two layers of slices) has been placed to one or more physical resources of DCs. Moreover, in \eqref{c22}  $\mathcal{E} = \{M,S,C\}$ and the constraint supports
that we have enough physical resources for VNFs of each slice.\newline
The optimization problem in \eqref{constraints} can be decomposed into two independent optimization sub-problems 1 and 2 since the variables can be obtained independently and respectively. Firstly we need to solve sub-problem 1. After obtaining $\boldsymbol{P}$ and $ \boldsymbol{A}$, sub-problem 2 can be solved by having the value of $ \boldsymbol{A}$. 
The sub-problem 1 is as follow
\begin{subequations}
\begin{alignat}{4}
\max\limits_{\boldsymbol{P}, \boldsymbol{A} }   \quad &   \eta(\boldsymbol{P},\boldsymbol{A})\\
\text{subject to} \quad  & \bar{p}_{r_{(s,i)}} \leq P_{max} && \quad \forall s, \forall i,   \\
&p_{u_{(v,k)}}  \geq 0  &&\quad \forall v, \forall k, \\
&\mathcal{R}_{u_{(v,k)}} \geq  \mathcal{R}_{u_{(v,k)}}^{min} && \quad \forall v, \forall k, \\
&C_{r_{(s,i)}} \leq C_{r_{(s,i)}}^{max}  &&\quad \forall s, \forall i,\label{cc14} \\
&D_{s} \leq D_{s}^{max}  &&\quad \forall s, \label{cc15} \\
& \textstyle  \sum_{s=1}^{S}a_{v,s} \geq 1 &&\quad \forall s.
\end{alignat}
\label{constraints1}
\end{subequations}
In sub-problem 2, $ \boldsymbol{Y}$ is obtained. The sub-problem 2 is
\begin{subequations}
\begin{alignat}{4}
\min\limits_{\boldsymbol{y} }   \quad &   \psi_{tot}(\boldsymbol{Y})\\
\text{s. t.} \quad & \textstyle \sum_{d=1}^{D_c}\sum_{v=1}^{V}y_{s,d}a_{v,s} \geq 1\times\sum_{v=1}^{V}a_{v,s} \forall s, \\
 &\textstyle  \sum_{s=1}^{S} y_{s,d} \bar{\Omega}_{\mathfrak{z},s}^{tot}  \leq   \tau_{\mathfrak{z}_d}  \forall d,, \forall \mathfrak{z}\in \mathcal{E};  \label{eqomega}
\end{alignat}
\label{constraints2}
\end{subequations}
\section{Numerical Results}\label{simul}

\section{Conclusion}



\bibliographystyle{IEEEtran}
\bibliography{ref}
\end{document} 