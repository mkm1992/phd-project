\chapter{تخصیص منابع در شبکه های \lr{Open RAN}}
در این فصل، هدف تخصیص منابع در شبکه های \lr{Open RAN} در لینک فروسو می باشد که شامل تخصیص توان و برش های شبکه است.
در این بخش فرض بر این است که شبکه ی نسل پنجم مخابرات با زیرساخت \lr{Open RAN} موجود است. این شبکه سرویس هایی از قبیل \lr{IoT}، سرویس تلفن، پیامک و ... در اختیار کاربران قرار می دهد. در اینجا از مفهوم برش شبکه استفاده شده است بدین صورت که به جای دیدن کاربران به صورت مجزا، کاربرانی که از یک سرویس خاص استفاده می نمایند در دسته ی آن سرویس قرار گرفته و دسته بندی می شوند. همچنین برش هایی از شبکه در اختیار کاربران هر سرویس خاص، قرار می گیرد که هر برش شبکه شامل تعدادی واحد رادیویی
\LTRfootnote{Radio Unit}
(\lr{RU}) 
،
بلوک های منبع فیزیکی
 \LTRfootnote{Physical Resource Block}
 (\lr{PRB})، 
 یک واحد توزیع شده
\LTRfootnote{Distributed Unit}
(\lr{DU}) 
 ، 
 یک واحد مرکزی
\LTRfootnote{Centralized Unit}
(\lr{CU})  
  می باشد که هر واحد توزیع شده و مرکزی شامل تعدادی توابع شبکه ی مجازی شده
  \LTRfootnote{Virtual Network Function}
(\lr{VNF}) 
می باشد. 
\section{مدل سیستم}
در این قسمت، سیستم مدل به صورت کامل مورد بررسی قرار می گیرد.
فرض می کنیم $S$ برش شبکه داریم که قرار است $V$ سرویس مختلف که شامل کاربرانی است که از سرویس خاص استفاده می نمایند را سرویس دهی نماید.
هر سرویس 
$v\in \{1,2,...,V \} $
شامل 
$U_v$
کاربر تک آنتنه می باشند که از سرویس خاصی را درخواست می نماید.
هر برش شبکه 
$s \in \{1,2,...,S \}$
شامل 
$R_s$
واحد رادیویی،
$K_s$
بلوک های منابع فیزیکی، یک واد توزیع شده و یک واحد مرکزی که شامل \lr{VNF} هایی می باشند.
همچنین برش های شبکه می توانند منابع مشترک داشته باشند.
تمام \lr{RU}هایی که به یک سرویس خاص خدمت رسانی می کنند به صورت مشارکتی سیگنال به تمام کاربران در آن سرویس ارسال می نمایند. 
\cite{motalleb2017optimal,mimoCran}
هر واحد رادیویی
$r \in \{1,2,...,R \}$
به یک واحد توزیع شده از طریق لینک فیبر نوری با ظرفیت \lr{fronthaul} 
محدود متصل می باشد.
در سیستم \lr{Open RAN}
دو لایه ی پردازشی که اولی در \lr{CU} و دومی در \lr{DU} قرار گرفته است که پردازش ها با \lr{VNF} ها صورت می گیرد.
لایه پایین تر (\lr{DU}) شامل 
\lr{high-PHY}
،
\lr{MAC}
و 
\lr{RLC}
می باشد و 
لایه ی بالاتر 
(\lr{CU})
شامل 
\lr{RRC}
،
\lr{PDCP}
و 
\lr{SDAP}
است.
فرض بر این است که $M_1$
\lr{VNF}
در \lr{DU}
و 
$M_2$
\lr{VNF}
در \lr{CU} قرار دارد.
هر \lr{VNF} به یک برش یا بیشتر تعلق دارد.
در $s^{th}$ امین برش شبکه $M_{s,1}$ 
\lr{VNF}
در \lr{DU}
و 
$M_{s,2}$
\lr{VNF}
دپر لایه ی \lr{CU} می باشد.
\lr{VNF}
های موجود در لایه ی \lr{DU} و \lr{VU} به ترتیب دارای ظرفیت محاسباتی 
$\mu_1$ 
و
 $\mu_2$
می باشند.
\subsection{نرخ قابل دسترس}
نرخ قابل دسترس برای $i^{th}$ امین کاربر در $v^{th}$امین سرویس به صورت زیر نوشته می شود
\begin{equation}\label{eq1}
\mathcal{R}_{u(v,i)} = B \log_2({1+ \rho_{u(v,i)}}),
\end{equation}
  که $B$ پهنای باند سیستم و 
  $\rho_{u(v,i)}$
  نسبت سیگنال به نویز $i^{th}$
  $i^{th}$ 
  امین کاربر در
   $v^{th}$
   امین سرویس
  می باشد 
  که از رابطه ی زیر بدست می آید
 \begin{equation}\label{eq2}
\rho_{u(v,i)} =  \frac{p_{u(v,i)}\sum_{s=1}^{S}|\bold{h}_{R_s,u(v,i)}^H \bold{w}_{R_s,u(v,i)}|^2 a_{v,s}}{BN_0 + I_{u(v,i)}},
\end{equation}
که   $p_{u(v,i)}$
نشان دهنده ی توان ارسالی توسط \lr{RU} به 
$i^{th}$ 
  امین کاربر در
   $v^{th}$
   امین سرویس
   است و 
 $\bold{h}_{R_s,u(v,i)} \in \mathbb{C}^{{R}_s}$
 بردار کانال گین لینک وایرلس از \lr{RU} ها در 
$s^{th}$
امین برش شبکه می باشد.
 همچنین 
$\bold{w}_{R_s,u(v,i)} \in \mathbb{C}^{{R}_s}$
نشان دهنده ی بردار بیم فرمینگ ارسالی برای \lr{RU}ها در $s$ امین برش شبکه به کاربر $i$ ام در سرویس $v$ ام می باشد.
   به علاوه، $BN_0$
   نشان دهنده ی توان نویز اضافه شونده ی گوسی می باشد و $I_{u(v,i)}$
   توان سیگنال تداخلی است.
همچنین $a_{v,s} \in \{0,1\}$
متغیر باینری است که نشان دهنده ی این است که برش شبکه ی $s$ به سرویس $v$ خدمات رسانی می کند یا نه.
درصورتی که 
 $a_{v,s} =1$
 برش شبکه ی $s$ به سرویس $v$ خدمات رسانی می کند. در غیر این صورت خدمت رسانی نمی کند.
\newline
برای بدست آوردن \lr{SNR} در فرمول \eqref{eq2}، 
فرض می شود که 
 $\bold{y}_{U_v}\in \mathbb{C}^{U_v} $
 بردار سیگنال دریافتی از همه ی کاربران در سرویس $v$ می باشد 
\begin{equation}\label{eq3}
\textstyle \bold{y}_{U_v} = \sum_{s = 1}^{S}\sum_{k=1}^{K_s} \boldsymbol{H}^H_{\mathcal{R}_s,\mathcal{U}_v} \
\mathfrak{y}_{R_s}\zeta_{U_v,k,s} a_{v,s}+ \boldsymbol{z}_{\mathcal{U}_v},
\end{equation}
که $\mathfrak{y}_{R_s} = \boldsymbol{W}_{\mathcal{R}_s,\mathcal{U}_v}\boldsymbol{P}_{U_v}^{\frac{1}{2}}\boldsymbol{x}_{\mathcal{U}_v}+ \boldsymbol{q}_{\mathcal{R}_s}$
و 
$\boldsymbol{x}_{ \mathcal{U}_v} = [x_{ u_{(v,1)}},...,x_{ u_{(v,\mathcal{U}_v)}}]^T \in \mathbb{C}^{{R}_s } $
نشان دهنده ی بردار سمبل ارسالی کاربران سرویس $v$ می باشد.
$\boldsymbol{z}_{U_v}$
نویز گوشی جمع شونده است و
$\boldsymbol{z_{U_v}} \backsim \mathcal{N}(0,N_0\boldsymbol{I}_{{U}_v})$
.
همچنین
$N_0$
توان نویز می باشد.
علاوه بر این
$\boldsymbol{q}_{R_s} \in \mathbb{C}^{{R}_s }  $
نشان دهنده ی نویز کوانتیزاسیون می باشد که از فشرده سازی سیگنال  در واحد توزیع شده بدست آمده است.
$\boldsymbol{P}_{U_v} = \diag{(p_{u_{(v,1)}}, ..., p_{u_{(v,\mathcal{U}_v)}})}$.
