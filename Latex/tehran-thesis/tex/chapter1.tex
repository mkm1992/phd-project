\chapter{مقدمه}
\subsection{مقدمه ای بر \lr{5G} }
\lr{5G}،
مخابرات نسل پنجم
سیستم های بیسیم \LTRfootnote{Wireless} وشبکه های مخابراتی بعد از نسل چهارم می باشد که تکاملی از لایه ی فیزیکی در تکنولوژی شبکه های مخابراتی سیار همانند \lr{LTE} می باشد که نسبت به \lr{4G} سرعت و پوشش بهتری را فراهم می کند.  
\lr{5G}
نوع جدیدی از شبکه را ایجاد می کند که به منظور اتصال تقریبا همه و همه چیز با هم از جمله ماشین ها، اشیاء و دستگاه ها ساخته شده است.
\lr{5G}
 فناوری بی سیم برای ارائه سرعت داده های چند گیگابیت بر ثانیه ، تأخیر فوق العاده کم ، قابلیت اطمینان بیشتر ، ظرفیت شبکه گسترده، افزایش در دسترس بودن و تجربه کاربری یکنواخت تر به کاربران بیشتر است. عملکرد بالاتر و بهره وری بهبود یافته باعث افزایش تجربیات کاربر جدید شده و صنایع جدیدی را به هم متصل می کند.
 
 
تکنولوژی سیگنال  \lr{5G} برای پوشش فراگیرتر و بازدهی بهتر سیگنال ایجاد شده است. این پیشرفت ها منجر به تغییراتی از قبیل\lr{IOT} \LTRfootnote{Internet of Things} و \lr{Pervasive Computing} در آینده ی نزدیک خواهد شد.
همچنین \lr{5G} منجر به توسعه و بهبود سرویس های مخابراتی و اینترنتی سیار و در ورای آن، ایجاد تجربه ی بهتری برای مصرف کنندگان خواهد شد.\newline
برای توسعه ی اینترنت سیار و \lr{IOT}، نیاز داریم تا شبکه های\lr{5G}، معنای اولیه برای دسترسی شبکه برای ارتباط انسان ها با یکدیگر و ارتباط ماشین با انسان گردد.
به طور کلی، 
\lr{5G}
 در سه نوع سرویس اصلی متصل از جمله پهن باند تلفن همراه، IoT عظیم و ارتباطات مهم برای ماموریت استفاده می شود.
\begin{enumerate}
\item 
پهن باند تلفن همراه پیشرفته 
\lr{(eMBB)}
 برای مقابله با نرخ داده های بسیار زیاد ، تراکم بالای کاربران و ظرفیت ترافیک بسیار بالا برای سناریوهای مختلف و همچنین پوشش یکپارچه و سناریوهای تحرک بالا با نرخ داده های استفاده شده بهبود یافته است.
\item
 ارتباطات  عظیم ماشین 
 \lr{(mMTC)}
  برای
\lr{IoT}،
   برای تعداد بسیار زیاد دستگاههای متصل به مصرف کم و نرخ داده کم نیاز دارد.
\item 
ارتباطات بسیار مطمئن و با تأخیر کم
 \lr{(URLLC)}
 برای برنامه های کاربردی مهم برای ایمنی و ماموریت
  مهم است.
\end{enumerate}
 از آنجا که
\lr{5G}
 تکامل می یابد و کمتر به زیرساخت های \lr{4G} وابسته می شود و طیف بیشتری در دسترس قرار می گیرد،
 تخمین ها سرعت بارگیری را حداکثر 1000 برابر سریعتر از \lr{4G} قرار می دهد، که بالقوه از 
 \lr{10Gbps} 
بیشتر است، که به شما امکان می دهد تا در کمتر از یک ثانیه فیلم کامل
\lr{HD}
   را بارگیری کنید.
   برخی تخمین ها محافظه کارانه تر هستند ، اما حتی محافظه کارانه ترین آن را چندین ده برابر سریعتر از \lr{4G} قرار می دهد.
دلایل نیاز به نسل پنجم اینترنت به طور خلاصه در ادامه بیان شده است \citep{etsi}.
\begin{itemize}
\item ترافیک داده های تلفن همراه به دلیل پخش ویدئو به سرعت رو به افزایش است.
\item با در اخیار داشتن چندین دستگاه به طور همزمان، هر کاربر تعداد فزاینده ای از اتصالات را در اختیار دارد.
\item اینترنت اشیاء به شبکه هایی نیاز دارد که میلیاردها دستگاه را اداره کنند.
\item با وجود تعداد فزاینده ای از دستگاه های ارتباطی و افزایش ترافیک داده ها ، هم دستگاه ها و هم شبکه ی  
آن
نیازمند افزایش بهره وری انرژی هستند.
\item 
 به دلیل تحت فشار قرار گرفتن اپراتورهای شبکه برای کاهش هزینه های عملیاتی و همچنین به دلیل اینکه کاربران به تعرفه های نرخ مسطح عادت می کنند و مایل نیستند مبلغ بیشتری بپردازند.
\item فناوری ارتباطات سیار می تواند موارد استفاده جدیدی را ایجاد کند (به عنوان مثال موارد تاخیر فوق العاده کم یا قابلیت اطمینان بالا) و برنامه های جدید برای صنعت که منجر به درآمد زایی بیشتر اپراتورها می گردد.
\end{itemize}
بنابراین عملکرد عملیاتی نسل پنجم می بایست به طور قابل توجهی افزایش یابد (به عنوان مثال افزایش راندمان طیفی ، سرعت بالاتر داده ، تأخیر کم).
\lr{5G}
می بایست
در حالی که هنوز سطح قابل قبولی از مصرف انرژی ، هزینه تجهیزات و استقرار شبکه و هزینه بهره برداری را ارائه می دهد،
 اینترنت اشیاء را به طور گسترده نیز تأمین کند.
 همچنین می بایست 
از طیف گسترده ای از برنامه ها و خدمات پشتیبانی کند.
 \begin{figure}[H]
  \centering
    \includegraphics[width=0.8\textwidth]{./fig/etsi}
  \caption{مقایسه قابلیت های کلیدی
\lr{IMT-Advanced}
 (نسل 4) با
\lr{IMT-2020}
 (نسل ۵) 
 با توجه به
\lr{ITU-R M.2083}
\cite{etsi}
  }
  \label{fig:C-RAN}
\end{figure}
\subsection{تاریخچه مخابرات}
در ابتدا می خواهیم بدانیم که چه چیزی منجر به رفتن محققان به سوی  \lr{5G} شده است. یکی از دلایل مهم، سرعت و نرخ انتقال بیشتری است که در ادامه به آن می پردازیم.
نیاز بشریت به ارتباط تلفنی (انتقال بدون سیم به صورت زمان حقیقی \LTRfootnote{Real Time} انسان را به سمت نسل اول ارتباطات \lr{1G} سوق داده است. نسل دوم ارتباطات \lr{2G} با سرویس های انتقال پیام کوتاه ایجاد شد. همچنین با موفقیت تکنولوژی شبکه های منطقه ای بیسیم، اتصال به داده های اینترنتی مورد توجه عموم مردم قرار گرفت که پلی به سوی نسل سوم ارتباطات \lr{3G} را فراهم نمود. به طور منطقی پله ی بعدی گام برداشتن در راستای کوچک شدن لپ تاپ و در آمیختن آن با تلفن که امروزه به صورت تلفن هوشمند\LTRfootnote{smart phone} است و دسترسی به  اینترنت، پهنای باند بالا و داده ها در نقاط مختلف جهان بوده است که \lr{4G} یا نسل چهارم را به همراه داشته است.
با توجه به افزایش تعداد کاربران تلفن های
هوشمند و تبلت ها و افزایش نرخ ارسال اطلاعات و داده ها در طی سال
های اخیر طبق پیش بینی های سیسکو میزان ترافیک \lr{IP} طی سالهای اخیر
  چندین برابر افزایش خواهد یافت.
در نتیجه اپراتورها برای حل این مشکل و خدمات
دهی بهتر ناچار به افزایش ظرفیت شبکه می باشند.
در ادامه به طور مختصر به ۵ نسل مخابراتی می پردازیم\cite{Gen}.
\subsubsection{نسل اول}
این اولین نسل از فناوری تلفن همراه بود. اولین نسل شبکه تلفن همراه تجاری در اواخر دهه 70 معرفی شد و استانداردهای کاملاً اجرا شده در دهه 80 تاسیس شد. استرالیا در سال 1987 توسط \lr{Telecom} (که امروزه با عنوان \lr{Telstra} شناخته می شود) معرفی شد ، استرالیا اولین شبکه تلفن همراه خود را با استفاده از یک سیستم آنالوگ \lr{1G} دریافت کرد. \lr{1G} یک فناوری آنالوگ است و به طور کلی تلفن ها از باتری ضعیف برخوردار هستند و کیفیت صدا بدون امنیت بسیار زیاد بود و گاهی اوقات تماس های کاهش یافته را تجربه می کنید. این استانداردهای ارتباطی آنالوگ ارتباطی است که در دهه 1980 معرفی شد و تا زمانی که جایگزین ارتباطات دیجیتال \lr{2G} شود ، ادامه یافت. حداکثر سرعت \lr{1G} 2.4 کیلوبیت بر ثانیه است.
\subsubsection{نسل دوم}
تلفنهای همراه وقتی از \lr{1G} به \lr{2G} رفتند ، اولین نسخه اصلی خود را دریافت کردند. تفاوت اصلی بین دو سیستم تلفن همراه (\lr{1G} و \lr{2G}) در این است که سیگنالهای رادیویی مورد استفاده شبکه \lr{1G} آنالوگ هستند ، در حالی که شبکه های 
\lr{2G}
 دیجیتال هستند. انگیزه اصلی این نسل، تهیه کانال ارتباطی ایمن و مطمئن بود. این نسل همچنین مفهوم \lr{CDMA} و \lr{GSM} را پیاده سازی کرد که خدماتی مانند پیامک را ارائه داده است. شبکه های مخابراتی سلولی نسل دوم بطور تجاری در سال 1991 توسط رادیولینجا (در حال حاضر بخشی از الیزا اویج) توسط استاندارد \lr{GSM} در فنلاند راه اندازی شد.
\lr{2G}
قابلیت هایی از قبیل
\lr{multiplexing}
دارد که 
به چندین کاربر در یک کانال منفرد 
اجازه ی انتقال داده می دهد.
انتقال داده و صدا در این نسل وجود داشته است.
در این نسل مخابرات، 
سرویسهای اساسی مانند پیام کوتاه، رومینگ داخلی، تماس های کنفرانسی، نگه داشتن تماس و صورتحساب مبتنی بر خدمات معرفی شده است. جداسازی هزینه های مبتنی بر تماس های مسافت طولانی و صورتحساب در زمان واقعی نیز از قابلیت های این نسل بود.
حداکثر سرعت برای سرویس \lr{GPRS}
\LTRfootnote{General Packet Radio Service}
\lr{50Kbps}
و برای سرویس 
\lr{EDGE}
\LTRfootnote{Enhanced Data Rates for GSM Evolution}
\lr{1Mbps}
می باشد.
\subsubsection{نسل سوم}
این نسل مخابراتی استانداردهای بسیاری از فناوری های بیسیم را تعیین نمود. مرورگر وب ، ایمیل ، بارگیری ویدیو ، به اشتراک گذاری عکس و سایر فناوری های هوشمند در نسل سوم معرفی شدند که در سال 2001 به طور تجاری معرفی شد. اهداف تعیین شده برای ارتباطات سیار نسل سوم، تسهیل ظرفیت بیشتر صدا و داده ، پشتیبانی از طیف گسترده تری از برنامه ها و افزایش انتقال داده با هزینه کمتری بود.
استاندارد \lr{3G} از فناوری جدیدی به نام
 \lr{UMTS}
 \LTRfootnote{Universal Telecommunications System Mobile}
 به عنوان معماری اصلی شبکه خود 
استفاده می کند.
\lr{3G}
 دارای پشتیبانی خدمات چندرسانه ای است.
\lr{3G}
  با بهبود چگونگی فشرده سازی صدا در طی تماس، راندمان طیف فرکانس را افزایش داده است، بنابراین تماس همزمان بیشتری می تواند در همان محدوده فرکانس اتفاق بیفتد.
حداکثر سرعت این نسل 
\lr{2Mbps}
و
حداکثر سرعت نظری برای HSPA + 21.6 Mbps است. 
\subsubsection{نسل چهارم}
\lr{4G}
 یک فناوری بسیار متفاوت در مقایسه با \lr{3G} است و هدف از آن ، فراهم آوردن سرعت بالا ، کیفیت بالا و ظرفیت بالا برای کاربران در عین بهبود امنیت و کاهش هزینه خدمات صوتی و دیتا ، چندرسانه ای و اینترنت از طریق \lr{IP} می باشد. برنامه های کاربردی بالقوه و جاری شامل دسترسی به وب موبایل اصلاح شده ، تلفن تلفنی \lr{IP}، خدمات بازی، تلویزیون همراه با کیفیت بالا ، کنفرانس ویدیویی، تلویزیون سه بعدی و محاسبات ابری از قابلیت های پشتیبانی آن می باشد.

فن آوری های کلیدی که این امکان را ایجاد کرده اند 
\lr{MIMO} \LTRfootnote{Multiple Output Multiple Output}
 و
  \lr{OFDM} \LTRfootnote{Multiplexing Division Frequency Division}
می باشد.
دو استاندارد مهم آن
\lr{LTE} \LTRfootnote{Long Term Evolution}
و
\lr{WiMAX}
می باشد.
حداکثر سرعت یک شبکه 4G هنگام حرکت دستگاه 100 مگابیت بر ثانیه یا 1 گیگابیت بر ثانیه برای ارتباطات کم تحرک مانند هنگام ایستادن یا راه رفتن است. تأخیر از حدود \lr{300ms} به  \lr{100ms} با کاهش تراکم دست می یابد.
 \subsubsection{نسل پنجم}
 با توجه
به این که نرخ داده و ظرفیت در سیستم های نسل چهارم به ظرفیت
شانون نزدیک شده است، در نتیجه روش هایی که برای
افزایش ظرفیت شبکه مورد استفاده می گیرند که به شرح زیر است:
\begin{itemize}
\item
استفاده از تکنیک \lr{Massive Mimo}
\item
استفاده از روش های پردازش های ابری
\item
شبکه ی تعریف شده ی نرم افزاری
\lr{SDN}
\LTRfootnote{Software Defined Networking}
\item
موج میلیمتری
\LTRfootnote{mm Wave}
\item 
ساختار شبکه های دسترسی رادیویی باز
\lr{ORAN}
\LTRfootnote{Open Radio Access Network}
\item 
مجازی سازی توابع شبکه
\lr{NFV}
\LTRfootnote{Network Function Virtualization}
\item 
برش شبکه
\LTRfootnote{Network Slicing}
\end{itemize}
\section{مقدمه ای بر ساختار \lr{ORAN}}
مجازی سازی \lr{RAN} توجه زیادی را از طرف اپراتورها به خود جلب می کند ، زیرا منجر به کاهش هزینه های اپراتور و \lr{opex} می شود و همچنین این امکان را برای آنها فراهم کند تا با سرعت بیشتری قابلیت های جدیدی به شبکه اضافه کنند.

این احتمال وجود دارد که همه این علاقه ها در ایجاد سه گروه مختلف باشد -
 انجمن 
 \lr{xRAN} 
 ، گروه
 \lr{ OpenRAN }
  شرکت 
 \lr{ Telecom Infra}
   و ابتکار عمل
\lr{Open VRAN}
که برای شرکت سیسکو می باشد.
اگرچه همه این گروه ها می گویند که در حال کار بر روی یک چیز هستند، که اساساً برای باز کردن \lr{RAN} با استفاده از رابط های استاندارد و عناصر شبکه جعبه سفید است، اما در بررسی دقیق تر اختلافاتی نیز وجود دارد.

\lr{Open RAN}(\lr{ORAN})
تبسیط و ترکیبی از دو ساختار \lr{C-RAN} \LTRfootnote{Cloud Radio Access Network} و \lr{xRAN} می باشد که انتظار می رود که در فناوری نسل پنجم مخابرات مورد استفاده قرار گرفته و منجر به بهبود عملکرد شبکه های دسترسی رادیویی \lr{RAN} گردد. 
این ساختار یک شبکه ی باز، انعطاف پذیر و هوشمند است.


\lr{ORAN} 
توابع
 شبکه ی دسترسی رادیویی 
 را به سه قسمت تقسیم می کند،
  که قسمت اول واحد از راه دور
 \lr{(RU)} \LTRfootnote{remote unit} 
 ، واحد توزیع شده
  \lr{(DU)} \LTRfootnote{Distributed unit} 
  و واحد مرکزی 
   \lr{(CU)} \LTRfootnote{Central unit} 
   می باشد.
   در حالی که  \lr{RU} دارای توابع  \lr{(PHY)} \LTRfootnote{Physical layer} پایین تر است،
    \lr{DU} حاوی \lr{(PHY)} بالاتر، 
    \lr{MAC} 
 \LTRfootnote{Medium Access Control}   
    و
     \lr{RLC}
   \LTRfootnote{Radio Link Control}  
      است     
    و 
    \lr{(CU)}
     حاوی
     \lr{RRC}
     \LTRfootnote{Radio Resource Control}
      ،\lr{PDCP}
      \LTRfootnote{Packet Data Convergence Protocol}
      و 
      \lr{SDAP}
      \LTRfootnote{Service Data Adaptation Protocol}
      است.
      
\lr{DU}
و
\lr{CU}
به عنوان توابع شبکه مجازی \lr{(VNFs)} پیاده سازی می شوند،
که در یک محیط ابر اجرا می شود.

رابط های بین \lr{RU} ، \lr{CU} و \lr{DU} رابط های استاندارد باز هستند.
\subsection{مقدمه ای بر ساختار شبکه های دسترسی رادیویی \lr{C-RAN}}
شبکه های دسترسی ابری منجر به افزایش پوشش ارسالی می گردد. با توجه به ساختار شبکه
   \lr{C-RAN} که معماری جدیدی را برای شبکه های نسل آینده
ارائه می دهد، نه تنها ظرفیت شبکه افزایش می یابد بلکه
مشکلاتی که در روش های دیگر وجود دارد را نیز هموار
می سازد.
مفهوم شبکه دسترسی رادیو ابر \lr{C-RAN}، به مجازی سازی کارکردهای ایستگاه  پایه \LTRfootnote{Base Station-BS} با استفاده از تکنولوژی رایانش ابری \LTRfootnote{Cloud Computing} اشاره می نماید. این مفهوم به ایجاد یک ساختار سلولی جدید منجر می شود که در آن، نقاط دسترسی بیسیم کم هزینه که با عنوان واحدهای رادیویی \LTRfootnote{Radio Units} و یا رادیو هد های
  راه دور 
  \LTRfootnote{Radio Remote Heads}
 شناخته می شوند- با استفاده از یک ابر متمرکز با قابلیت پیکربندی مجدد و یا واحد مرکزی \LTRfootnote{Control Unit} مدیریت می شوند. شبکه امکان کاهش هزینه های سرمایه گذاری و عملیاتی مورد نیاز برای اپراتور ها به منظور توسعه و نگهداری شبکه های ناهمگن متراکم را فراهم می آورد. این مزیت مهم در کنار بازده طیفی، تسهیم آماری \LTRfootnote{Statisitical Multiplexing}، و مزیت های متعادل سازی بار باعث می شود تا شبکه \lr{C-RAN} به عنوان یکی از تکنولوژی های کلیدی در توسعه سیستم های \lr{5G} در جایگاه بسیار مناسبی قرار بگیرد. در ادامه، یک بررسی کلی و مختصر از تحقیقات جدید در مورد ساختار \lr{C-RAN} ارائه می شود و موضوعات مورد تاکید عبارتند از فشرده سازی لینک \lr{fronthaul} پردازش باند پایه، کنترل دسترسی به محیط واسط، تخصیص منابع، ملاحظات سطح سیستم، و تلاش های انجام شده در راستای ارائه استاندارد ها.
\subsubsection{ساختار شبکه های مختلف }
با توجه به مقاله ی\cite{checko2015cloud}،
\begin{figure}
  \centering
    \includegraphics[scale=0.7]{./fig/c11}
  \caption{ساختار سنتی ایستگاه پایه \cite{checko2015cloud}}
  \label{fig:c11}
\end{figure}
هر ایستگاه پایه دو نوع پردازش انجام می دهد : پردازش
رادیویی که توسط واحد رادیویی \LTRfootnote{RRH} انجام می شود و شامل پردازش
دیجیتالی، فیلترینگ فرکانسی، تقویت توان و ....میباشد و
پردازش باند پایه که توسط واحد باند پایه \LTRfootnote{BBU} که همان واحد کنترل است \LTRfootnote{CU} انجام شده و از جمله
مهمترین وظایف آن می توان به کدینگ، مدولاسیون و
تبدیل فوریه ی سریع اشاره کرد. در ساختار جدیدی که
تحت عنوان \lr{C-RAN}  معرفی خواهیم نمود نحوه ی ارتباط
پردازشگرهای رادیویی و باند پایه متحول شده و در نتیجه
مزایایی برای شبکه حاصل خواهد شد.در ادامه ، انواع ساختارها را بیان خواهد شد.
\subsubsection{ساختار سنتی ایستگاه پایه }

در ساختارهای سنتی ایستگاه پایه، پردازش های رادیویی و باند پایه در
داخل ایستگاه پایه انجام می شد و مدول آنتن نیز در فاصله
ی چند متری از مدول رادیویی نصب شده و ارتباط آنها
توسط کابل کواکسیال برقرار می شد که همین امر سبب
افزایش تلفات در شبکه می باشد. این نوع ساختار در شکل
\ref{fig:c11} نشان داده شده است. همان گونه که مشاهده می کنید
ارتباط بین ایستگاههای پایه توسط ارتباط  $X_2$ و ارتباط بین
ایستگاه پایه و شبکه ی هسته توسط ارتباط $ S_1$ برقرار می
شود. این نوع ساختار در شبکه های \lr{1G} و \lr{2G} به کار گرفته
شده است 
\cite{checko2015cloud}.

%Figure \ref{fig:gull} shows a photograph of a gull.
\subsubsection{ ساختار ایستگاه پایه و واحد رادیویی}

\begin{figure}
  \centering
    \includegraphics[scale=0.7]{./fig/c22}
  \caption{ ساختار ایستگاه پایه و واحد رادیویی \cite{checko2015cloud}}
  \label{fig:c22}
\end{figure}
در این ساختار واحد رادیویی و واحد پردازشی سیگنال، از هم
مجزا شده و واحد رادیویی که تحت عنوان \lr{RRH} یا \lr{RRU}
نیز شناخته می شود، توسط فیبر نوری به واحد باند پایه یا \lr{BBU} اتصال می
یابد. همان طور که پیشتر بیان شد واحد رادیویی مسئولیت
انجام پردازش های دیجیتالی از جمله تبدیل انالوگ به
دیجیتال، دیجیتال به انالوگ، تقویت توان و فیلترینگ رابر عهده دارد، که تفکیک وظایف واحد پردازشی و واحد
رادیویی در این ساختار در شکل \ref{fig:c22} قابل مشاهده است. این
نوع ساختار برای شبکه های نسل سوم معرفی شده و امروزه
نیز بیشتر ایستگاههای پایه از همین ساختار بهره می گیرند.
از جمله ویژگی های بارز این ساختار امکان ایجاد فاصله
بین واحد رادیویی و پردازشی می باشد، که این فاصله به
دلیل تاخیر پردازشی و انتشاری نمی تواند از  $40$کیلومتر
فراتر رود. در این ساختار تجهیزات مرتبط با \lr{BBU} می
توانند به مکانی مناسبتر که قابل دسترس تر بوده و هزینه
ی اجاره و نگهداری کمتری را به اپراتورها تحمیل می
کنند منتقل شوند و واحد های رادیویی نیز در در پشت بام
ساختمان ها و مکان های مرتفع نصب می شوند که این
خود سبب کاهش هزینه های خنک سازی ادوات موجود
می شود. نحوه ی ارتباط بین \lr{RRH} و \lr{BBU} مشابه ساختار
سنتی بوده و \lr{RRH} ها نیز توسط معماری زنجیروار با هم
در ارتباطند.
\begin{figure}[H]
  \centering
    \includegraphics[scale=0.7]{./fig/c33}
  \caption{ساختار  \lr{C-RAN} \cite{checko2015cloud}}
  \label{fig:c33}
\end{figure}
\subsubsection{ساختار \lr{C-RAN}}
در ادامه ساختار های شبکه دسترسی رادیویی ابری و ساختارهای بهبود یافته ی آن را معرفی می نماییم.
\begin{itemize}
\item \textbf{شبکه های دسترسی رادیویی ابری}


ایده اصلی \lr{C-RAN} جداسازی بخش رادیویی (\lr{RRH}) 
\LTRfootnote{Radio Remote Head}
 از واحد پردازشی باند پایه (\lr{BBU})
 \LTRfootnote{Baseband Unit}
  است.
از تجمیع \lr{BBU} ها بر روی سرور ابری، \lr{BBU-Pool} ایجاد می شود.
در این ساختار، در راستای بهینه سازی عملکرد \lr{BBU}
 ها در مواجهه باایستگاههای پایه پر ترافیک و کم ترافیک،
 \lr{BBU}ها به صورت یک مجموعه ی واحد تحت عنوان 
\lr{BBU Pool}
 در آمده اند که این مجموعه بین چندین سلول 
 به اشتراک گزارده شده و مطابق شکل زیر مجازی سازی
می شود. 
در توضیح بیشتر این ساختار می توان این گونه
عنوان کرد که \lr{BBU Pool} به عنوان یک خوشه ی مجازی
در نظر گرفته می شود که شامل پردازش گرهایی می باشد
که پردازش های باند پایه را انجام می دهند. ارتباط بین
  \lr{BBU}ها در ساختار های فعلی به شکل  $X_2$ برقرار می شود
که در این ساختار ارتباط بین خوشه ها از فرم جدید $X_2$
تحت عنوان  $X_2 +$برقرار می شود.
\newline
در شکل \ref{fig:C-RAN} ساختار کلی شبکه ی  \lr{C-RAN} در سیستم های
\lr{ LTE}
 نمایش داده شده است. همان طور که در شکل قابل
مشاهده می باشد ساختار کلی شبکه  \lr{C-RAN} به دو بخش
 \lr{backhaul} و \lr{fronthaul} تقسیم بندی شده است. بخش
 \lr{fronthaul}شبکه به مرحله ی اتصال سایت های \lr{ RRH}به
 به \lr{BBU Pool} به اتصال \lr{backhaul} و بخش \lr{BBU Pool}
هسته ی شبکه ی سیار اطلاق می شود. همان گونه که قبلا
ذکر شد  \lr{ RRH}ها در نزدیکی انتن نصب شده و از طریق
لینک های انتقالی نوری با پهنای باند وسیع و تاخیر کم به
پردازشگرهای قوی در  \lr{BBU}متصل می شوند. توسط این
لینک های انتقالی است که سیگنال های دیجیتالی باند
پایه از نوع \lr{IQ} بین \lr{RRH} و \lr{BBU} انتقال می یابند \cite{checko2015cloud}.
\begin{figure}[H]
  \centering
    \includegraphics[width=\textwidth]{./fig/CRAN}
  \caption{ساختار شبکه ی \lr{C-RAN} \cite{checko2015cloud}}
  \label{fig:C-RAN}
\end{figure}
\item \textbf{شبکه های دسترسی رادیویی ابری نا متجانس (\lr{H-CRAN})}


برای غلبه بر چالش های شبکه های \lr{C-RAN} با محدودیت های \lr{fronthaul} ، شبکه های دسترسی ابری نامتحانس (\lr{H-CRAN}) معرفی می شود\cite{ fogComputing, heterogeneous, fogEdge}.
\begin{figure}
  \centering
    \includegraphics[scale = 0.8]{./fig/hc}
  \caption{ ساختار شبکه های دسترسی ابری نامتحانس \cite{heterogeneous}  }
  \label{fig:hc}
\end{figure}

کاربر و صفحه ی کنترلگر در چنین شبکه هایی از هم مجزا می باشند. که در این شبکه ها ، نودهای توان بالا   \LTRfootnote{High Power Node}\lr{HPN} ، عمدتا برای فراهم کردن پوشش بدون درز و اجرای عملکرد صفحه کنترل می باشد. در حالی که \lr{RRH} ها برای فراهم نمودن سرعت بالای نرخ داده برای انتقال بسته در ترافیک قرار گرفته اند. \lr{HPN} ها از طریق لینکهای \lr{backhaul}  به \lr{BBU Pool} متصلند ( برای هماهنگ کردن تداخل ).\newline
ساختار این شبکه شبیه به ساختار \lr{C-RAN} می باشد . همانطور که در شکل \eqref{fig:hc} نشان داده شده است ، تعداد زیادی \lr{RRH} ، همراه با انرژی مصرفی کم در ساختار \lr{H-CRAN} ، با یکدیگر در \lr{BBU Pool} مرکزی ، همکاری می کنند تا گین مشترک بالایی بدست آورند.   تنها ، فرکانس رادیویی جلو ،(\lr{RF}) و عملکردهای پردازشی  ساده ، در \lr{RRH} ، صورت می گیرد ، در حالی که پردازشهای مهم دیگر ، در \lr{BBU Pool} انجام می گیرد. همچنین تنها بخشی از عملکردها در لایه ی \lr{PHY} در \lr{RRH} به مشارکت می انجامد که این مدل در شکل \eqref{fig:hc} نشان داده شده است.\newline
اگرچه ، برخلاف \lr{C-RAN} ، \lr{BBU Pool} در \lr{H-CRAN} ، به \lr{HPN} ها متصلند که این، برای کاهش تداخل متقابل بین \lr{RRH} ها و \lr{HPN} ها از طریق محاسبات ابری متمرکز بر اساس تکنیکهای پردازشی مشترک می باشد. همچنین ، داده و واسط کنترل ، بین \lr{BBU Pool} و \lr{HPN} های $S_1$ و $X_2$ شناخته شده اند که تعریف آنها بر اساس تعریف استاندارد \lr{3G} ایجاد شده است.\newline
همانطور که سرویسهای صدا ، می توانند به صورت بهینه در طول مد سوییچ بسته در \lr{4G} فراهم گردند ، \lr{H-CRAN} می تواند به طور همزمان سرویس صدا و داده را پشتیبانی کند. سرویس صدا مرجح به اداره از طریق \lr{HPN} ها می باشد ، در حالی که ترافیک بسته ی پر داده ، بیشتر توسط \lr{RRH} اداره می گردد. 
در مقایسه با ساختار \lr{C-RAN} ،ساختار \lr{H-CRAN} ، نیازهای \lr{fronthaul} را بوسیله ی مشارکت \lr{HPN} ها برطرف می سازد. با توجه به حضور \lr{HPN} ها ،سیگنالهای کنترلی و سمبلهای داده در \lr{H-CRAN} جدا از هم می باشند. تمام کنترل کننده های سیگنال و سیستم هایی که اطلاعات را ارسال می نمایند ، توسط \lr{HPN} ها به \lr{UE} ، منتقل می گردد که منجر به سادگی در ظرفیت و در محدودیت تاخیر زمان در لینکهای \lr{fronthaul } بین \lr{RRH} ها و \lr{BBU Pool}  می گردد و منجر به صرفه جویی در مصرف انرژی می گردد. همچنین ، برخی از ترافیک های شدید و ناگهانی \LTRfootnote{Burst Traffic} و یا سرویس پیام همراه با مقدار داده ی کم ، می تواند به صورت بهینه توسط \lr{HPN} ها پشتیبانی گردد. مکانیزم کنترل بین ارتباط داشتن و نبود ارتباط ، توسط \lr{H-CRAN} پشتیبانی می گردد که منجر به حفظ کردن مقدار قابل توجهی \lr{Overhead} در رادیو بوسیله ی مکانیزم ارتباط جهت دار خالص می گردد. در \lr{RRH} ، تکنولوژی های مختلف انتقال در لایه ی \lr{PHY} ، قابل استفاده برای بهبود نرخ انتقال (همانند موج میلیمتری و نور مرئی) می گردد. در \lr{HPN}ها، \lr{MIMO}\LTRfootnote{Multiple Input Multiple Output}، یکی از راه های افزایش پوشش در بهبود ظرفیت می باشد.

\item \textbf{ساختار دسترسی رادیویی مهی}


برای حل کردن مشکلات \lr{H-CRAN} و \lr{C-RAN}، نیاز به معرفی ساختار جدید دیگری می باشیم که آن را \lr{F-RAN} می نامیم.
\lr{F-RAN} تمام ویژگی های مثبت محاسبات ابری و شبکه های نامتجانس و محاسبات مهی را همزمان در بر می گیرد.
محاسبات مهی ، اصطلاحی برای جایگزین کردن محاسبات ابری است که مقدار قابل توجهی از ذخیره سازی ، ارتباطات ، کنترل کردن ، اندازه گیری و مدیریت را در لبه ی شبکه انجام می دهد (نه در کانال و ابر مرکزی) \cite{fogComputing, fogEdge}.
 سیستمهای \lr{F-RAN} تحولی از سیستمهای \lr{C-RAN} می باشد . برخی از ارتباطات توزیع شده و عملکردهای ذخیره سازی در منطق لایه ی مه قرار دارد. همچنین چهار نوع ارتباطات ابری تعریف شده است.
  \begin{figure}[H]
  \centering
    \includegraphics[scale =0.7]{./fig/fr}
  \caption{ مدل سیستم \lr{F-RAN} \cite{fogComputing} }
  \label{fig:fr}
\end{figure}
 \begin{itemize}
 \item
 ابر ذخیره گر و ارتباطات مرکزی جامع :
 که همانند ابر مرکزی \lr{C-RAN} می باشد
 \item
 ابر کنترل گر مرکزی :که برای تکمیل عملکردهای کنترلی  می باشد و در \lr{HPN} ها قرار دارد
 \item
 ابر ارتباطات منطقی توزیع شده که در برنامه های محاسبات مهی و ابزار های این محاسبات قرار دارد.
 \item
  ابر ذخیره گر منطق توزیع شده:
  که همانند قبل در \lr{F-RAN} قرار دارد.
 \end{itemize}
 در این ساختار ، برای کاهش تاخیر ناشی از انتقال داده ها به ابر مرکزی ، ساختار های \lr{RRH} را دارای حافظه قرار می دهیم که برای ارتباطات محلی، به جای اینکه پردازش ها در \lr{BBU Pool} صورت بگیرد، بدون نیاز به انتقال به ابر مرکزی، درون \lr{RRH} ها انجام پذیرد. 
\end{itemize}
\subsection{\lr{xRAN}}
\lr{xRAN}
در سال ۲۰۱۶ با هدف استانداردسازی یک جایگزین انعطاف پذیر و باز برای \lr{RAN}
مبتنی بر سخت افزار سنتی بدست آمده است.
 در این ساختار، سه حوزه ی مهم مورد بررسی قرار گرفته است.
اولین حوزه ی مورد بررسی، جداسازی بخش
صفحه ی کنترل
 \LTRfootnote{control plane} از 
 صفحه ی کاربر
\LTRfootnote{user plane}
می باشد. حوزه ی دوم،
ساختن یک پشته نرم افزاری \lr{eNodeB} مدولار که از سخت افزار \lr{COTS} استفاده می کند، می باشد.
حوزه ی سوم مورد بررسی، انتشار رابط های باز شمال و جنوب است\cite{xran}.
در ادامه این سه حوزه به طور دقیق تر مورد بررسی قرار می گیرد\cite{xran1}.

\begin{itemize}
\item \textbf{ جداسازی بخش صفحه ی کنترل از 
صفحه ی کاربر}:
این انتقال صفحه ی کنترل، که قبلاً کاملاً به دستگاههای سخت افزاری \lr{RAN} متصل بود، به دستگاههای محاسباتی در دسترس امکان می دهد \lr{RAN} بتواند به عنوان یک استخر منطقی از ظرفیت ، با کارایی بیشتری کار کند.
نرم افزار \lr{eNodeB} از سخت افزار خاص فروشنده جدا می شود و الهام بخش نوآوری در هر دو نرم افزار و سخت افزار به صورت مشارکتی اما به طور مستقل است.
برنامه نویسی و کنترل زمان واقعی بی سابقه در زیرساخت های \lr{RAN} به دست آمده است، که به راحتی از برنامه های کاربردی تلفن همراه و خدمات تجاری پشتیبانی می کند.
\item \textbf{ساختن یک پشته نرم افزاری \lr{eNodeB} مدولار}:
رویکرد \lr{xRAN} به خوبی با طرح های مجازی سازی عملکرد شبکه حامل \lr{(NFV)} مطابقت دارد، و همچنین منجر به کنترل عملکرد ترافیک با کارایی بالا، مدیریت تداخل و کنترل منابع رادیویی روی سیستم عامل های استاندارد \lr{x86} و می شود.
\item \textbf{انتشار رابط های باز شمال و جنوب}: 
رابط های استاندارد و باز قابیت پشتیبانی از فروشنده های متعدد همکاری اثبات شده دارند. 
\lr{xRAN.org}
و اعضای آن به تصویب رساندن این رابط ها از طریق فرآیندهای استاندارد منجر به در دسترس قرار دادن معماری \lr{xRAN} و پشتیبانی مورد نیاز می شوند.

\end{itemize}
\subsubsection{مزایای ساختار \lr{xRAN}}
\begin{itemize}
\item 
جداسازی بخش صفحه ی کنترل از 
صفحه ی کاربر 
منجر به 
برنامه ریزی زمان واقعی بی سابقه و کنترل در زیرساخت \lr{RAN} می شود که به راحتی برنامه های کاربردی تلفن همراه و خدمات تجاری را پشتیبانی می کند.
\item 
یک پشته \lr{eNB} مدولار مبتنی بر نرم افزار ، منجر به امکان قرارگیری انعطاف پذیر توابع \lr{eNB} و کنترل ترکیبی آن با یک برنامه ریز امکان پذیر می شود تا بتواند زمان تاخیر متغیر در \lr{fronthaul} را کنترل کند.
\item 
رابط های مرزی جنوبی استاندارد، پیاده سازی شبکه با خرید سیستم از چندین شرکت متفاوت را امکان پذیر می سازد و رابط های شمال مرزی، برش کامل شبکه برای بهینه سازی \lr{QoE} \LTRfootnote{Quality of Experience} کاربر را فراهم می کند.
رابط های \lr{xRAN} به خوبی با لبه ابر حامل هماهنگ هستند و اجازه می دهد تا  محاسبه و ذخیره سازی منابع در شبکه تلفن همراه 
به صورت دینامیکی مدیریت شود.
\item 
این ساختار هزینه ی رشد ظرفیت دسترسی رادیویی و هزینه ی بهره برداری را کاهش می دهد.
\end{itemize}
\subsection{\lr{ORAN}}
معماری \lr{ORAN} برای ایجاد زیرساخت های \lr{RAN} نسل بعدی طراحی شده است.
معماری \lr{ORAN} با تکیه بر اصول هوشمندی و باز بودن، پایه و اساس ساخت \lr{RAN} مجازی بر روی سخت افزار آزاد ، با کنترل رادیویی ایجاد شده توسط هوش مصنوعی است که توسط اپراتورهای سراسر جهان پیش بینی شده است.
این معماری بر روی رابط های استاندارد و تعریف شده ای بنا شده است تا یک زنجیره اکوسیستم با قابلیت باز ایجاد کند که دارای پشتیبانی کامل از استانداردهای تبلیغ شده توسط \lr{3GPP} و سایر سازمان های استاندارد صنعت فراهم شود.
\begin{figure}[H]
  \centering
    \includegraphics[width=0.8\textwidth]{./fig/oran1}
  \caption{ساختار شبکه ی \lr{ORAN} \cite{oranWP}}
  \label{fig:ORAN}
\end{figure}
اتحاد \lr{ORAN} در جستجوی چشم انداز باز بودن و هوشمندی برای شبکه های بی سیم نسل بعدی و فراتر از آن است\cite{oranWP}.
\begin{itemize}
\item \textbf{باز بودن}:
ایجاد یک \lr{RAN} مقرون به صرفه نیاز به باز بودن ارتباط ها دارد.
رابط های باز برای فعال کردن فروشندگان و اپراتورهای کوچکتر به سرعت می توانند خدمات خود را معرفی کنند و یا اپراتورها را قادر می سازد تا شبکه را متناسب با نیازهای منحصر به فرد خود تنظیم کنند.
رابط های باز همچنین استقرار چند سازنده ای را قادر می سازد و اکوسیستم تأمین کننده رقابتی تر و پر جنب و جوش بیشتری را ایجاد می کند.
 همچنین نرم افزارهای منبع باز و طرحهای مرجع سخت افزار باعث نوآوری سریعتر و دموکراتیک تر می شود.
 \item \textbf{هوشمندی}
 شبکه ها با ظهور برنامه 5G پیچیده تر و متراکم تر شده و خواستار برنامه های غنی تر می شوند.
 برای کاستن این پیچیدگی نمی توان از ابزارهای سنتی انسانی  برای استقرار، بهینه سازی و بهره برداری از شبکه استفاده کرد.
 در نتیجه، شبکه ها باید خود متحرک شوندتا بتوانند از فن آوری های جدید مبتنی بر یادگیری برای خودکارسازی عملکرد شبکه های عملیاتی و کاهش \lr{OPEX} استفاده کنند.
 اتحاد \lr{ORAN} تلاش خواهد کرد تا از تکنیک های یادگیری عمیق در حال ظهور استفاده کند تا بتواند هر لایه از معماری \lr{RAN}  را به طور هوشمند پیاده سازی کند.
 پیاده سای هوشمند هم در مولفه ها و خم در سطح شبکه اعمال می گردد و منجر به تخصیص دینامیکی منابع رادیویی و بهینه سازی بازدهی شبکه می گردد.
 همراه با رابط های باز \lr{ORAN}، اتوماسیون حلقه بسته بهینه شده با هوش مصنوعی دست یافتنی است و دوره جدیدی را برای عملیات شبکه امکان پذیر می کند.
\item \textbf{روش های هوش مصنوعی \lr{AI}
 \LTRfootnote{Artificial Intelligent}
  منجر به هوشمند سازی بخش رادیویی با استفاده از نرم افزار تعریف شده \LTRfootnote{Software Defined} 
  می شود:}
   مفهوم 
  \lr{SDN}
  \LTRfootnote{software defined network}
  که مبنی بر جداسازی 
   بخش صفحه ی کنترل \lr{CP} از
   صفحه ی کاربر 
\lr{UP}
می باشد، در ساختار 
\lr{ORAN}
مورد بررسی قرار می گیرد.
این جداسازی منجر به بهبود 
\lr{RRM}
برای استفاده از زمان غیر واقعی و زمان نزدیک به واقعی در کنترلگر هوشمند شبکه ی دسترسی رادیویی \LTRfootnote{RAN Intelligent Controller} \lr{RIC} 
با استفاده از رابط های 
\lr{A1}
و
\lr{E2}
 می گردد.
همچنین 
منجر به جداسازی 
 \lr{CU}
 از 
 \lr{CP/UP}
 می شود
 که از طریق رابط \lr{E1} در \lr{3GPP } توسعه می یابد.
\item \textbf{مجازی سازی بخش \lr{RAN}}:
 ابری سازی RAN یکی از اصول مهم ساختار 
 \lr{ORAN}
  می باشد.
 اپراتورها برای پشتیبانی از شکافهای مختلف در شبکه، الزامات NFVI/VIM را برای تقویت سیستم عامل مجازی ارائه می دهند.
 به عنوان مثال: لایه ی بالا بین PDCP و RLC تقسیم می شود و لایه ی پایین در PHY تقسیم می شود.
\item \textbf{رابط های باز}:
معماری مرجع ORAN بر روی مجموعه ای از رابط های کلیدی بین چندین جزء جدا شده ی RAN ساخته شده است.
اینها شامل رابط های \lr{3GPP} پیشرفته 
(
\lr{F1}،
\lr{W1}،
\lr{E1}،
\lr{X2}،
\lr{Xn}
)
 برای قابلیت همکاری بین چندین شرکت مختلف تولید کننده است.
رابط های مشخص شده \lr{ORAN Alliance} شامل یک رابط fronthaul باز بین DU و RRU ، رابط \lr{E2} و یک رابط \lr{A1} بین لایه 
Orchestration/NMS 
است که شامل عملکرد
 غیر واقعی زمانی \LTRfootnote{non real time RIC} و عملکرد eNB / gNB حاوی عملکرد RIC نزدیک به زمان واقعی 
\LTRfootnote{near-real time RIC} 
است.
\item \textbf{سخت افزار جعبه سفید}:
برای بهره مندی کامل از مقیاسی از اقتصاد ارائه شده توسط یک رویکرد  محاسباتی باز، \lr{O-RAN Alliance } 
طرح های مرجع 
سخت افزاری و ایستگاه پایه به صورت جعبه سفید با کارایی بالا را مشخص می کند. 
سیستم عامل های مرجع از یک رویکرد جدا شده پشتیبانی می کنند و نقشه های مفصلی را برای معماری سخت افزار و نرم افزار ارائه می دهند تا هم BBU و RRU را فعال کنند. 
\item \textbf{نرم افزار منبع باز}:
اتحادیه ORAN ارزش انجمن هایی که منابع باز ارا‌ئه می دهند را درک کرده   
 و از آنها پشتیبانی می کند.
 بسیاری از مؤلفه های معماری ORAN به صورت منبع باز از طریق جوامع موجود تحویل داده می شود.
 این مؤلفه ها عبارتند از: کنترلر هوشمند RAN ، پشته پروتکل ، پردازش لایه PHY و بستر مجازی سازی.
  چارچوب نرم افزار منبع باز ORAN نه تنها رابط های 
(
\lr{F1}،
\lr{W1}،
\lr{E1}،
\lr{E2}،
\lr{X2}،
\lr{Xn}
)
  را پیاده سازی می کند ، بلکه انتظار دارد که طراحی مرجع را برای نسل بعدی RRM با هوش جاسازی شده ارائه دهد تا RIC را امکان پذیر کند.
\end{itemize}

\lr{ORAN}،
 المانهای شبکه ی دسترسی رادیویی را مجازی می کند، آنها را جدا کرده و رابط های باز مناسب را 
برای اتصال این عناصر
تعیین می کند. همچنین، 
\lr{ORAN}
از روشهای یادگیری ماشین برای هوشمندسازی لایه های 
\lr{RAN}
 استفاده می نماید. 
 در ساختار نوآورانه ی 
 \lr{ORAN}
 نرم افزار قابل برنامه ریزی 
 \lr{RAN}
 از سخت افزار جدا می شود.
  یکی از مهم ترین خصوصیات
  \lr{ORAN}
  رابط کاربری باز است که به اپراتورهای موبایل این قابلیت را می دهد تا بتوانند سرویس های مورد نیاز خود را تعریف نمایند.

در ساختار
\lr{ORAN}،
واحد توزیع شده \lr{DU}،
نود منطقی می باشد که شامل لایه های 
\lr{RLC}
،
\lr{MAC}،
و
\lr{High-PHY}
است.
علاوه بر این، واحد مرکزی 
\lr{CU}
نود منطقی است که شامل لایه های 
\lr{RRC}،
\lr{SDAP} 
و 
\lr{PDCP}
می باشد.
نود منطقی واحد رادیویی
\lr{RU}
نیز، شامل لایه ی 
\lr{LOW-PHY}
و بخش پردازش رادیویی می باشد.
\lr{ORAN}
،
رابطهایی از جمله رابط 
\lr{fronthaul}
باز را شامل می شود که بخش \lr{DU} را به \lr{RU} متصل می نماید
(رابط 
\lr{E2}). 
همچنین
 رابط \lr{A1}
 بین لایه ی 
  \lr{orchestration/NMS}
  که شامل 
  تابع غیر واقعی زمان است و 
  \lr{eNB/qNB}
  که شامل تابع نزدیک به زمان است. 


  با افزایش ترافیک تلفن همراه، شبکه های تلفن همراه و تجهیزاتی که آنها را اجرا می کند باید نرم افزاری تر ، مجازی، انعطاف پذیر، هوشمند و کارآمدتر شوند.
اتحادیه ی ORAN متعهد است در حال تکامل شبکه های دسترسی رادیویی باشد که باعث می شود آنها نسبت به نسلهای قبل بازتر و باهوش تر شوند.
تجزیه و تحلیل در زمان واقعی که توسط سیستم های یادگیری ماشین تعبیه شده است و ماژول های پایانی هوش مصنوعی را هدایت می کند، باعث تقویت هوش شبکه می شود.
عناصر شبکه مجازی با رابط های باز و استاندارد، جنبه های اصلی طرح های مرجع توسعه یافته توسط اتحادیه ی ORAN خواهد بود.
فن آوری های موجود از عناصر شبکه منبع باز و جعبه سفید، نرم افزار و اجزای سخت افزاری مهم این طرح های مرجع خواهد بود.
\section{مجازی سازی توابع شبکه}
برای بهبود سرویس دهی در نسل پنجم مخابرات، جداسازی المان های نرم افزاری و سخت افزاری شبکه صورت گرفته است و به عنوان 
مجازی سازی توابع شبکه (\lr{NFV}) \LTRfootnote{network function virtualization}
معرفی شده است.
  حال توابع شبکه ی مجازی
  \lr{VNF}
  \LTRfootnote{Virtual network function} ،
  بلوکهای توابع سیستم هستند.
در نسل پنجم مخابرات 
  انتظار می رود که
   میزبان چندین سرویس
   با نیازهای مختلف به طور همزمان
    باشند.
    ایده اصلی NFV جداسازی تجهیزات شبکه فیزیکی از توابع اجرا شده بر روی آنها است. این بدان معنی است که یک عملکرد شبکه - مانند فایروال - می تواند به عنوان نمونه ای از نرم افزارهای ساده به فراهم آورندگان سرویس (SP) \LTRfootnote{Service Provider} ارسال شود.
    این امر امکان ادغام بسیاری از انواع تجهیزات شبکه بر روی سرورهای با حجم بالا ، سوئیچ ها و انبارها را فراهم می کند ، که می توانند در مراکز داده، نودهای شبکه توزیع شده و در محل کاربر نهایی قرار بگیرند.
    به این ترتیب، یک سرویس خاص می تواند به مجموعه ای از توابع شبکه مجازی (VNFs) تجزیه شود، که می تواند در نرم افزارهایی که روی یک یا چند سرور فیزیکی استاندارد در صنعت قرار دارند، اجرا شود.
    سپس VNF ها ممکن است در مکانهای مختلف شبکه (به عنوان مثال، با هدف معرفی خدمات هدفمند به مشتریان در یک موقعیت جغرافیایی خاص) جابجا شده و خدمات رسانی کنند، بدون اینکه لزوماً به خرید و نصب سخت افزار جدید نیاز داشته باشند.
    NFV به 
     هاSP
   با انعطاف پذیری بیشتری وعده می دهد تا بتواند بیشتر قابلیت ها و خدمات شبکه خود را به کاربران و سایر خدمات باز کنند و امکان استقرار یا پشتیبانی از سرویس های جدید شبکه را  به طور سریعتر و ارزانتر داشته باشند تا بتوانند  سرویس بهتری داشته باشند.
   برای دستیابی به این مزایا، NFV مسیر را برای کاهش اختلافات در نحوه ارائه خدمات شبکه در مقایسه با عملکرد فعلی ایجاد می کند. خلاصه این ویژگی ها به شرح زیر است
   \cite{NFV}.
 \begin{itemize}
  \item \textbf{جدا سازی بخش نرم افزار از سخت افزار}:
از آنجا که عنصر شبکه، ترکیبی از سخت افزارها و نرم افزارهای یکپارچه نخواهد بود، تکامل هر دو مستقل از یکدیگر می باشد.
که این ویژگی منجر به جداسازی زمان بندی توسعه و نگهداری نرم افزار و سخت افزار می گردد.
\item \textbf{استقرار عملکرد شبکه انعطاف پذیر:}
جدا کردن نرم افزار از سخت افزار به تنظیم مجدد و به اشتراک گذاری منابع زیرساختی کمک می کند،
بنابراین، سخت افزار و نرم افزار، باهمدیگر
می توانند در زمان های مختلف عملکردهای مختلفی را انجام دهد که به اپراتورهای شبکه کمک می کند تا خدمات جدید شبکه را سریعتر در همان پلت فرم فیزیکی مستقر کنند.
بنابراین،
مؤلفه ها را می توان در هر دستگاه با قابلیت NFV در شبکه قرار داد و اتصالات آنها به روشی انعطاف پذیر تنظیم کرد.
\item \textbf{مقیاس گذاری پویا}:
  جداشدن عملکرد شبکه به اجزای نرم افزاری  نعطاف پذیری بیشتری را برای  عملکرد واقعی VNF به روشی پویاتر، 
   با توجه به ترافیک واقعی که اپراتور شبکه برای تأمین ظرفیت نیاز دارد،
  فراهم می کند.
\end{itemize}  
 \subsection{ساختار شبکه }
 VNF
 ها برای به اشتراک گذاشتن منابع مختلف فیزیکی و مجازی زیرساخت ها می توانند مستقر و مجدداً تنظیم شوند ، تا مقیاس پذیری و کارآمدی سیستم را تضمین کنند که منجر می شود SP ها به سرعت سرویس های جدید را در سیستم وارد کنند.
 به طور کلی ، سه مؤلفه اصلی در NFV وجود دارد:
 خدمات ، NFVI و مدیریت NFV و orchestration
 \LTRfootnote{NFV-MANO}
 که در شکل \eqref{fig:NFV} دیده می شود.
 \begin{figure}
  \centering
    \includegraphics[width=0.8\textwidth]{./fig/NFV}
  \caption{ساختار NFV \cite{NFVArch}}
  \label{fig:NFV}
\end{figure} 
این مؤلفه ها به شرح زیر بیان می گردد\cite{NFVArch}.
\begin{enumerate}
\item 
خدمات: یک سرویس مجموعه ای از VNF ها است که می توانند در یک یا چند ماشین مجازی پیاده سازی شوند.
در بعضی مواقع ، VNF ها می توانند در ماشینهای مجازی نصب شده در سیستم عامل یا سخت افزار بطور مستقیم نصب شوند. آنها توسط سرپرستان بومی یا مانیتورهای ماشین مجازی اداره می شوند.
معمولاً توسط یک سیستم مدیریت عناصر \LTRfootnote{Element Management System} (EMS)،
 که مسئولیت ایجاد، تنظیمات، نظارت، عملکرد و امنیت آن است، اداره می شود.
 EMS 
 اطلاعات ضروری مورد نیاز سیستم پشتیبانی عملیات \LTRfootnote{Operations Support System}(OSS) را در یک محیط SP فراهم می کند.
 OSS
  سیستم مدیریت عمومی است، که  همراه با سیستم پشتیبانی از تجارت 
\LTRfootnote{Business Support System}  
  (BSS)
  ، به ارائه دهندگان کمک می کند تا چندین سرویس ارتباطی از راه دور را به کار ببندند و مدیریت کنند.
  (به عنوان مثال سفارش ، صورتحساب ، تمدید ، عیب یابی مشکل و غیره).
مشخصات NFV بر ادغام با راه حل های موجود OSS / BSS متمرکز است.
\item NFVI
:
زیرساخت های NFV تمام منابع سخت افزاری و نرم افزاری را که شامل محیط NFV است ، پوشش می دهد.
NFVI شامل اتصال شبکه بین مکان ها ، به عنوان مثال ، بین
مراکز داده و ابرهای ترکیبی عمومی یا خصوصی است.
منابع فیزیکی به طور معمول شامل محاسبات، ذخیره سازی و سخت افزار شبکه است که وظیفه ی آن پردازش، ذخیره سازی و اتصال VNF ها از طریق لایه مجازی سازی است و دقیقاً بالای سخت افزار قرار دارد و منابع فیزیکی را چکیده می کند (که به صورت منطقی تقسیم شده و به VNF ها اختصاص می یابد).
هیچ راه حل خاصی برای استقرار NFV وجود ندارد. در عوض معماری NFV می تواند از یک لایه مجازی سازی موجود مانند Hypervisor با ویژگی های استاندارد که منابع سخت افزاری را به راحتی استخراج می کند و آنها را به VNF ها اختصاص می دهد ، استفاده کند.
وقتی این پشتیبانی در دسترس نباشد ، اغلب ، لایه مجازی سازی از طریق یک سیستم عامل حاصل می شود که نرم افزاری را در بالای سرور غیر مجازی یا با اجرای یک VNF به عنوان یک برنامه اضافه می کند.
\item NFV-MANO
:
NFV-MANO
 از این موارد تشکیل شده است:
 orchestrator
 ،
 مدیران VNFs و مدیران زیرساخت مجازی.
 چنین بلوکی عملکردهای مورد نیاز برای کارهای مدیریتی را که برای VNF ها اعمال می شود، به عنوان مثال تهیه و پیکربندی را  ارائه می دهد. 
 NFV-MANO شامل orchestration و مدیریت چرخه منابع فیزیکی یا مجازی است که از مجازی سازی زیرساخت ها و مدیریت چرخه VNF ها پشتیبانی می کند.
 همچنین شامل بانکهای اطلاعاتی است که برای ذخیره اطلاعات و مدل های داده استفاده می شود که ویژگی های چرخه عمر توابع، خدمات و منابع را تعریف می کند.
 NFV-MANO روی کلیه وظایف مدیریتی مجازی سازی ویژه لازم در چارچوب NFV تمرکز دارد.
 علاوه بر این ، این چارچوب رابط هایی را تعیین می کند که می توانند برای ارتباطات بین مؤلفه های مختلف
  NFV MANO، 
 و همچنین هماهنگی با سیستم های سنتی مدیریت شبکه (یعنی OSS و BSS) مورد استفاده قرار گیرند تا امکان عملکرد هر دو VNF و کارکردهای اجرا شده بر روی تجهیزات فراهم شود.
 به طور خلاصه، اگر برش شبکه با استفاده از فایروال و DPI مستقر شده باشد، آنگاه NFV-MANO وظیفه دارد بگوید این VNF ها در کجای شبکه فیزیکی قرار دارند. همچنین این VNF ها توسط EMS و همان MANO کنترل می شوند.
\end{enumerate}
\section{شبکه تعریف شده نرم افزار (SDN)}
بنیاد شبکه باز 
\LTRfootnote{Open Networking Foundation}
(ONF) 
یک مجموعه ای است که به توسعه، استاندارد سازی و تجاری سازی SDN 
\LTRfootnote{Software Defined Network} 
پرداخت.
ONF
 به طور صریح و دقیق SDN را بدین صورت تعریف کرد:
 شبکه تعریف شده توسط نرم افزار (SDN) یک معماری شبکه است که کنترل شبکه از ارسال جدا می شود و به طور مستقیم قابل برنامه ریزی است.
 SDN توسط دو ویژگی تعریف می شود ، یعنی جدا شدن صفحه ی کنترل و داده و قابلیت برنامه ریزی در صفحه کنترل.
 با این وجود ، هیچ یک از این دو امضای SDN در معماری شبکه کاملاً جدید نیستند
 \cite{SDN1}.
 SDN 
 در اصل یک الگوی شبکه سازی متمرکز است که در آن هوش شبکه (یعنی عملکرد کنترل یا صفحه کنترل) به طور منطقی در یک یا مجموعه ای از موجودیت های کنترل (یعنی کنترل کننده های SDN) متمرکز می شود در حالی که صفحه ی انتقال داده،  ساده و چکیده شده برای برنامه های کاربردی می باشد و سرویس های شبکه درخواست خود را از طریق کنترل کننده های SDN
 بیان می کنند.
 در حالی که در مورد هسته اصلی شبکه موبایل LTE ، EPC صحبت می کنیم ، مفهوم SDN برای دستیابی به جدایی واضح بین صفحات کنترل و کاربر در اشخاص SGW و PGW استفاده می شود.
 با تقسیم دروازه به این روش (یعنی از SGW به SGW-C و
SGW-U 
و از PGW به PGW-C و (PGW-U 
  مقیاس بندی این مؤلفه ها به طور مستقل امکان پذیر است و طیف وسیعی از گزینه های استقرار را نیز ممکن می کند.
  
 پروتکل مورد استفاده بین صفحه ی کنترل و صفحه ی کاربر می تواند یا افزونه پروتکل موجود OpenFlow باشد، که توسط گروه کاری بی سیم و موبایل ONF (WMWG)با رابط های جدید، یعنی Sxa و Sxb ساخته می شود، که توسط  \lr{3GPP CUPS} تعریف و مشخص می شوند\cite{SDN2}. 
 جداسازی صفحه ی کنترل از کاربر منجر به کنترل بیشتر شبکه بوسیله ی برنامه می گردد که منجر به بهبود تنظیمات و کارآمدی سیستم می گردد.
 SDN
  با ساختار برنامه ریزی شده ی قوانین ترافیک، جایگزین امیدوار کننده ای برای فرماندهی ترافیک ارائه می دهد.
  ساختار SDN در شکل \eqref{fig:SDN} آورده شده است.
  \begin{figure}
  \centering
    \includegraphics[width=0.8\textwidth]{./fig/SDN}
  \caption{ ساختار SDN \cite{SDN3}}
  \label{fig:SDN}
\end{figure} 
  در این ساختار ۳ لایه ی مختلف وجود دارد که در ادمه بیان می کنیم\cite{SDN3}.
  \begin{enumerate}
  \item لایه ی برنامه :
  این لایه مجموعه ای از برنامه های متمرکز بر خدمات شبکه را پوشش می دهد و آنها عمدتا برنامه های نرم افزاری هستند که با لایه کنترل ارتباط برقرار می کنند.
  \item لایه ی کنترل:
  به عنوان هسته اصلی SDN ، لایه کنترل از یک کنترلر متمرکز تشکیل شده است که منطقاً نمای شبکه جهانی و پویا را حفظ می کند ،  که از لایه برنامه درخواست می کند و دستگاه های شبکه را از طریق پروتکل های استاندارد مدیریت می کند.
  \item لایه ی داده:
  این لایه، زیرساخت ها شامل سوئیچ ها، روترها و لوازم شبکه می باشد. در زمینه SDN ، این دستگاه ها قابل برنامه ریزی هستند و از رابط های استاندارد پشتیبانی می کنند.
  \end{enumerate}
 \section{برش شبکه}
 پیش بینی می شود شبکه های \lr{5G} چندین سرویس را با نیازهای مختلف به طور همزمان پشتیبانی کند.
 برش شبکه
 \LTRfootnote{Network Slicing}
به عنوان راه حلی برای چنین تقاضا در نظر گرفته شده است.
یک برش شبکه، یک شبکه منطقی \lr{end-to-end} است که خدمات  با نیازهای خاص را ارائه می دهد.
 چندین برش شبکه
در یک زیرساخت یکسان
  اجرا و مدیریت می شوند و
به طور مستقل کار می کنند.
برش شبکه
 با هدف تقسیم منطقی مجموعه توابع و منابع شبکه در یک نهاد شبکه در نظر گرفته شده است که مطابق با خواسته های فنی یا تجاری خاص می باشد.
 با خرد کردن یک شبکه فیزیکی به چندین شبکه منطقی، برش شبکه می تواند از خدمات متناسب با تقاضا برای سناریوهای برنامه مشخص در همان زمان با استفاده از همان شبکه فیزیکی پشتیبانی کند.
با استفاده از برش شبکه، منابع شبکه می توانند به صورت پویا و کارآمد به برش های شبکه منطقی با توجه به خواسته های QoS مربوطه اختصاص داده شوند\cite{NS1}. 

\begin{figure}
  \centering
    \includegraphics[width=0.8\textwidth]{./fig/NS}
  \caption{سه ساختار برش شبکه \cite{NS2}}
  \label{fig:NS}
\end{figure} 
پیاده سازی های مختلفی از برش شبکه وجود دارد که شامل برش هسته ی شبکه، برش \lr{RAN} و برش هر دو بخش می باشد\cite{NS2}.
\begin{itemize}
\item \textbf{برش هسته:}
هسته ی شبکه (CN) \LTRfootnote{core network}
به عنوان برش های شبکه، مجازی سازی می شوند که با ویژگی هایی مانند ویژگی های قابل برنامه ریزی و قابل اعتماد بودن که شامل مدیریت حرکت و تأیید اعتبار می باشد.
برش های شبکه فقط در CN وجود دارد.
  بنابراین، نه RAN و نه تجهیزات کاربر (UE) برای CNهای برش داده شده نیاز به تنظیم ویژه ندارند.
  در برش هسته ی شبکه، برش تنها در بخش هسته ی شبکه است و تمام واسط ها و فرآیندها، بدون تغییر باقی می مانند
  به جز مواردی که در ابتدا UE ها به شبکه ها وصل می شوند ، زیرا UE ها باید به برش صحیح CN ها اختصاص داده شوند.
\item \textbf{
برش شبکه ی دسترسی رادیویی:
}
برخلاف برش CN،
برش های RAN روی سخت افزار رادیویی و استخر منابع باند پایه، به نام یک سطح بی سیم، اجرا می شوند که دارای کشش کمتری نسبت به زیرساخت مجازی بالغ شده در CNها هستند.
با چند BS منطقی، برش های RAN پارامترهای مختلفی از رابط های هوا (به عنوان مثال ، طول نماد ، فاصله زیر حامل ، طول پیشوند چرخه و پارامترهای درخواست تکرار خودکار هیبریدی \LTRfootnote{HARQ}) را اعمال می کند.
علاوه بر این، پارامترهای دیگری مانند انتخاب سلول و آستانه انتقال، و همچنین سیاست های انتقال هماهنگ را می توان برای هر برش تعریف کرد تا یک تجربه بی سیم برجسته را به کاربران ارائه دهد.
\item \textbf{برش هسته و شبکه ی دسترسی رادیویی}:
در این سناریو، هر برش از RAN به یک برش از هسته متصل می شود، بنابراین اپراتورها می توانند یک شبکه منطقی انتهای به مشتریان ارائه دهند.
روش انتخاب برش همان روش برش RAN است، بنابراین کاربران پس از دسترسی به سیستم، نیازی به انتخاب برش CN ندارند.
این مدل از برش مزایای هر دو مدل از برش را باهم دارد.
 در نتیجه این روش برش، قادر به برنامه ریزی ویژگی های CN و همچنین دارای قابلیت تغییر رابط های هوایی RAN
 می باشد.

\end{itemize}
\section{نتیجه گیری}
  در این فصل ابتدا مروری بر تاریخچه ی مخابرات و ۵ نسل مخابراتی شد. سپس ساختار های مختف دسترسی رادیویی به طور خلاصه بیان شد و در نتیجه ی آن ساختار CRAN که ساختار ابری است تعریف شد. سپس ساختار xRAN
  مورد توجه قرار گرفت و در نهایت ساختار ORAN 
  که ترکیب و تکاملی از CRAN و xRAN می باشد مورد توجه قرار گرفت.
  
 بعد از بیان ساختارهای رادیویی، ساختار هسته ی شبکه را در نسل پنجم بیان کردیم که شامل 
 NFV و SDN 
 می باشد که منجر به جداسازی صفحه ی کنترل از کاربر می شود و سیستم هوشمندتر همراه با قابلیت برنامه ریزی بیشتر می گردد.
 در ادامه برش شبکه در بخش رادیویی و هسته و هردو مورد توجه قرار گرفته شد.
\section{خلاصه ای از فصل های آتی}
در فصل دوم مروری بر ادبیات پیشین و خلاصه ای از مدل سیستم مقالات موجود، بیان می گردد.
در فصل سوم مدل سیستم در نظر گرفته بیان می شود و صورت مسئله به نمایش گذاشته می شود و روش های حل آن بیان می گردد.
در فصل چهارم نتایج شبیه سازی قرار داده می شود.
در فصل پنجم نیز نتیجه گیری و کارهای آتی مورد نظر بیان می شود.   
 