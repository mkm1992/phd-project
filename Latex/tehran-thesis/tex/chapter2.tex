\chapter{مروری بر کارهای پیشین}

\section{مقدمه}
در این فصل، به مرور کارهای گذشته می پردازیم.
در ابتدا مروری بر مقالاتی که از برش شبکه استفاده کرده اند می پردازیم.
برش شبکه در سه بخش رادیویی، بخش هسته و هردو بخش هسته و رادیویی صورت می گیرد.
سپس مقالاتی در زمینه ی سیر عبور از شبکه های دسترسی رادیویی ابری به شبکه های دسترسی رادیویی باز را بررسی می نماییم. 
\section{برش شبکه}
یک برش شبکه یک شبکه منطقی انتها به انتهای مستقل است که بر روی یک زیرساخت فیزیکی مشترک کار می کند و قادر به ارائه خدمات می باشد.
برش شبکه قادر به سرویس دهی به خدمات مختلف می باشد.
در این بخش در مورد برش شبکه در بخش رادیویی و هسته و هردو صحبت می نماییم.
%\subsection{برش شبکه در شبکه های دسترسی رادیویی }
برش RAN یکی از کلیدهای اصلی برای انعطاف پذیری سفارشات ومدیریت مجازی سازی ایستگاه پایه می باشد تا بتواند منابع رادیویی را در میان سرویس های مختلف تقسیم کرده و منجر به سازگاری اپراتورها و برطرف کردن نیاز سرویس ها گردد.
%در ادامه مروری بر این مدل مقالات می نماییم.
%\subsubsection{  برش شبکه در شبکه های دسترسی رادیویی ابری و متجانس و مهی}

شبکه های دسترسی رادیویی ابر (C-RAN) به عنوان یک چارچوب امیدوار کننده برای سیستم های ارتباط بی سیم نسل پنجم ظاهر شده اند.
از آنجا که آنها می توانند پیچیدگی رمزگشایی، مصرف انرژی و دخالت های ناشی از افزایش تراکم تلفن همراه را کاهش دهند\cite{cranInt}.
در ادامه در مورد برش شبکه در بخش رادیویی شبکه های دسترسی رادیویی ابری صحبت می کنیم.

در مقاله ی \cite{lee2018dynamic}
برش شبکه به صورت دینامیکی در بخش رادیویی مورد بررسی قرار گرفته شده است.
چارچوب طرح برش شبکه شامل یک سطح بالاتر، که مدیریت کنترل پذیرش، تشکل و تخصیص منابع باند پایه و یک سطح پایین تر، که تخصیص منابع رادیو در میان کاربران می باشد.
در این مدل فرض می کنیم که هر سرویس دارای شبکه اصلی خود (یا قطعه اصلی شبکه) است که به H-CRAN متصل می شود.
سلول بزرگ RRH (M-RRH) و سلولهای کوچک RRHs (S-RRHs) به ترتیب از طریق پیوندهای پشتی و fronthaul به یک استخر ابر
BBU  متصل می شوند.
همچنین ، تقسیم C/U در مدل سیستم فرض می شود، که به موجب آن صفحات کنترل و داده از هم جدا می شوند به گونه ای که صفحات کنترل توسط M-RRH در شبکه مدیریت می شود.
\begin{figure}[H]
  \centering
    \includegraphics[scale = 0.7]{./fig/dynamicNS}
  \caption{روند برش شبکه\ \cite{lee2018dynamic}}
  \label{fig:dns}
\end{figure}
%برای حل این مسئله، می بایست مسئله را به بخش های کوچکتر تقسیم کرده و حل نماییم.
همانطور که در شکل \eqref{fig:dns} 
مشخص شده است ابتدا پذیرش کاربر مورد توجه قرار می گیرد و سپس کاربر به RRH متصل می شود و پس از آن ظرفیت BBU به آن تخصیص می دهد که تا این بخش از کار در سطح بالا قرار داریم.
در سطح بالا، یک مسئله کنترل پذیرش با برنامه نویسی پویا می باشد که در آن پیچیدگی را می توان تنظیم کرد.
همچنین مسئله ی ارتباط کاربر با استفاده از یک الگوریتم حریص با پیچیدگی کم بهینه و حل می شود.
 مسئله ی تخصیص ظرفیت BBU نیز فرموله شده و با برنامه ریزی خطی حل می شود.
 حال وارد الگوریتم سطح پایین تر می شویم که تخصیص توان و منبع فیزیکی می باشد.
برای مساله ی سطح پایین تر، مشکل تخصیص منابع به عنوان یک مشکل برنامه نویسی mixed-integer غیر محدب است که با استفاده از روش دوگانه لاگرانژ حل می شود.

در مقاله ی 
\cite{larsen2018fronthaul,costanzo2018network}
برش شبکه در شبکه های دسترسی رادیویی ابری متوجه قرار گرفته است. 
در بخش fronthaul شبکه مشکلاتی از قبیل پیچیدگی شبکه و محدودیت نرخ وجود دارد که در برش شبکه، منجر به بهبود آن می شود.
علاوه بر این ، C-RAN می تواند مجازی سازی مجموعه ای از توابع RAN را امکان پذیر کرده و راه را برای اصطلاحاً RAN مجازی باز کند. با این کار می توان چندین شبکه مجازی یا برش ایجاد کرد.
 مقاله ی
\cite{larsen2018fronthaul}
نشان داده است که با استفاده از برش شبکه و برخورداری از سوییچ بسته در fronthaul
مزایای زیادی را خواهد داشت که از جمله برخورداری از تقسیمات عملکردی مختلف خواهد بود. همچنین از معایب این کار تاخیر نسبتا اندکی خواهد بود.

در مقاله ی 
\cite{fran}
برش شبکه در بخش رادیویی برای ساختار مه \LTRfootnote{Fog Radio Access Network}
 یا F-RAN
  در نظر گرفته شده است که در آن دو نمونه برش شبکه برای هات اسپات و سناریوهای وسیله نقلیه با زیرساخت مربوط تنظیم می شود. به طور خاص ، چارچوب برای برش RAN به عنوان یک مشکل بهینه سازی مشترک برای مقابله با ذخیره کردن و انتخاب حالت است.
  با توجه به خواسته های کاربران مختلف و منابع محدود، پیچیدگی مسئله بهینه سازی اصلی بسیار زیاد است و همین امر باعث می شود که رویکردهای بهینه سازی سنتی به طور مستقیم سخت باشد.
 برای مقابله با این معضل ، یک الگوریتم یادگیری تقویت عمیق ارائه شده است ، که ایده اصلی آن این است که سرور ابر تصمیمات صحیحی را در زمینه ذخیره محتوا و انتخاب حالت برای به حداکثر رساندن عملکرد پاداش در وضعیت کانال پویا و وضعیت حافظه نهان ارائه می دهد.
%\subsubsection{برش شبکه در شبکه های دسترسی رادیویی }

در مقاله ی 
\cite{ranSlice, ranSlice1}
اجرای مفهوم برش در سطح RAN توسط اپراتور شبکه تلفن همراه (MNO) برای پاسخگویی به نیازها می باشد. همچنین مساله ی تخصیص منابع (در اینجا پهنای باند) مورد توجه قرار گرفته شد.
چالش های پیش رو برش RAN نیز مورد بررسی قرار گرفته است که یکی از چالش ها
شامل طراحی و مدیریت چندین برش در زیرساخت مشترک به روشی کارآمد و در عین حال ضمانت SLA توافق شده برای هر یک از آنها است.
این چالش ما را نیازمند مفهوم ایزولاسیون برش می کند.

در مقاله ی 
\cite{ranslice2}
برش در بخش RAN مورد توجه قرار گرفته است.
همچنین
در این مقاله یک برنامه تخصیص منابع پویا، با هدف به طور مشترک بهینه سازی مصرف برق و تخصیص پهنای باند در حالی که رضایت از تأخیر مربوطه برای ورود ترافیک پراکنده uRLLC و کیفیت خدمات eMBB را تا حد ممکن ارائه می دهد، پیشنهاد می کند.
طرح پیشنهادی براساس کنترل بهینه توان برای تخصیص منابع آگاه از تأخیر است.
در نتیجه در این سیستم هدف مینیمم کردن توان با شروط برآورده شدن شروط پهنای باند و  تاخیر که با شرط صف پردازش نشان داده، می باشد.
%\subsection{برش شبکه در بخش هسته  }

در مقاله ی 
\cite{onet}
اجرای عملی برش شبکه را پیشنهاد شده است.
در مدل پیشنهادی ، نویسندگان فرض می کنند که در هر شکاف زمانی مشخص ، کاربران فقط می توانند یک برش شبکه واحد را درخواست کنند.
در اینجا تابع هدفی بر اساس نسبت میزان منابع اختصاص داده شده به کاربرن در هر زمان t به ظرفیت کل منابع مشخص شده است و هدف بیشینه سازی آن  می باشد.  
مدل پیشنهادی بر اساس مسئله \lr{multi armed bandit} ساخته شده است و نویسندگان سه نوع  آن را برای حل جنبه های مختلف تخصیص برش شبکه معرفی کرده اند.
آنها با استفاده از MATLAB مدل بهینه سازی را شبیه سازی کرده و نتایج را با یک الگوریتم حریص مقایسه کردند. آنها همچنین اثبات مفهوم برش شبکه را ارائه دادند.

برش شبکه یکی از فناوری های کلیدی است که به شبکه های 5G اجازه می دهد منابع اختصاصی به صنایع مختلف (خدمات) ارائه دهند.
در مقاله ی
\cite{li2019latency}
نویسندگان یک روش تخصیص منابع (تأخیر بهینه) برای برشهای شبکه حمل و نقل 5G برای پشتیبانی از خدمات URLLC ارائه داده اند.
آنها ویژگی های منبع شبکه و ویژگی های توپولوژی تخصیص منابع در تقسیم شبکه را معرفی کردند.

در 
\citep{vnf1,coreSlice}
ایزوله کردن برش شبکه ی هسته 
مورد توجه قرار گرفته است.
\citep{vnf1}
برای کاهش تأثیر حملات DDoS در احراز هویت برش ، از ایزوله کردن برش شبکه ی هسته استفاده شده و حل آن با ترکیبی از شبیه سازی و یک آزمایش عملی ارزیابی شده است.
نویسندگان
\citep{coreSlice}
دو چالش مهم برش شبکه در بخش هسته مورد توجه قرار داده اند که شامل ایزوله کردن برش شبکه و تضمین میزان تاخیر انتها به انتها می باشد.
در این مقاله، مساله ی بهینه سازی به صورت 
\lr{mixed integer linear programming}
 می باشد که
 تابع هدف درخواست های برش ورودی را به سروری که کمترین میزان استفاده از آن شده است، اختصاص داده و مسیری را با حداقل تأخیر پیدا می کند. خروجی این مساله VNF ها را به سرور اختصاص می دهد.  
%\subsection{برش شبکه در هسته و بخش رادیویی}

در این دسته مقالات، سرویس ها به دو بخش تقسیم می شوند در بخش اول سرویس هایی که  نسبت به تاخیر حساسند و دسته ی دوم سرویس هایی که نسبت به نرخ انتقال حساسند. همچنین در برخی مقالات هر دو ویژگی برای یک سرویس مد نظر می باشد.
در این مدل های سیستم، تاخیر با استفاده از M/M/1 در ساده ترین حالت یا برای نزدیک تر شدن به حالت حقیقی از M/D/1 نیز استفاده می شود. می توان در این مدل ها تاخیر را کمینه و نرخ انتقال را بیشینه کرده و یا 
برای کاربران نرخ را از حد مورد نیاز بیشتر و تاخیر را کمتر از حد مورد نیاز فرض کرد\cite{frdl,luong2018novel,luong2018novel1,guo2016exploiting}.
 \begin{figure}%[H]
  \centering
    \includegraphics[scale = 0.7]{./fig/Delay}
  \caption{مدل پردازشی شبکه صف \cite{frdl}.}
  \label{fig:Delay}
\end{figure}
همانطور که در شکل \eqref{fig:Delay}، مشخص است، در این شبکه برای هر بخش تعدادی VNF قرار دارند که پردازش ها را انجام می دهند. در مسیر لینک پایین
بسته ها با نرخ $\alpha$ به صف های مختلف وارد شده و پس از پردازش با همدیگر ادغام شده و سپس بسته ی هر کاربر از طریق وایرلس منتقل می شوند.
در این پردازش ها، از روش M/M/1 استفاده شده است.
در این مدل مقالات اشاره ی مستقیم به برش شبکه نشده است ولی
در آنها ترکیبی از مفهوم برش RAN و Core به چشم می خورد.
\section{رفتن به سمت شبکه های دسترسی رادیویی باز}
در فوریه 2018 ، شبکه دسترسی رادیویی آزاد (O-RAN) با ادغام  xRAN و اتحاد C-RAN برای ایجاد سطح جدیدی از باز بودن در شبکه دسترسی رادیویی ایجاد شد که از نسل 5G و 6G پشتیبانی می کند.
هدف اصلی O-RAN افزایش عملکرد RAN از طریق عناصر شبکه مجازی و واسط های باز است که دارای هوش در RAN است.
صراحت و هوش دو ستون اصلی تلاشهای انجام شده توسط اتحاد O-RAN است که یک نیروی جهانی متشکل از بیش از 160 شرکت کننده از فروشندگان بزرگ ، شرکت های کوچک و متوسط ، اپراتورهای شبکه ، مبتدیان و مؤسسات دانشگاهی است
\cite{oranpaper}
.

در مقاله ی 
\cite{oranInt}
 مقدمه ای در مورد مفاهیم، اصول و الزامات \lr{Open RAN} که توسط اتحاد O-RAN مشخص شده، بیان شده است.
 در این مقاله،
 به منظور نشان دادن نقش هوش در O-RAN، طرح مدیریت منابع رادیویی هوشمندی را برای رسیدگی به ازدحام ترافیک و نشان دادن اثربخشی آن در یک مجموعه داده در دنیای واقعی پیشنهاد شده است.
 یک معماری سطح بالا از این سناریوی استقرار که سازگار با الزامات O-RAN است نیز مورد بحث قرار گرفته است. مقاله با چالشهای کلیدی فنی و مشکلات باز برای تحقیقات و توسعه آینده به پایان می رسد.
 
در مقاله ی
\cite{c2o}
تعاریف عمومی، ویژگی های اساسی و روند تحقیقاتی فعلی در شبکه های دسترسی رادیویی ابری و مشتقات آن، شبکه های دسترسی رادیویی مجازی و شبکه های دسترسی رادیویی باز ارائه شده است.
علاوه بر این، نتایج عملی و آموزه های آموخته شده در مورد محدودیت ها و مسائل پیش بینی نشده مجازی سازی شبکه های دسترسی رادیویی را ارائه داده شده است.

در مقاله ی 
\cite{sree2019open, kawahara2019ran}
 ساختار و مدیریت منابع رادیویی (RRM) هوشمند 
 و همچنین نقش مدیریت لینک رادیویی (RLM) در بهینه سازی انرژی در RRM
در نظر گرفته شده است.
ساختار RLM در 
زیرساخت O-RAN مورد بررسی قرار گرفته است.
علاوه بر این، دیدگاه
O-RAN
 و معماری آن مورد توجه قرار گرفته است.
\section{قرار دادن VNF ها}
NFV
 الگویي است که عملکردهای شبکه سنتی را مجازی می کند و آنها را در سخت افزارهای عمومی و ابرها در مقابل سخت افزارهای تعیین شده، قرار می دهد.
اپراتورهای شبکه تلفن همراه عهده دار تصمیم گیری مدیریت زیرساخت است.
این وظیفه بخشی از تنظیمات شبکه است و شامل تصمیم گیری در مورد قرار دادن VNF های مورد نیاز در سراسر زیرساخت و اختصاص CPU، حافظه و منابع ذخیره سازی به VNF ها و مسیریابی داده ها از طریق گره های شبکه
می باشد.

