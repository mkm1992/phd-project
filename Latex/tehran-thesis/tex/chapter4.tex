\chapter{تخصیص برش شبکه به صورت دینامیکی}
\section{مقدمه}
در این فصل هدف تخصیص برش شبکه به صورت می‌باشد. در فصل قبلی مدل سیستم به طور کامل نوشته شده است و در حالت آفلاین حل گردیده است، در این فصل پارامترها مورد نیاز را نسبت به فصل قبلی کمتر کرده و با استفاده از روش دینامیکی به حل سیستم می‌پردازیم. برای حل این سیستم از روش یادگیری تقویتی عمیق استفاده می‌کنیم.
\section{  مدل سیستم و صورت مسئله}
همانند سیستم فصل قبل، فرض می کنیم $S$ برش شبکه داریم که قرار است $V$ سرویس مختلف که شامل کاربرانی است که از سرویس خاص استفاده می‌نمایند را سرویس دهی نماید.
هر سرویس 
$v\in \{1,2,...,V \} $
شامل تعدادی کاربر تک آنتنه می باشند که سرویس خاصی را درخواست می‌نماید.
هر برش شبکه
$s\in \{1,2,...,S \} $
 شامل تعدادی
 PRB
  RU،
   BBU، 
   و
    VNF 
 می‌باشد.
در این بخش سعی برا‌ین است که در ابتدا مسئله را به ساده‌ترین حالت ممکن حل نماییم. فرض می‌کنیم سه مدل سرویس مختلف داریم که سرویسهای دسته‌ی اول نیازمند تاخیر خاص و سرویسهای دسته‌ی دوم نیازمند داشتن تاخیر کم هستند و سرویس سوم نیازمند داشتن هر دو حالت تاخیر کم و نرخ زیاد است.
در بخش اول این مسئله، هدف بیشینه‌سازی تعدای سرویسهای پذیرفته شده می‌باشد. در اینجا فرض براین است که تعداد برشهاس شبکه محدود می‌باشد. فرض می‌کنیم هر سرویس $v$ دارای اولویت $p_v$ می‌باشد. 
همچنین فرض براین است که هر سرویس شامل ماکسیمم $U-v$ کاربر است و به طور میانگین کاربران آن نیازمند داشتن نرخ بیشتر از $R_v$ و تاخیر کمتر از $D_v$ هستند. درصورتی که کاربری از دسته‌ی اول باشد 
$D_v =  M$ 
که $M$ برای تاخیر یک عدد بزرگ می‌باشد.
و در صورتی که سرویس از دسته‌ی دوم باشد 
$R_v = N $
که $N$ یک عدد کوچک برای نرخ می‌باشد.
صورت مسئله بدین صورت می‌باشد.
\begin{subequations}
	\begin{alignat}{4}
		\max\limits_{\boldsymbol{a} }   \quad &   \sum_{s=1}^{S}\sum_{v=1}^{V} p_v a_{v,s}\\
		\text{\lr{subject to}} \quad & \textstyle \sum_{d=1}^{D_c}\sum_{v=1}^{V}y_{s,d}a_{v,s} \geq 1\times\sum_{v=1}^{V}a_{v,s} \forall s, \\
		&\textstyle  \sum_{s=1}^{S} y_{s,d} \bar{\Omega}_{\mathfrak{z},s}^{tot}  \leq   \tau_{\mathfrak{z}_d}  \forall d,, \forall \mathfrak{z}\in \mathcal{E};  \label{eqomega}
	\end{alignat}
	\label{constraints2}
\end{subequations}
 

 

