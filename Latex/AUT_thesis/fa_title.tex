%!TEX root = AUT_thesis.tex
% در این فایل، عنوان پایان‌نامه، مشخصات خود، متن تقدیمی‌، ستایش، سپاس‌گزاری و چکیده پایان‌نامه را به فارسی، وارد کنید.
% توجه داشته باشید که جدول حاوی مشخصات پروژه/پایان‌نامه/رساله و همچنین، مشخصات داخل آن، به طور خودکار، درج می‌شود.
%%%%%%%%%%%%%%%%%%%%%%%%%%%%%%%%%%%%
% دانشکده، آموزشکده و یا پژوهشکده  خود را وارد کنید
\faculty{ دانشکده برق}
% گرایش و گروه آموزشی خود را وارد کنید
\department{گرایش مخابرات سیستم}
% عنوان پایان‌نامه را وارد کنید
\fatitle{بهبود عملکرد شبکه هاي دسترسي راديويي ابري با مشارکت توزيع شده }
% نام استاد(ان) راهنما را وارد کنید
\firstsupervisor{دکتر محمد جواد عمادی}
%\secondsupervisor{استاد راهنمای دوم}
% نام استاد(دان) مشاور را وارد کنید. چنانچه استاد مشاور ندارید، دستور پایین را غیرفعال کنید.
\firstadvisor{دکتر عباس محمدی}
%\secondadvisor{استاد مشاور دوم}
% نام نویسنده را وارد کنید
\name{مژده }
% نام خانوادگی نویسنده را وارد کنید
\surname{کربلایی مطلب}
%%%%%%%%%%%%%%%%%%%%%%%%%%%%%%%%%%
\thesisdate{شهریور 96}

% چکیده پایان‌نامه را وارد کنید
\fa-abstract{
به دلیل نیاز به افزایش نرخ داده و سرعت انتقال داده، ایجاد نسل جدید در مخابرات مورد  توجه قرار گرفته شده است. نسل پنجم مخابرات مفاهیم جدیدی را بیان می کند که نیازها را برآورده خواهد کرد.
این مفاهیم شامل 
\lr{mm wave}
،
\lr{MIMO} \LTRfootnote{Multiple Input Multiple Output}،
\lr{CRAN} \LTRfootnote{Cloud Radio Access Network}
و ... می باشد.\\
امروزه شبکه های دسترسی رادیویی ابری \LTRfootnote{CRAN} به عنوان شبکه
های سلولی نسل آینده \lr{5G} توجه بسیاری را به خود جلب
کرده اند. در این نوع شبکه ها پردازش های باند پایه از
ایستگاههای باند پایه به مرکز کنترل واقع در ابر \LTRfootnote{Cloud} منتقل می
شوند. اطلاعات دریافتی در ایستگاههای باند پایه که به
شکل \lr{inphase} و  \lr{quadrature} می باشند \cite{mobile} ، توسط لینک
های \lr{fronthaul} به ابر منتقل شده و پردازش های لازم
بر روی آنها صورت می گیرد. 
 همجنین لینک \lr{fronthaul} دارای محدودیت در ظرفیت می باشد. \newline
 در ابتدا ساختار
شبکه های \lr{C-RAN} ، \LTRfootnote{Cloud Radio Access Network} مورد نقد و بررسی قرار می گیرد ، سپس
مزایای این ساختار جدید و چالش های آن عنوان می شود.
 در این پروژه، هدف تخصیص توان  برای ساختار \lr{CRAN} در حالات مختلف لینک فراسو و فروسو می باشد. 
بنابراین برای لینک فراسو و فروسو، سیستم مدلهای مختلف از جمله خوشه بندی کاربران و واحدهای رادیوی، اعمال محدودیت ظرفیت لینک \lr{fronthaul} و در نهایت  شبکه های \lr{D2D} مورد بررسی قرار گرفته است و در رابطه با تخصیص توان در این سیستم مدل ها صحبت می شود. \newline
همچنین سیستم مدلی برای لینک فراسو و فروسو به طور مجزا بیان می گردد که شامل چندین خوشه می باشد و فرض محدودیت لینک \lr{fronthaul} در نظر گرفته می شود.\newline
در آخر نیز سیستم مدلی که به صورت همزمان لینک فراسو و فروسو را باهم بهینه می کند بیان نمودیم. فرض بر این است که جندین خوشه در لینک فراسو و جندین خوشه در لینک فروسو عمل می کنند و بر یکدیگر تداخل اعمال می نمایند. تخصیص توان بر روی این خوشه ها صورت گرفته است.
}
% کلمات کلیدی پایان‌نامه را وارد کنید
\keywords{شبکه های دسترسی رادیویی ابری، لینک \lr{fronthaul}، \lr{MIMO}، خوشه بندی، بازدهی انرژی}
\AUTtitle
%%%%%%%%%%%%%%%%%%%%%%%%%%%%%%%%%%