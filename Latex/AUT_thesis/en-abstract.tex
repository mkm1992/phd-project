%کلمات کلیدی انگلیسی
\latinkeywords{ Cloud Radio Access Network, Multiple-Input Multiple-Output, Energy efficiency, Clusterization, Power allocation, Lagrangian function.}
%چکیده انگلیسی

\en-abstract{
Since more rate and speed is needed in technology, new generation of technology is considered that new concepts such as CRAN, mm wave, Massive MIMO and etc are defined. 
Cloud radio access networks generate a new architechture for 5G that is proposed to enhance both spectral efficiency and energy efficiency. 
The architecture of cran and the difference between this architucture and traditional one is expressed.
Also some system models such as clustering and limited fronthaul capacity is considered. In addition, D2D system in C-RAN is described too.
The optimal power allocation for the downlink and uplink of Multiple-Input Multiple-Output (MIMO) Cloud Radio Access Network (C-RAN) with limited fronthaul capacity in terms of maximizing Energy Efficiency (EE) is investigated. 
In the considered system, in downlink the compressed and precoded message generated by Central Unit (CU) is transmitted to Remote Radio Heads (RRHs) via a fronthaul link with limited capacity, and the RRHs and Users Equipments (UEs) are assumed to be clustered into $S$ cluster sets.
In uplink, the recieved message by RRHs which are clusteres into $S$ clusters, is transmitted through fronthaul link to CU and in CU beamforming vector is applied to the message.
Here, we use an iterative algorithm with Lagrangian function to optimize the EE.
Also both uplink and downlink clusters are considered in a system model and optimization is done for both together.
}
%%%%%%%%%%%%%%%%%%%%% کدهای زیر را تغییر ندهید.

\newpage
\thispagestyle{empty}
\begin{latin}
\section*{\LARGE\centering Abstract}

\een-abstract

\vspace*{.5cm}
{\large\textbf{Key Words:}}\par
\vspace*{.5cm}
\elatinkeywords
\end{latin}