 %%%  کلاس AUT_thesis، نسخه آبان 95
%%%   دانشگاه صنعتی امیرکبیر                 http://www.aut.ac.ir
%%%  تالار گفتگوی پارسی‌لاتک،       http://forum.parsilatex.com
%%%   آپدیت شده در آبان 95
%%      پشتیبانی و راهنمایی          badali_farhad@yahoo.com

%-----------------------------------------------------------------------------------------------------
%        روش اجرا.: 2 بار F1 ، 2 بار  F11(به منظور تولید مراجع) ، دوبار Ctrl+Alt+I (به منظور تولید نمایه) و دو بار F1 -------> مشاهده Pdf
%        اگر قصد نوشتن رساله دکتری را دارید، در خط زیر به جای msc،
%      کلمه phd را قرار دهید. کلیه تنظیمات لازم، به طور خودکار، اعمال می‌شود.
% !TEX TS-program = XeLaTeX
\documentclass[oneside,msc]{AUT_thesis}
%       فایل commands.tex را حتماً به دقت مطالعه کنید؛ چون دستورات مربوط به فراخوانی بسته زی‌پرشین 
%       و دیگر بسته‌ها و ... در این فایل قرار دارد و بهتر است که با نحوه استفاده از آنها آشنا شوید. توجه شود برای نسخه نهایی پایان‌نامه حتماً hyperref را 
%        غیرفعال کنید.



\input{commands}
\begin{document}
\baselineskip=.75cm
%!TEX root = AUT_thesis.tex
% در این فایل، عنوان پایان‌نامه، مشخصات خود، متن تقدیمی‌، ستایش، سپاس‌گزاری و چکیده پایان‌نامه را به فارسی، وارد کنید.
% توجه داشته باشید که جدول حاوی مشخصات پروژه/پایان‌نامه/رساله و همچنین، مشخصات داخل آن، به طور خودکار، درج می‌شود.
%%%%%%%%%%%%%%%%%%%%%%%%%%%%%%%%%%%%
% دانشکده، آموزشکده و یا پژوهشکده  خود را وارد کنید
\faculty{ دانشکده برق}
% گرایش و گروه آموزشی خود را وارد کنید
\department{گرایش مخابرات سیستم}
% عنوان پایان‌نامه را وارد کنید
\fatitle{بهبود عملکرد شبکه هاي دسترسي راديويي ابري با مشارکت توزيع شده }
% نام استاد(ان) راهنما را وارد کنید
\firstsupervisor{دکتر محمد جواد عمادی}
%\secondsupervisor{استاد راهنمای دوم}
% نام استاد(دان) مشاور را وارد کنید. چنانچه استاد مشاور ندارید، دستور پایین را غیرفعال کنید.
\firstadvisor{دکتر عباس محمدی}
%\secondadvisor{استاد مشاور دوم}
% نام نویسنده را وارد کنید
\name{مژده }
% نام خانوادگی نویسنده را وارد کنید
\surname{کربلایی مطلب}
%%%%%%%%%%%%%%%%%%%%%%%%%%%%%%%%%%
\thesisdate{شهریور 96}

% چکیده پایان‌نامه را وارد کنید
\fa-abstract{
به دلیل نیاز به افزایش نرخ داده و سرعت انتقال داده، ایجاد نسل جدید در مخابرات مورد  توجه قرار گرفته شده است. نسل پنجم مخابرات مفاهیم جدیدی را بیان می کند که نیازها را برآورده خواهد کرد.
این مفاهیم شامل 
\lr{mm wave}
،
\lr{MIMO} \LTRfootnote{Multiple Input Multiple Output}،
\lr{CRAN} \LTRfootnote{Cloud Radio Access Network}
و ... می باشد.\\
امروزه شبکه های دسترسی رادیویی ابری \LTRfootnote{CRAN} به عنوان شبکه
های سلولی نسل آینده \lr{5G} توجه بسیاری را به خود جلب
کرده اند. در این نوع شبکه ها پردازش های باند پایه از
ایستگاههای باند پایه به مرکز کنترل واقع در ابر \LTRfootnote{Cloud} منتقل می
شوند. اطلاعات دریافتی در ایستگاههای باند پایه که به
شکل \lr{inphase} و  \lr{quadrature} می باشند \cite{mobile} ، توسط لینک
های \lr{fronthaul} به ابر منتقل شده و پردازش های لازم
بر روی آنها صورت می گیرد. 
 همجنین لینک \lr{fronthaul} دارای محدودیت در ظرفیت می باشد. \newline
 در ابتدا ساختار
شبکه های \lr{C-RAN} ، \LTRfootnote{Cloud Radio Access Network} مورد نقد و بررسی قرار می گیرد ، سپس
مزایای این ساختار جدید و چالش های آن عنوان می شود.
 در این پروژه، هدف تخصیص توان  برای ساختار \lr{CRAN} در حالات مختلف لینک فراسو و فروسو می باشد. 
بنابراین برای لینک فراسو و فروسو، سیستم مدلهای مختلف از جمله خوشه بندی کاربران و واحدهای رادیوی، اعمال محدودیت ظرفیت لینک \lr{fronthaul} و در نهایت  شبکه های \lr{D2D} مورد بررسی قرار گرفته است و در رابطه با تخصیص توان در این سیستم مدل ها صحبت می شود. \newline
همچنین سیستم مدلی برای لینک فراسو و فروسو به طور مجزا بیان می گردد که شامل چندین خوشه می باشد و فرض محدودیت لینک \lr{fronthaul} در نظر گرفته می شود.\newline
در آخر نیز سیستم مدلی که به صورت همزمان لینک فراسو و فروسو را باهم بهینه می کند بیان نمودیم. فرض بر این است که جندین خوشه در لینک فراسو و جندین خوشه در لینک فروسو عمل می کنند و بر یکدیگر تداخل اعمال می نمایند. تخصیص توان بر روی این خوشه ها صورت گرفته است.
}
% کلمات کلیدی پایان‌نامه را وارد کنید
\keywords{شبکه های دسترسی رادیویی ابری، لینک \lr{fronthaul}، \lr{MIMO}، خوشه بندی، بازدهی انرژی}
\AUTtitle
%%%%%%%%%%%%%%%%%%%%%%%%%%%%%%%%%%
\vspace*{7cm}
\thispagestyle{empty}
\begin{center}
\includegraphics[height=5cm,width=12cm]{besm}
\end{center}
% تاییدیه دفاع
\input{taid}
% چنانچه مایل به چاپ صفحات «تقدیم»، «نیایش» و «سپاس‌گزاری» در خروجی نیستید، خط‌های زیر را با گذاشتن ٪  در ابتدای آنها غیرفعال کنید.
% پایان‌نامه خود را تقدیم کنید
% نیایش خود را در فایل زیر بنویسید.
\begin{acknowledgementpage}

\vspace{1.5cm}

{\nastaliq
{


از استاد گرامیم جناب آقای دکتر عمادی بسیار سپاسگزارم چرا که بدون راهنماییهای ایشان تامین این پایان نامه بسیار مشکل مینمود. 
این پایان نامه را به پدر و مادرم، اساتید عزیز و خواهر و برادر مهربانم تقدیم میکنم  امیدوارم قادر به درک زیباییهای وجودشان باشم با تشکر




















}}\end{acknowledgementpage}
\newpage
%سپاسگزاری را در فایل زیر بنویسید.
%%%%%%%%%%%%%%%%%%%%%%%%%%%%%%%%%%%%
\newpage\thispagestyle{empty}
% سپاس‌گزاری
{\nastaliq
سپاس‌گزاری
}
\\[2cm]

از استاد گران قدر جناب آقای دکتر عمادی کمال سپاس را دارم و از همه دوستانی که مرا یاری نمودند تشکر می کنم. 














% با استفاده از دستور زیر، امضای شما، به طور خودکار، درج می‌شود.
\signature








%%%%%%%%%%%%%%%%%%%%%%%%%%%%%%%%%%%%%%%%%
%%%%%%%%%%%%%%%%%%%%%%%%%%%%%%%%%کدهای زیر را تغییر ندهید.
\newpage\clearpage

\pagestyle{style2}

\vspace*{-1cm}
\section*{\centering چکیده}
%\addcontentsline{toc}{chapter}{چکیده}
\vspace*{.5cm}
\ffa-abstract
\vspace*{2cm}


{\noindent\large\textbf{واژه‌های کلیدی:}}\par
\vspace*{.5cm}
\fkeywords
%دستور زیر برای شماره گذاری صفحات قبل از فصل اول با حروف ابجد است.
\pagenumbering{alph}
%-----------------------------------------------------------------------------
%در فایل زیر دستورات مربوط به نمایش صفحات فهرست مطالب- فهرست اشکال و جداول است.
\input{TOC-TOF-LOT}
% در صورت تمایل می‌توانید با فعال کردن دستور بالا، لیست تصاویر را به  پایان‌نامه خود اضافه کنید.
%-------------------------------------------------------------------------symbols(فهرست نمادها)
% وجود لیست نمادها الزامیست.(لطفاً نمادهای خود را جایگذین نمادهای پیش‌فرض کنید.)
%%%%%%%%%%%%%

{\centering\LARGE\textbf{فهرست نمادها}\par}%

\pagenumbering{alph}
\setcounter{page}{\thesavepage}
%\setcounter{page}{6}
\vspace*{1cm}

\pagestyle{style3}
%\thispagestyle{empty}
%\addcontentsline{toc}{chapter}{فهرست نمادها}
\symb{\text{ نماد}}{مفهوم}
\\
%مقادیر بالا را تغییر ندهید
%%%%%%%%%%%%%%%%%%%%%%%%%%%%%%%%%%%%%%%%%%%%%%%%%%%%%%%%%
\symb{\mathfrak{R}_{d_{(s,k)}}}{
نرخ قابل دسترسی برای $k$ امین کاربر در $s$ امین خوشه
}
\symb{\gamma_{d_{(s,k)}}}{
\lr{SINR}
 برای $k$ امین کاربر در $s$ امین خوشه
}
\symb{B}{
پهنای باند سیستم
}
\symb{N_0}{
توان نویز گوسی جمع شونده
}
\symb{W}{
ماتریس پیش کدگذاری شده و یا پرتو دهی
}
\symb{\sigma_q^2}{
واریانس نویز کوانتیزاسیون
}
\symb{\bar{p}_{r_{(s,i)}} }{
توان واحد رادیویی $i$ ام در$s$ امین خوشه
}
\symb{\eta}{
بازدهی انرژِی
}
\symb{p_{d_{(s,k)}}}{
توان $k$ امین کاربر در $s$ امین خوشه
}
\symb{ C_{r_{(s,i)}} }{
ظرفیت لینک \lr{fronthaul} واحد رادیویی $i$ ام در$s$ امین خوشه
}
\symb{h}{
بردار کانال
}
%%%%%%%%%%%%%%%%%%%%%%%%%%%%%%%%%%%%%%%

\thispagestyle{style3}
\newpage
%\pagestyle{style1}
%%%%%%%%%%%%%%%%%%%%%%%%%%%%%%%%%%%%

%%%%%%%%%%%%%

{\centering\LARGE\textbf{فهرست مخففها}\par}%

\pagenumbering{alph}
\setcounter{page}{\thesavepage}
%\setcounter{page}{6}
\vspace*{1cm}

\pagestyle{style3}
%\thispagestyle{empty}
%\addcontentsline{toc}{chapter}{فهرست نمادها}
\begin{center}
 \begin{tabular}{||c || c||} 
 \hline
 کلمه اصلی & کلمه مخفف \\ [0.5ex] 
 \hline\hline
 %\begin{latin}
 \lr{Cloud Radio Access Network} & \lr{C-RAN} \\ 
 \hline
 \lr{Fog Radio Access Network} & \lr{F-RAN} \\ 
 \hline
 \lr{Hetrogeneous Cloud Radio Access Network} & \lr{H-CRAN} \\ 
 \hline
   \lr{Radio Remote Head} & \lr{RRH} \\
 \hline
\lr{Base Band Unit} &  \lr{BBU}  \\
 \hline
  \lr{Control Unit} & \lr{CU}   \\
 \hline
   \lr{Radio Unit} & \lr{RU}   \\
 \hline
  \lr{Base Station} &  \lr{BS}  \\
 \hline
  \lr{Time Division Duplexing} &  \lr{TDD}  \\
 \hline
   \lr{High Power Node} &  \lr{HPN}  \\
 \hline
  \lr{Multiple Input Multiple Output} &  \lr{MIMO}  \\
 \hline
  \lr{Uplink} &  \lr{UL}  \\
 \hline
   \lr{Downlink} &  \lr{DL}  \\
 \hline
   \lr{Bandwidth} &  \lr{BW}  \\
 \hline
    \lr{Channel State Information} &  \lr{CSIT}  \\
 \hline
  \lr{Energy Efficiency} & \lr{EE}  \\ [1ex] 
 \hline
 %\end{latin}
\end{tabular}
\end{center}
%%%%%%%%%%%%%%%%%%%%%%%%%%%%%%%%%%%%%%%

\thispagestyle{style3}
\newpage
%\pagestyle{style1}
%%%%%%%%%%%%%%%%%%%%%%%%%%%%%%%%%%%%

\pagenumbering{arabic}
\pagestyle{style1}
%--------------------------------------------------------------------------chapters(فصل ها)
\chapter{مقدمه}
\section{مقدمه ای بر ساختار \lr{ORAN}}
\lr{Open RAN}(\lr{ORAN})
تبسیط و ترکیبی از دو ساختار \lr{C-RAN} و \lr{xRAN} می باشد که انتظار می رود که در فناوری نسل پنجم مخابرات مورد استفاده قرار گرفته و منجر به بهبود عملکرد شبکه های دسترسی رادیویی \lr{RAN} گردد. 
ایده اصلی \lr{C-RAN} جداسازی بخش رادیویی (\lr{RRH}) 
\LTRfootnote{Radio Remote Head}
 از واحد پردازشی باند پایه (\lr{BBU})
 \LTRfootnote{Baseband Unit}
  است.
از تجمیع \lr{BBU} ها بر روی سرور ابری، \lr{BBU-Pool} ایجاد می شود.
این ساختار جدید جزو یکی از ساختارهایی است که در \lr{5G} امکان استفاده را دارد.در این ساختار جدید در راستای بهینه سازی عملکرد \lr{BBU}
 ها در مواجهه باایستگاههای پایه پر ترافیک و کم ترافیک،
 \lr{BBU}ها به صورت یک مجموعه ی واحد تحت عنوان 
\lr{BBU Pool}
 در آمده اند که این مجموعه بین چندین سلول 
 به اشتراک گزارده شده و مطابق شکل زیر مجازی سازی
می شود. 
در توضیح بیشتر این ساختار می توان این گونه
عنوان کرد که \lr{BBU Pool} به عنوان یک خوشه ی مجازی
در نظر گرفته می شود که شامل پردازش گرهایی می باشد
که پردازش های باند پایه را انجام می دهند. ارتباط بین
  \lr{BBU}ها در ساختار های فعلی به شکل  $X_2$ برقرار می شود
که در این ساختار ارتباط بین خوشه ها از فرم جدید $X_2$
تحت عنوان  $X_2 +$برقرار می شود.
\newline
در شکل \ref{fig:C-RAN} ساختار کلی شبکه ی  \lr{C-RAN} در سیستم های
\lr{ LTE}
 نمایش داده شده است. همان طور که در شکل قابل
مشاهده می باشد ساختار کلی شبکه  \lr{C-RAN} به دو بخش
 \lr{backhaul} و \lr{fronthaul} تقسیم بندی شده است. بخش
 \lr{fronthaul}شبکه به مرحله ی اتصال سایت های \lr{ RRH}به
 به \lr{BBU Pool} به اتصال \lr{backhaul} و بخش \lr{BBU Pool}
هسته ی شبکه ی سیار اطلاق می شود. همان گونه که قبلا
ذکر شد  \lr{ RRH}ها در نزدیکی انتن نصب شده و از طریق
لینک های انتقالی نوری با پهنای باند وسیع و تاخیر کم به
پردازشگرهای قوی در  \lr{BBU}متصل می شوند. توسط این
لینک های انتقالی است که سیگنال های دیجیتالی باند
پایه از نوع \lr{IQ} بین \lr{RRH} و \lr{BBU} انتقال می یابند \cite{checko2015cloud}.
\begin{figure}[H]
  \centering
    \includegraphics[width=\textwidth]{./fig/CRAN}
  \caption{ساختار شبکه ی \lr{C-RAN} \cite{checko2015cloud}}
  \label{fig:C-RAN}
\end{figure}
\section{مزایای شبکه ی \lr{C-RAN}}

در این بخش قصد داریم مزایای شبکه ی \lr{C-RAN} و هدف از استفاده ی آن در \lr{5G} را بیان کنیم.
\newline
در هر دو نوع سلولهای ماکرو و میکرو، می توان از ساختار \lr{C-RAN} بهره برد. در حالت ماکرو، متمرکز کردن \lr{BBU} ها به صورت \lr{BBU Pool}، منجر به استفاده ی بهینه از \lr{BBU} ها و کاهش هزینه ی ایستگاه پایه \LTRfootnote{base station} می شود. همچنین منجر به کاهش مصرف توان و فراهم کردن انعطاف پذیری بیشتر در شبکه و تطبیق آن با ترافیک غیر یکسان می شود. علاوه بر این، باعث تبدیل سیگنال تداخل به سیگنال مفید تبدیل می شود. در ادامه این مزایا به صورت گسترده تر بیان می گردد.   

\chapter{مروری بر کارهای پیشین}

\chapter{تخصیص منابع در شبکه های \lr{Open RAN}}
در این فصل، هدف تخصیص منابع در شبکه های \lr{Open RAN} در لینک فروسو می باشد که شامل تخصیص توان و برش های شبکه است.
در این بخش فرض بر این است که شبکه ی نسل پنجم مخابرات با زیرساخت \lr{Open RAN} موجود است. این شبکه سرویس هایی از قبیل \lr{IoT}، سرویس تلفن، پیامک و ... در اختیار کاربران قرار می دهد. در اینجا از مفهوم برش شبکه استفاده شده است بدین صورت که به جای دیدن کاربران به صورت مجزا، کاربرانی که از یک سرویس خاص استفاده می نمایند در دسته ی آن سرویس قرار گرفته و دسته بندی می شوند. همچنین برش هایی از شبکه در اختیار کاربران هر سرویس خاص، قرار می گیرد که هر برش شبکه شامل تعدادی واحد رادیویی
\LTRfootnote{Radio Unit}
(\lr{RU}) 
،
بلوک های منبع فیزیکی
 \LTRfootnote{Physical Resource Block}
 (\lr{PRB})، 
 یک واحد توزیع شده
\LTRfootnote{Distributed Unit}
(\lr{DU}) 
 ، 
 یک واحد مرکزی
\LTRfootnote{Centralized Unit}
(\lr{CU})  
  می باشد که هر واحد توزیع شده و مرکزی شامل تعدادی توابع شبکه ی مجازی شده
  \LTRfootnote{Virtual Network Function}
(\lr{VNF}) 
می باشد. 
\section{مدل سیستم}
در این قسمت، سیستم مدل به صورت کامل مورد بررسی قرار می گیرد.
فرض می کنیم $S$ برش شبکه داریم که قرار است $V$ سرویس مختلف که شامل کاربرانی است که از سرویس خاص استفاده می نمایند را سرویس دهی نماید.
هر سرویس 
$v\in \{1,2,...,V \} $
شامل 
$U_v$
کاربر تک آنتنه می باشند که از سرویس خاصی را درخواست می نماید.
هر برش شبکه 
$s \in \{1,2,...,S \}$
شامل 
$R_s$
واحد رادیویی،
$K_s$
بلوک های منابع فیزیکی، یک واد توزیع شده و یک واحد مرکزی که شامل \lr{VNF} هایی می باشند.
همچنین برش های شبکه می توانند منابع مشترک داشته باشند.
تمام \lr{RU}هایی که به یک سرویس خاص خدمت رسانی می کنند به صورت مشارکتی سیگنال به تمام کاربران در آن سرویس ارسال می نمایند. 
\cite{motalleb2017optimal,mimoCran}
هر واحد رادیویی
$r \in \{1,2,...,R \}$
به یک واحد توزیع شده از طریق لینک فیبر نوری با ظرفیت \lr{fronthaul} 
محدود متصل می باشد.
در سیستم \lr{Open RAN}
دو لایه ی پردازشی که اولی در \lr{CU} و دومی در \lr{DU} قرار گرفته است که پردازش ها با \lr{VNF} ها صورت می گیرد.
لایه پایین تر (\lr{DU}) شامل 
\lr{high-PHY}
،
\lr{MAC}
و 
\lr{RLC}
می باشد و 
لایه ی بالاتر 
(\lr{CU})
شامل 
\lr{RRC}
،
\lr{PDCP}
و 
\lr{SDAP}
است.
فرض بر این است که $M_1$
\lr{VNF}
در \lr{DU}
و 
$M_2$
\lr{VNF}
در \lr{CU} قرار دارد.
هر \lr{VNF} به یک برش یا بیشتر تعلق دارد.
در $s^{th}$ امین برش شبکه $M_{s,1}$ 
\lr{VNF}
در \lr{DU}
و 
$M_{s,2}$
\lr{VNF}
دپر لایه ی \lr{CU} می باشد.
\lr{VNF}
های موجود در لایه ی \lr{DU} و \lr{VU} به ترتیب دارای ظرفیت محاسباتی 
$\mu_1$ 
و
 $\mu_2$
می باشند.
\subsection{نرخ قابل دسترس}
نرخ قابل دسترس برای $i^{th}$ امین کاربر در $v^{th}$امین سرویس به صورت زیر نوشته می شود
\begin{equation}\label{eq1}
\mathcal{R}_{u(v,i)} = B \log_2({1+ \rho_{u(v,i)}}),
\end{equation}
  که $B$ پهنای باند سیستم و 
  $\rho_{u(v,i)}$
  نسبت سیگنال به نویز $i^{th}$
  $i^{th}$ 
  امین کاربر در
   $v^{th}$
   امین سرویس
  می باشد 
  که از رابطه ی زیر بدست می آید
 \begin{equation}\label{eq2}
\rho_{u(v,i)} =  \frac{p_{u(v,i)}\sum_{s=1}^{S}|\bold{h}_{R_s,u(v,i)}^H \bold{w}_{R_s,u(v,i)}|^2 a_{v,s}}{BN_0 + I_{u(v,i)}},
\end{equation}
که   $p_{u(v,i)}$
نشان دهنده ی توان ارسالی توسط \lr{RU} به 
$i^{th}$ 
  امین کاربر در
   $v^{th}$
   امین سرویس
   است و 
 $\bold{h}_{R_s,u(v,i)} \in \mathbb{C}^{{R}_s}$
 بردار کانال گین لینک وایرلس از \lr{RU} ها در 
$s^{th}$
امین برش شبکه می باشد.
 همچنین 
$\bold{w}_{R_s,u(v,i)} \in \mathbb{C}^{{R}_s}$
نشان دهنده ی بردار بیم فرمینگ ارسالی برای \lr{RU}ها در $s$ امین برش شبکه به کاربر $i$ ام در سرویس $v$ ام می باشد.
   به علاوه، $BN_0$
   نشان دهنده ی توان نویز اضافه شونده ی گوسی می باشد و $I_{u(v,i)}$
   توان سیگنال تداخلی است.
همچنین $a_{v,s} \in \{0,1\}$
متغیر باینری است که نشان دهنده ی این است که برش شبکه ی $s$ به سرویس $v$ خدمات رسانی می کند یا نه.
درصورتی که 
 $a_{v,s} =1$
 برش شبکه ی $s$ به سرویس $v$ خدمات رسانی می کند. در غیر این صورت خدمت رسانی نمی کند.
\newline
برای بدست آوردن \lr{SNR} در فرمول \eqref{eq2}، 
فرض می شود که 
 $\bold{y}_{U_v}\in \mathbb{C}^{U_v} $
 بردار سیگنال دریافتی از همه ی کاربران در سرویس $v$ می باشد 
\begin{equation}\label{eq3}
\textstyle \bold{y}_{U_v} = \sum_{s = 1}^{S}\sum_{k=1}^{K_s} \boldsymbol{H}^H_{\mathcal{R}_s,\mathcal{U}_v} \
\mathfrak{y}_{R_s}\zeta_{U_v,k,s} a_{v,s}+ \boldsymbol{z}_{\mathcal{U}_v},
\end{equation}
که $\mathfrak{y}_{R_s} = \boldsymbol{W}_{\mathcal{R}_s,\mathcal{U}_v}\boldsymbol{P}_{U_v}^{\frac{1}{2}}\boldsymbol{x}_{\mathcal{U}_v}+ \boldsymbol{q}_{\mathcal{R}_s}$
و 
$\boldsymbol{x}_{ \mathcal{U}_v} = [x_{ u_{(v,1)}},...,x_{ u_{(v,\mathcal{U}_v)}}]^T \in \mathbb{C}^{{R}_s } $
نشان دهنده ی بردار سمبل ارسالی کاربران سرویس $v$ می باشد.
$\boldsymbol{z}_{U_v}$
نویز گوشی جمع شونده است و
$\boldsymbol{z_{U_v}} \backsim \mathcal{N}(0,N_0\boldsymbol{I}_{{U}_v})$
.
همچنین
$N_0$
توان نویز می باشد.
علاوه بر این
$\boldsymbol{q}_{R_s} \in \mathbb{C}^{{R}_s }  $
نشان دهنده ی نویز کوانتیزاسیون می باشد که از فشرده سازی سیگنال  در واحد توزیع شده بدست آمده است.
$\boldsymbol{P}_{U_v} = \diag{(p_{u_{(v,1)}}, ..., p_{u_{(v,\mathcal{U}_v)}})}$.
همچنین در اینجا، 
$\zeta_{k,s}^{U_v} \delequal \{\zeta_{k,s}^{u(v,1)},\zeta_{k,s}^{u(v,2)},...,\zeta_{k,s}^{u(v,N_{U_v})}\}$
و 
$\zeta_{k,s}^{u(v,i)} \in \{0,1\}$
پارامتر باینری می باشد که نشان دهنده ی این است که $i$ امین کابر در $v$امین سرویس امکان ارسال سیگنال خود را از طریق 
\lr{PRB}
$k$
ام دارد یا نه، در ضمن این \lr{PRB} 
 متعلق به برش $s$ ام می باشد و یا نه.
 $\boldsymbol{H}_{\mathcal{R}_s,\mathcal{U}_v}=\left[\boldsymbol{h}_{\mathcal{R}_s,u_{(v,1)}},\ldots,\boldsymbol{h}_{\mathcal{R}_s,v_{(v,\mathcal{U}_v)}}\right]^T  \in \mathbb{C}^{{R}_s\times {U}_v }$
نشان دهنده ی ماتریس کانال بین دسته واحد رادیویی 
 $\mathcal{R}_s$
 به دسته 
 $\mathcal{U}_v$
 کاربران می باشد.
بردار کانال بین $s$امین برش و $i$امین کابر در $v$امین سرویس $\boldsymbol{h}_{\mathcal{R}_s,u_{(v,i)}}\in \mathbb{C}^{{R}_s}$ به صورت زیر نشان داده شده است 
\begin{equation}
\boldsymbol{h}_{\mathcal{R}_s,u_{(s,i)}} = \boldsymbol{\beta}^\frac{1}{2}_{\mathcal{R}_s,u_{(v,i)}} \boldsymbol{g}_{\mathcal{R}_s,u_{(v,i)}},
\end{equation} 
که در اینجا 
$\boldsymbol{g}_{\mathcal{R}_s,u_{(v,i)}} \backsim \mathcal{N}(0,N_0\boldsymbol{I}_{\mathcal{U}_v})$
نشان دهنده ی بردار کانال فیدینگ سریع و تخت می باشد و
 $\boldsymbol{\beta}_{\mathcal{R}_s,u_{(v,i)}}=\text{diag}(b_{r_{(s,1),u_{(v,i)}}},\ldots,b_{r_{(s,\mathcal{R}_s),u_{(v,i)}}})$
 نشان دهنده ی ماتریس فیدینگ مقیاس بزرگ می باشد.
در اینجا فرض بر این است که اطلاعات حالت کانال \lr{CSI}، به صورت کامل می باشد.\newline
 $\boldsymbol{W}_{\mathcal{R}_s,\mathcal{U}_v} = [\boldsymbol{w}_{\mathcal{R}_s,u(v,1)},...,\boldsymbol{w}_{\mathcal{R}_s,u(v,U_v)}] \in \mathbb{C}^{{R}_s\times U_v} $
 بردار بیم فرمینگ \lr{zero forcing}
  می باشد که برای مینیمم کردن تداخل می باشد و بدین صورت است
\begin{equation}
\textstyle \boldsymbol{W}_{\mathcal{R}_s,\mathcal{U}_v} = \boldsymbol{H}_{\mathcal{R}_s,\mathcal{U}_v}(\boldsymbol{H}_{\mathcal{R}_s,\mathcal{U}_v}^H \boldsymbol{H}_{\mathcal{R}_s,\mathcal{U}_v})^{-1}.
\end{equation}  
توان تداخلی کابر $i$ام به سرویس $v$ام به صورت زیر بیان می شود
\begin{equation}
\begin{split}
 I_{u_{(v,i)}} &=
 \underbrace{\sum_{s=1}^{S}\sum_{n=1}^{S}\sum_{\substack{l=1 \\ l\neq i}}^{{U}_v} \gamma_{1}  p_{u_{(v,l)}}a_{v,s}\zeta_{u_(v,i),n,s}\zeta_{u_(v,l),n,s}}_{\text{(intra-service interference)}}\\
&+ \underbrace{\sum_{\substack{y=1 \\ l\neq v}}^{V}\sum_{s=1}^{S}\sum_{n=1}^{S}\sum_{l=1}^{{U}_y} \gamma_{2}  p_{u_{(y,l)}}a_{y,s} \zeta_{u_(v,i),n,s}\zeta_{u_(y,l),n,s}}_{\text{(inter-service interference)}}\\
&+\underbrace{ \sum_{s=1}^{S} \sum_{j=1}^{{R}_s} {\sigma_q}_{r_{(s,j)}}^2 |\boldsymbol{h}_{r_{(s,j)}, u_{(v,i)}}|^2 a_{v,s}}_{\text{(quantization noise interference)}},
\end{split}
\end{equation}
که 
$\gamma_{1} =|\boldsymbol{h}_{\mathcal{R}_s, u_{(v,i)}}^H \boldsymbol{w}_{\mathcal{R}_{s},u_{(v,l)}}|^2$
و 
$\gamma_{2} =|\boldsymbol{h}_{\mathcal{R}_s, u_{(v,i)}}^H \boldsymbol{w}_{\mathcal{R}_{s},u_{(y,l)}}|^2$.
همچنین 

${\sigma_q}_{r_{(s,j)}}$
واریانس نویز کوانتیزاسیون
$j$
امین 
واحد رادیویی در برش $s$ می باشد.
سیگنال تداخلی برای هر کاربر از سیگنالهای کاربرانی بدست می آید که از \lr{PRB} مشترکی استفاده نموده اند.
درصورت قرار دادن $P_{max}$
به جای 
$p_{u_{(v,l)}}$
و 
$p_{u_{(y,l)}}$
باند بالایی  
$\bar{I}_{u_{(v,i)}}$
برای 
$I_{u_{(v,i)}}$
بدست می آید
بنابراین،
$\bar{\mathcal{R}}_{u_{(v,i)}} \forall v , \forall i$ 
از قرار گرفتن 
$\bar{I}_{u_{(v,i)}}$
به جای 
$I_{u_{(v,i)}}$
در رابطه ی 
\eqref{eq1} و \eqref{eq2}
بدست می آید.
\newline
فرض کنید $\bar{p}_{r_{(s,j)}}$
نشان دهنده ی سیگنال ارسالی از $j$ امین واحد رادیویی در $s$ امین برش می باشد.
از رابطه ی \eqref{eq3} داریم
\begin{equation}
\bar{p}_{r_{(s,j)}} = \sum_{v=1}^{V}\boldsymbol{w}_{r_{(s,j)},\mathcal{U}_{v}} \boldsymbol{P}_{\mathcal{U}_v}^{\frac{1}{2}} \boldsymbol{P}_{\mathcal{U}_v}^{H \frac{1}{2}}   \boldsymbol{w}_{r_{(s,j)},\mathcal{U}_{v}}^H a_{v,s} + \sigma_{q_{r(s,j)}}^2.
\end{equation}
در این صورت نرخ کاربران در لینک \lr{fronthaul} بین $j$امین واحد رادیویی در برش $s$ام با واحد توزیع شده ی موجود در این برش، بدین صورت می باشد  \cite{simeone2016cloud, 1111}
\begin{equation}
C_{R_{(s,j)}} = \log{(1+\sum_{v=1}^{V}\frac{w_{r_{(s,j)},\mathcal{D}_{s}} \boldsymbol{P}_{\mathcal{U}_v}^{\frac{1}{2}} \boldsymbol{P}_{\mathcal{U}_v}^{H \frac{1}{2}}   w_{r_{(s,j)},\mathcal{U}_{v}}^H a_{v,s}}{ \sigma_{q_{r(s,j)}}^2})},
\end{equation}
که در اینجا 
$a_{v,s}$
متغیر باینری است که نشان دهنده ی این است که برش $s$ام به سرویس $v$ خدمات رسانی می کند یا نه.
\subsection{میانگین تاخیر}
فرض کنید ورود بسته های کاربران، فرآیند پوآسون با نرخ ورود 
$\lambda_{u(v,i)}$
برای $i$امین کاربر در سرویس $v$ ام می باشد.
بنابراین، در لایه ی واحد مرکزی (\lr{CU})، میانگین نرخ ورود داده ی کاربری که از خدمات برش $s$ استفاده می نماید 
$\alpha_{s_1} = \sum_{v=1}^{V}\sum_{u=2}^{U_v}a_{v,s}\lambda_{u(v,i)}$
می باشد. 
همچنین، میانگین نرخ داده ی ورودی در لایه ی \lr{DU}، تقریبا مساوی میانگین نرخ داده ی ورودی لایه ی اول 
$\alpha_{s} =\alpha_{s_1} \approx \alpha_{s_2}$
می باشد.
میانگین نرخ داده ی ورودی در لایه ی \lr{DU}، تقریبا برابر میانگین نرخ داده ی ورودی در لایه ی \lr{CU} می باشد $\alpha_{s} =\alpha_{s_1} \approx \alpha_{s_2}$.
در اینجا فرض بر این است کهدر هر لایه متعادل کننده ی ترافیک موجود است که ترافیک ورودی را به طور مساوی بین \lr{VNF} ها تقسیم می نماید\cite{frdl,luong2018novel,luong2018novel1}.
فرض کنید پردازش باند پایه هر \lr{VNF}، بوسیله ی پردازش صف \lr{M/M/1} نشان داده می شود.
هر بسته بوسیله ی یکی از \lr{VNF}های برش شبکه، پردازش می شود.
بنابراین، میانگین تاخیر برش $s$ در لایه ی اول و دوم با استفاده از مدل \lr{M/M.1} به این صورت نشان داده می شود
\begin{equation}
\begin{split}
d_{s_1} &= \frac{1}{\mu_1 - \alpha_{s}/{M_{s,1}}},\\
d_{s_2} &= \frac{1}{\mu_2 - \alpha_{s}/{M_{s,2}}}.
\end{split}
\end{equation}
در اینجا
$1/\mu_1$ و $1/\mu_2$
میانگین زمان سرویس دهی در لایه ی اول و دوم به ترتیب می باشند.
$\alpha_{s}$
نرخ ورودی است که توسط متعادل کننده ی ترافیک به طور مساوی بین \lr{VNF} ها تقسیم می شود.
نرخ ورودی هر \lr{VNF} در هر لایه از برش \lr{s}، $\alpha_{s}/{M_{s,i}}$ $ i \in \{1,2\}$ می باشد.
$d_{s_{tr}}$
تاخیر ارسال برای برش $s$ در لینک وایرلس می باشد.
نرخ داده ی ورودی لینک وایرلس برابر نرخ داده ی ورودی متعادل کننده ی ترافیک برای هر برش است
\cite{frdl}.
همچنین زمان سرویس برای صف ارسال در هر برش $s$ دارای توزیع نمایی با میانگین $1/(R_{{tot}_s})$
است و می توان به صورت صف \lr{M/M/1} مدل کرد  
\cite{frdl,luong2018novel,luong2018novel1,guo2016exploiting}.
بنابراین میانگین تاخیر لایه ی ارسال بدین صورت است
\begin{equation}
 d_{s_{tr}} = \frac{1}{R_{{tot}_s} - \alpha_{s}};
\end{equation}
 $R_{{tot}_s} =  \sum_{v=1}^{V}\sum_{u=2}^{U_v}a_{v,s}R_{u(v,i)}$ 
 کل نرخ قابل دسترس در هر برش می باشد که به سرویس خاص، تخصیص داده است.
میانگین تاخیر در هر برش نیز بدین صورت است
\begin{equation}
D_{s} = d_{s_1} + d_{s_2} + d_{s_{tr}} \forall s.
\end{equation}
\chapter{تخصیص برش شبکه به صورت دینامیکی}
\section{مقدمه}
در این فصل هدف تخصیص برش شبکه به صورت می‌باشد. در فصل قبلی مدل سیستم به طور کامل نوشته شده است و در حالت آفلاین حل گردیده است، در این فصل پارامترها مورد نیاز را نسبت به فصل قبلی کمتر کرده و با استفاده از روش دینامیکی به حل سیستم می‌پردازیم. برای حل این سیستم از روش یادگیری تقویتی عمیق استفاده می‌کنیم.
\section{  مدل سیستم و صورت مسئله}
همانند سیستم فصل قبل، فرض می کنیم $S$ برش شبکه داریم که قرار است $V$ سرویس مختلف که شامل کاربرانی است که از سرویس خاص استفاده می‌نمایند را سرویس دهی نماید.
هر سرویس 
$v\in \{1,2,...,V \} $
شامل تعدادی کاربر تک آنتنه می باشند که سرویس خاصی را درخواست می‌نماید.
هر برش شبکه
$s\in \{1,2,...,S \} $
 شامل تعدادی
 PRB
  RU،
   BBU، 
   و
    VNF 
 می‌باشد.
در این بخش سعی برا‌ین است که در ابتدا مسئله را به ساده‌ترین حالت ممکن حل نماییم. فرض می‌کنیم سه مدل سرویس مختلف داریم که سرویسهای دسته‌ی اول نیازمند تاخیر خاص و سرویسهای دسته‌ی دوم نیازمند داشتن تاخیر کم هستند و سرویس سوم نیازمند داشتن هر دو حالت تاخیر کم و نرخ زیاد است.
در بخش اول این مسئله، هدف بیشینه‌سازی تعدای سرویسهای پذیرفته شده می‌باشد. در اینجا فرض براین است که تعداد برشهاس شبکه محدود می‌باشد. فرض می‌کنیم هر سرویس $v$ دارای اولویت $p_v$ می‌باشد. 
همچنین فرض براین است که هر سرویس شامل ماکسیمم $U-v$ کاربر است و به طور میانگین کاربران آن نیازمند داشتن نرخ بیشتر از $R_v$ و تاخیر کمتر از $D_v$ هستند. درصورتی که کاربری از دسته‌ی اول باشد 
$D_v =  M$ 
که $M$ برای تاخیر یک عدد بزرگ می‌باشد.
و در صورتی که سرویس از دسته‌ی دوم باشد 
$R_v = N $
که $N$ یک عدد کوچک برای نرخ می‌باشد.
صورت مسئله بدین صورت می‌باشد.
\begin{subequations}
	\begin{alignat}{4}
		\max\limits_{\boldsymbol{a} }   \quad &   \sum_{s=1}^{S}\sum_{v=1}^{V} p_v a_{v,s}\\
		\text{\lr{subject to}} \quad & \textstyle \sum_{d=1}^{D_c}\sum_{v=1}^{V}y_{s,d}a_{v,s} \geq 1\times\sum_{v=1}^{V}a_{v,s} \forall s, \\
		&\textstyle  \sum_{s=1}^{S} y_{s,d} \bar{\Omega}_{\mathfrak{z},s}^{tot}  \leq   \tau_{\mathfrak{z}_d}  \forall d,, \forall \mathfrak{z}\in \mathcal{E};  \label{eqomega}
	\end{alignat}
	\label{constraints2}
\end{subequations}
 

 


\chapter{پیشنهادات و کارهای آتی}
\section{مقدمه}
در فصل اول، مقدمه‌ای بر مفاهیم مورد استفاده را بیان کردیم و در مورد نسل پنجم مخابرات و مفاهیم آن صحبت نمودیم. سپس در فصل دوم مروری بر کارهای انجام شده کردیم و مقالات مرتبط با برش شبکه و شبکه‌های دسترسی باز و قرارگیری توابع مجازی شبکه را بیان نمودیم تا مروری بر چالشهای مطرح شده نسل پنجم مخابرات کرده و حل این چالشها را مورد بررسی قرار دادیم . در فصل سوم صورت مسئله‌ای در زمینه‌ی برش شبکه در شبکه‌های دسترسی باز، معرفی کرده و با روش ابتکاری، آن را حل نمودیم و نتایج را با مقدار بهینه مقایسه کردیم.
در  فصل چهارم، دو مسئله‌ی بیان شده در فصل سوم را به صورت کاملا ساده با روش یادگیری عمیق تقویتی به صورت دینامیکی و در هر بازه‌ی زمان حل نمودیم. این دو مسئله، MDP \LTRfootnote{Markov Decision Processs}
بوده و قابل حل با این روش هستند. 
حال در این فصل در مورد مزایا و معایب کارهای انجام شده در فصل سوم و چهارم صحبت کرده و کارهای آتی و پیشنهادات را بیان می‌کنیم.
\section{نتیجه‌گیری}
در اینجا، مسئله‌ی برش شبکه در بخش رادیویی و قرارگیری توابع مجازی شبکه برروی مراکز داده باهم مورد بررسی قرار گرفته شد.
برای حل این مسئله، ابتدا مسئیله به دو بخش مخنلف شکسته شد که در بخش اول، تخصیص برش شبکه به کاربران سرویسها و تخصیص توان حل شده و پس از آن، برشهایی از شبکه که به سرویس اختصاص داده شده را به مراکز داده نگاشت می‌دهیم.
در این مسئله، تاخیر و نرخ هر کاربر در سرویس مورد بررسی قرار گرفته شده و چالش تخصیص منابع که شامل برش بخش رادیویی به هر سرویس است و جاگیری توابع شبکه حل می‌شود.
 الگوریتم ارائه شده سرعت بسیار بیشتری از الگوریتم بهینه که با MOSEK و CVX بدست می‌آید، دارد.
 سپس مسئله به صورت ساده‌تر برای حالت دینامیکی با روش یادگیری تقویتی حل گردیده است. 
 \subsection{مزایای این چالش و حل آن}
 در مسئله‌ی بیان شده‌ی فصل سوم، مدل سیستم به صورت دقیق بیان شده و نرخ کاربر، ظرفیت لینک fronthaul و تاخیر به طور دقیق مورد بررسی قرار گرفته شده است. همچنین مسئله به واقعیت نزدیکی زیادی دارد. همچنین الگوریتم ابتکاری تعریف شده در فصل سوم برای حالتی که تداخل به نسبت کم باشد به حالت بهینه بسیار نزدیک است. در فصل چهارم همین مسئله با فرض اینکه سرویسها نیازمند تاخیر کم یا نرخ بالا هستند به صورت پارامتریک در هر لحظه از زمان حل می‌گردند. 
 در بخش بعدی چالشهای قرارگیری توابع مجازی برروی مراکز داده به طور دقیق بررسی شده و ‌در فصل چهارم این مسئله به صورت دینامیکی در هر لحظه حل گردیده است. در حل مسئله در حالت دینامیکی سعی براین است که مراکز داده کمترنی انرژی را مصرف نموده و از هدر رفت انرژی بپرهیزیم.
 \subsection{معایب  پروژه انجام شده}
 در فصل سوم از الگوریتم ابتکاری در این کار استفاده شده است. زمانی که تعداد بلوکهای منابع فیزیکی به نسبت کاربران بسیار کم باشد و تداخل به شدت زیاد گردد، الگوریتم مسئله‌ی اول به خوبی قادر به پاسخ‌گویی نیست و از حالت بهینه فاصله ‌می‌گردد. همچنین در فصل چهارم صورت مسئله بسیار ساده‌تر از واقعیت است و مسئله در حالت دینامیکی برای این حالت حل گردیده است.
 در مسئله‌ی دوم، زمانی که تعداد مراکز داده زیاد گردد فاصله‌ی حالت بهینه از الگوریتم ابتکاری زیاد شده است. 
 \subsection{نوآوری‌های این پروژه}
 در این پروژه، تخصیص توان و برش شبکه در شبکه‌های دسترسی باز مورد بررسی قرار گرفته است.
 ما مسئله‌ی اختصاص UE به خدمات، خدمات به برش‌ها و منابع فیزیکی بی سیم و همچنین مرکز داده به برش‌ها را به عنوان یک مشکل بهینه سازی فرمول‌بندی کرده‌ایم. سپس با ارائه‌ی روشهای ابتکاری، به حل آنها پرداختیم. در نهایت مسئله‌ی ساده شده را در حالت دینامیکی و متغیر با زمان حل کردیم.
 \section{پیشنهادات}
 
  

%--------------------------------------------------------------------------appendix( مراجع و پیوست ها)
\chapterfont{\vspace*{-2em}\centering\LARGE}%

\appendix
%\bibliographystyle{plain-fa}
%\bibliographystyle{ieeetr}
%\bibliography{references}
\begin{latin}
\bibliographystyle{IEEEtran}
\renewcommand{\bibname}{\rl{{کتاب‌نامه}\hfill}} 
\bibliography{IEEEabrv,references}
\end{latin}
%\include{appendix1}
%--------------------------------------------------------------------------dictionary(واژه نامه ها)
%اگر مایل به داشتن صفحه واژه‌نامه نیستید، خط زیر را غیر فعال کنید.
%\parindent=0pt
%\include{dicfa2en}%
%\include{dicen2fa}
%--------------------------------------------------------------------------index(نمایه)
%اگر مایل به داشتن صفحه نمایه نیستید، خط زیر را غیر فعال کنید.
\pagestyle{style7}
\printindex
\pagestyle{style7}
%کلمات کلیدی انگلیسی
\latinkeywords{ Cloud Radio Access Network, Multiple-Input Multiple-Output, Energy efficiency, Clusterization, Power allocation, Lagrangian function.}
%چکیده انگلیسی

\en-abstract{
Since more rate and speed is needed in technology, new generation of technology is considered that new concepts such as CRAN, mm wave, Massive MIMO and etc are defined. 
Cloud radio access networks generate a new architechture for 5G that is proposed to enhance both spectral efficiency and energy efficiency. 
The architecture of cran and the difference between this architucture and traditional one is expressed.
Also some system models such as clustering and limited fronthaul capacity is considered. In addition, D2D system in C-RAN is described too.
The optimal power allocation for the downlink and uplink of Multiple-Input Multiple-Output (MIMO) Cloud Radio Access Network (C-RAN) with limited fronthaul capacity in terms of maximizing Energy Efficiency (EE) is investigated. 
In the considered system, in downlink the compressed and precoded message generated by Central Unit (CU) is transmitted to Remote Radio Heads (RRHs) via a fronthaul link with limited capacity, and the RRHs and Users Equipments (UEs) are assumed to be clustered into $S$ cluster sets.
In uplink, the recieved message by RRHs which are clusteres into $S$ clusters, is transmitted through fronthaul link to CU and in CU beamforming vector is applied to the message.
Here, we use an iterative algorithm with Lagrangian function to optimize the EE.
Also both uplink and downlink clusters are considered in a system model and optimization is done for both together.
}
%%%%%%%%%%%%%%%%%%%%% کدهای زیر را تغییر ندهید.

\newpage
\thispagestyle{empty}
\begin{latin}
\section*{\LARGE\centering Abstract}

\een-abstract

\vspace*{.5cm}
{\large\textbf{Key Words:}}\par
\vspace*{.5cm}
\elatinkeywords
\end{latin}
% در این فایل، عنوان پایان‌نامه، مشخصات خود و چکیده پایان‌نامه را به انگلیسی، وارد کنید.
%%%%%%%%%%%%%%%%%%%%%%%%%%%%%%%%%%%%
\baselineskip=.6cm
\begin{latin}

\latinfaculty{Department of Electrical Engineering}


\latintitle{Improveing the performance of Cloud Radio Access Network with distributed cooperation}


\firstlatinsupervisor{Dr. Mohammad Javad Emadi}

%\secondlatinsupervisor{Second Supervisor}

\firstlatinadvisor{Dr. Abbas Mohammadi }

%\secondlatinadvisor{Second Advisor}

\latinname{Mojdeh}

\latinsurname{Karbalaee Motalleb}

\latinthesisdate{Month 6 \& Year 1396}

\latinvtitle
\end{latin}

\end{document}