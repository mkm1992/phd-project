\documentclass[conference]{IEEEtran}
\IEEEoverridecommandlockouts
% The preceding line is only needed to identify funding in the first footnote. If that is unneeded, please comment it out.
\usepackage[T1]{fontenc}
\usepackage{cite}
\usepackage{mathtools}
\usepackage{stackengine}
\def\delequal{\mathrel{\ensurestackMath{\stackon[1pt]{=}{\scriptstyle\Delta}}}}
\usepackage{amsmath,amssymb,amsfonts}
\usepackage{amsmath,epsfig,cite,amsfonts,amssymb,psfrag,subfig}
\usepackage{graphicx}
\usepackage{textcomp}
\usepackage{xcolor}
\usepackage{algorithm}
\usepackage[noend]{algpseudocode}
\usepackage{amsthm}
\def\BibTeX{{\rm B\kern-.05em{\sc i\kern-.025em b}\kern-.08em
    T\kern-.1667em\lower.7ex\hbox{E}\kern-.125emX}}
\allowdisplaybreaks
\newtheorem{remark}{Remark}
\newtheorem{theorem}{Theorem}
\newtheorem{lemma}{Lemma}
\newtheorem{proposition}{Proposition}
\newtheorem{corollary}{Corollary}
\newcommand{\diag}{\mathop{\mathrm{diag}}}
\DeclareMathOperator{\E}{\mathbb{E}}
\usepackage[margin=0.7in]{geometry}
\setlength{\columnsep}{11mm}
\begin{document}

\title{Joint Power Allocation and Network Slicing in an Open RAN System \vspace{-.1cm}
}
%
%\author{\IEEEauthorblockN{1\textsuperscript{st} Mojdeh Karbalaee Motalleb}
%\IEEEauthorblockA{\textit{Electrical and Computer Engineering} \\
%\textit{Tehran University}\\
%Tehran, Iran \\
%mojdeh.karbalaee@ut.ac.ir}
%\and
%\IEEEauthorblockN{2\textsuperscript{nd} Vahid Shah-Mansouri}
%\IEEEauthorblockA{\textit{Electrical and Computer Engineering} \\
%\textit{Tehran University}\\
%Tehran, Iran \\
%vmansouri@ut.ac.ir}
%\and
%\IEEEauthorblockN{3\textsuperscript{rd} Salar Nouri Naghadeh}
%\IEEEauthorblockA{\textit{Electrical and Computer Engineering} \\
%\textit{Tehran University}\\
%Tehran, Iran \\
%salar.nouri@ut.ac.ir}
%}
  \author{
    \IEEEauthorblockN{Mojdeh Karbalaee Motalleb, Vahid Shah-Mansouri, Salar Nouri Naghadeh}
    \IEEEauthorblockA{School of ECE, College of Engineering, University of Tehran, Iran \\
    Email: \{mojdeh.karbalaee, vmansouri, salar.nouri\}@ut.ac.ir,
    \vspace{-.2cm}
  }
  }

\maketitle

\begin{abstract}

\end{abstract}

\begin{IEEEkeywords}

\end{IEEEkeywords}

\section{System Model and Problem Formulation}\label{systemmodel}
In this section, we consider the downlink of an ORAN system consisting of $V$ services served by an enterprise deploying $S$ slices. We denote $v\in \{1,2,...,V \} $ and $s \in \{1,2,...,S \}$ the set of
services and slices, respectively. Each service $v$, contains $U_v$ single-antenna UEs. Moreover, each slice $s$, consists of $R_s$ multi-antenna RUs ,which contains $M$ antennas, $K_s$ physical resource blocks (PRBs), one DU and one CU. Each DU and CU consist of $M_{s,1}$ and $M_{s,2}$ VNFs, with the computational capacity
of $\mu_1$ and $\mu_2$, respectively.
\subsection{The Achievable Rate}
Here, we want to obtain achievable data rate. The achievable data rate for the $j^{th}$ UE in $v^{th}$ service can be formulated as 
\begin{equation}\label{eq1}
\mathsf{R}_{u(v,j)} = B \log_2({1+ \rho_{u(v,j)}}),
\end{equation}
where, $\rho_{u(v,j)}$ is the SNR of $j^{th}$ UE in the $v^{th}$ service that is described as
\begin{equation}\label{eq2}
\rho_{u(v,j)} =  \frac{p_{u(v,j)}\sum_{s=1}^{S}|\bold{h}_{R_s,u(v,j)}^H \bold{w}_{R_s,u(v,j)}|^2 a_{v,s}}{BN_0 + I_{u(v,j)}},
\end{equation}



\bibliographystyle{IEEEtran}
\bibliography{ref}
\end{document} 