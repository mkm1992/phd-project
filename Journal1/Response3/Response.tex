%\documentclass[12pt, draftclsnofoot, onecolumn,letterpaper]{IEEEtran}
\documentclass[12pt, letterpaper]{article}
%\documentclass[12pt, letterpaper]{IEEEtran}
\usepackage{ multirow }
\usepackage{longtable}
\usepackage{geometry}
\usepackage{ragged2e}
\usepackage[table]{xcolor}
\usepackage{booktabs}
\usepackage{graphicx}
\usepackage{caption}
\usepackage{subcaption}
\usepackage{lipsum}
\usepackage{makeidx}
\usepackage{enumerate}
\usepackage{color}
\usepackage{refstyle}
\usepackage{cite}
\usepackage{amsmath}
\usepackage{amssymb}
\usepackage{nomencl}
\usepackage{amsmath}
\usepackage{multirow}
\usepackage{graphicx}
\usepackage{multirow}
\usepackage{anysize}
\usepackage{float}
\usepackage{epstopdf}
\usepackage{threeparttable}
\usepackage{multicol}
\usepackage{amssymb}
\usepackage{adjustbox}
\usepackage{hyperref}
%\usepackage[none]{hyphenat}
%\usepackage{float}

%\usepackage{fixltx2e}
\usepackage{amsmath, amssymb, upgreek, amsthm}
\usepackage{graphicx}
\usepackage{tikz}
\geometry{letterpaper, left=20mm, right=20mm, top=20mm, bottom=20mm}
\usetikzlibrary{patterns} % LATEX and plain TEX when using Tik Z
\allowdisplaybreaks
\setlength{\textfloatsep}{2ex}
\usepackage{array}
\usepackage{enumitem}
\setlength{\parindent}{1 em}
\setlength{\parskip}{0.5 em}
\renewcommand{\baselinestretch}{1.25}
\def\dsd{d_\text{SD}}
\def\Rcoop{R_\text{Coop}}
\def\rhd{R_\text{HD}}
\def\rsd{R_\text{SD}}
\def\rsh{R_\text{SH}}
\def\Pcoops{\mathcal{P}^\text{Succ}_\text{Coop}}
\def\dsh{d_\text{SH}}
\def\dhd{d_\text{HD}}
\def\psibar{\overline{\mathcal{P}}^\text{Succ}_{i}}
\def\psbara{\overline{\mathcal{P}}^\text{Succ}_{1}}
\def\psbarb{\overline{\mathcal{P}}^\text{Succ}_{2}}
\def\psbarc{\overline{\mathcal{P}}^\text{Succ}_{3}}
\def\psbard{\overline{\mathcal{P}}^\text{Succ}_{4}}
\def\psbare{\overline{\mathcal{P}}^\text{Succ}_{5}}
\def\Ri{R_{i}}
\def\Ps{\mathcal{P}^\text{Succ}_\text{Direct}}
\def\frk{\mathrm{f}_{r_k}(r)}
\def\Rcoopj{R_\text{coop}^j}
\title{\bf \vspace*{-4ex} Statement of Responses to the Editor and the Reviewers of Paper-TNSM \\[-6ex]}
\date{}

\begin{document}
%\vspace*{-10ex}
%\sloppy
\maketitle
We would like to thank the editor and reviewers for their constructive comments on our manuscript. It has been beneficial in revising this paper, and we have improved both the technical content and presentation quality through their assistance. We greatly appreciate their generous help. Moreover, we have reviewed and incorporated all the comments and suggestions.
We hope that the modifications we have made to the manuscript and the responses we have provided herein will alleviate the reviewers' concerns. Below, please find our detailed responses to the editor and reviewers' comments and suggestions.
\\ [-3.ex]
% % % % % % % % % % % % % % % Editor % % % % % % % % % % % % % % % % % % % %


\clearpage
\noindent
\begin{longtable}{|p{0.975\textwidth}|}
\hline \hline
\Centering
\cellcolor{gray!60}
\textbf{Editor} \\
\hline \hline %\hline \hline \hline
\RaggedRight
\cellcolor{violet!15}
\textbf{\noindent  Comments to the Author} ``I think the paper should undergo a major revision. It is important to take special attention to the comments of reviewer 2.''\\
\hline
\end{longtable}

\vspace*{-1\baselineskip}
\noindent \textbf{Response:\\}
We would like to thank the Editor for his comments and concluding the revisions on our manuscript and for giving us this opportunity to improve our paper. We have used the comments to improve our paper and eliminate problems.

%\begin{longtable}{|p{0.975\textwidth}|}
%\hline \hline
%\RaggedRight
%\cellcolor{green!10}
%[1] F. Patolsky, B. P. Timko, G. Yu, Y. Fang, A. B. Greytak, G. Zheng, and C. M. Lieber, ``Detection, stimulation, and inhibition of neuronal signals with high-density nanowire transistor arrays,'' Science, vol. 313, no. 5790, pp. 1100-1104, 2006.
%\\
%\hline
%\end{longtable}




% % % % % % % % % % % % % % % Reviewer 1 % % % % % % % % % % % % % % % % % % % %
\clearpage
\noindent
\begin{longtable}{|p{.975\textwidth}|}
\hline \hline %\hline \hline \hline
\Centering
\cellcolor{gray!60}
\textbf{Reviewer 1} \\
\hline \hline %\hline \hline \hline
\RaggedRight
\cellcolor{violet!15}
\textbf{\noindent Comments to the Author} ``
The authors have addressed the comments provided in a previous review. The contributions are clear and the results are compared to some related work, so the reader can somewhat contextualize the proposed solution. ''\\
\hline
\end{longtable}
\vspace*{-1\baselineskip}
\noindent \textbf{Response:\\}
We would like to thank the reviewer for the careful and thorough reading of this manuscript. We hope that the responses provided herein can alleviate the reviewer's concerns.

\begin{longtable}{|p{0.975\textwidth}|}
\hline \hline
\RaggedRight
\cellcolor{gray!15}
\textbf{\noindent Comment1:} ``However, the paper still needs some improvement in terms of readability. Some paragraphs are too long  ''\\
\hline
\end{longtable}
\vspace*{-1\baselineskip}
\noindent \textbf{Response:\\}
We have changed the paper's introduction and checked the text entirely according to this comment.
\begin{longtable}{|p{0.975\textwidth}|}
\hline \hline
\RaggedRight
\cellcolor{gray!15}
\textbf{\noindent Comment2:} `` the figures are difficult to locate from their reference in the text (figures at top, as typical IEEE template is better), the text in the figures is too small ''\\
\hline
\end{longtable}
\vspace*{-1\baselineskip}
\noindent \textbf{Response:\\}
We have changed the paper's introduction and checked the text entirely according to this comment.
\begin{longtable}{|p{0.975\textwidth}|}
\hline \hline
\RaggedRight
\cellcolor{gray!15}
\textbf{\noindent Comment3:} ``the contributions discussed in the introduction are not easily found in the other parts of the paper.  ''\\
\hline
\end{longtable}
\vspace*{-1\baselineskip}
\noindent \textbf{Response:\\}
We have changed the paper's introduction and checked the text entirely according to this comment.





\end{document}


