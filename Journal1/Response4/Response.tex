\documentclass[12pt, letterpaper]{article}
%\documentclass[12pt, letterpaper]{IEEEtran}
\usepackage{ multirow }
\usepackage{longtable}
\usepackage{geometry}
\usepackage{ragged2e}
\usepackage[table]{xcolor}
\usepackage{booktabs}
\usepackage{graphicx}
\usepackage{caption}
\usepackage{subcaption}
\usepackage{lipsum}
\usepackage{makeidx}
\usepackage{enumerate}
\usepackage{color}
\usepackage{refstyle}
\usepackage{cite}
\usepackage{amsmath}
\usepackage{amssymb}
\usepackage{nomencl}
\usepackage{amsmath}
\usepackage{multirow}
\usepackage{graphicx}
\usepackage{multirow}
\usepackage{anysize}
\usepackage{float}
\usepackage{epstopdf}
\usepackage{threeparttable}
\usepackage{multicol}
\usepackage{amssymb}
\usepackage{adjustbox}
\usepackage{hyperref}
%\usepackage[none]{hyphenat}
%\usepackage{float}

%\usepackage{fixltx2e}
\usepackage{amsmath, amssymb, upgreek, amsthm}
\usepackage{graphicx}
\usepackage{tikz}
\geometry{letterpaper, left=20mm, right=20mm, top=20mm, bottom=20mm}
\usetikzlibrary{patterns} % LATEX and plain TEX when using Tik Z
\allowdisplaybreaks
\setlength{\textfloatsep}{2ex}
\usepackage{array}
\usepackage{enumitem}
\setlength{\parindent}{1 em}
\setlength{\parskip}{0.5 em}
\renewcommand{\baselinestretch}{1.25}
\def\dsd{d_\text{SD}}
\def\Rcoop{R_\text{Coop}}
\def\rhd{R_\text{HD}}
\def\rsd{R_\text{SD}}
\def\rsh{R_\text{SH}}
\def\Pcoops{\mathcal{P}^\text{Succ}_\text{Coop}}
\def\dsh{d_\text{SH}}
\def\dhd{d_\text{HD}}
\def\psibar{\overline{\mathcal{P}}^\text{Succ}_{i}}
\def\psbara{\overline{\mathcal{P}}^\text{Succ}_{1}}
\def\psbarb{\overline{\mathcal{P}}^\text{Succ}_{2}}
\def\psbarc{\overline{\mathcal{P}}^\text{Succ}_{3}}
\def\psbard{\overline{\mathcal{P}}^\text{Succ}_{4}}
\def\psbare{\overline{\mathcal{P}}^\text{Succ}_{5}}
\def\Ri{R_{i}}
\def\Ps{\mathcal{P}^\text{Succ}_\text{Direct}}
\def\frk{\mathrm{f}_{r_k}(r)}
\def\Rcoopj{R_\text{coop}^j}
\title{\bf \vspace*{-4ex} Statement of Responses to the Editor and the Reviewers of Paper-TNSM \\[-6ex]}
\date{}



\begin{document}
%\vspace*{-10ex}
%\sloppy
\maketitle
We would like to thank the editor and reviewers for their valuable comments on our manuscript. We have improved this paper's technical content and presentation quality through their assistance. 
We hope that the modifications undergone by the manuscript and the responses we have provided herein alleviate the reviewers' concerns. Below, please find our detailed responses to the editor and reviewers' comments and suggestions.
\\ [-3.ex]
% % % % % % % % % % % % % % % Editor % % % % % % % % % % % % % % % % % % % %


\clearpage
\noindent
\begin{longtable}{|p{0.975\textwidth}|}
\hline \hline
\Centering
\cellcolor{gray!60}
\textbf{Editor} \\
\hline \hline %\hline \hline \hline
\RaggedRight
\cellcolor{violet!15}
\textbf{\noindent  Comments to the Author} ``I think that the paper has improved substantially. However, to make the paper suitable for publication, the comments of reviewer 2 should be taken into account.''\\
\hline
\end{longtable}

\vspace*{-1\baselineskip}
\noindent \textbf{Response:\\}
Thank you for the constructive comments, which allowed us to improve our paper further. We are confident we addressed all the suggestions.
Next, you can find our point-by-point response to the reviewers' comments and how the manuscript has been modified accordingly.

%\begin{longtable}{|p{0.975\textwidth}|}
%\hline \hline
%\RaggedRight
%\cellcolor{green!10}
%[1] F. Patolsky, B. P. Timko, G. Yu, Y. Fang, A. B. Greytak, G. Zheng, and C. M. Lieber, ``Detection, stimulation, and inhibition of neuronal signals with high-density nanowire transistor arrays,'' Science, vol. 313, no. 5790, pp. 1100-1104, 2006.
%\\
%\hline
%\end{longtable}




% % % % % % % % % % % % % % % Reviewer 1 % % % % % % % % % % % % % % % % % % % %
\clearpage
\noindent
\begin{longtable}{|p{.975\textwidth}|}
\hline \hline %\hline \hline \hline
\Centering
\cellcolor{gray!60}
\textbf{Reviewer 1} \\
\hline \hline %\hline \hline \hline
\RaggedRight
\cellcolor{violet!15}
\textbf{\noindent Comments to the Author} ``
The authors have taken into consideration all the comments and suggestions raised by the Reviewer and have provided a substantial work to improve the quality of the manuscript. In that regard the Reviewer recommends the acceptance of the paper''\\
\hline
\end{longtable}
\vspace*{-1\baselineskip}
\noindent \textbf{Response:\\}
Thank you very much for your appreciation of our work and the valuable feedback we received from you.



% % % % % % % % % % % % % % % Reviewer 2 % % % % % % % % % % % % % % % % % % % %
\clearpage
\noindent
\begin{longtable}{|p{.975\textwidth}|}
\hline \hline %\hline \hline \hline
\Centering
\cellcolor{gray!60}
\textbf{Reviewer 2} \\
\hline \hline %\hline \hline \hline
\RaggedRight
\cellcolor{violet!15}
\textbf{\noindent Comments to the Author} ``
The paper addresses most comments from previous revisions. However, its organization and presentation is still too weak. ''\\
\hline
\end{longtable}
\vspace*{-1\baselineskip}
\noindent \textbf{Response:\\}
We thank the reviewer for taking the time to read the manuscript carefully, commenting thoroughly and offering suggestions that have made the manuscript stronger and more valuable. 
In this response, we hope to alleviate the reviewer's concerns.

\begin{longtable}{|p{0.975\textwidth}|}
\hline \hline
\RaggedRight
\cellcolor{gray!15}
\textbf{\noindent Comment1:} ``First, the introduction is too long and does not synthesize the expectations of the rest of the paper. It must be changed to follow a the structure followed by other papers published in TNSM. ''\\
\hline
\end{longtable}
\vspace*{-1\baselineskip}
\noindent \textbf{Response:\\}
Based on this comment, we have read several TNSM papers and reorganized the Introduction section accordingly. As a result, we shortened the Introduction section. Specifically, we revised it thoroughly and included an additional subsection about the organization of our paper. We trimmed extra parts and added more related literature reviews in the related literature. Additionally, we modified the background section by including some sentences/ideas that were originally in the Introduction section and reallocating others.
\begin{longtable}{|p{0.975\textwidth}|}
\hline \hline
\RaggedRight
\cellcolor{gray!15}
\textbf{\noindent Comment2:} ``Second, many paragraphs are too long, what makes the path of the paper difficult to follow. For instance, paragraphs in the introduction give too much detail. Such details must be moved to the appropriate section and place a summary in the introduction. ''\\
\hline
\end{longtable}
\vspace*{-1\baselineskip}
\noindent \textbf{Response:\\}
We have rewritten the introduction and literature review section, and reallocated some parts to more appropriate sections.
We have shortened long paragraphs and sentences in this paper in order to make it easier to read. Furthermore, we modified the literature review and background too. We shortened the long paragraphs in the introduction, literature review, and background sections and added some additional literature reviews to make the paper more useful.


\begin{longtable}{|p{0.975\textwidth}|}
\hline \hline
\RaggedRight
\cellcolor{gray!15}
\textbf{\noindent Comment3:} ``Regarding the technical content, the performance evaluation (Table III) must also include the related work, so that the reader knows the position of the proposed scheme in relation to dynamic resource allocation and baseline schemes. ''\\
\hline
\end{longtable}
\vspace*{-1\baselineskip}
\noindent \textbf{Response:\\}
Based on this comment, we added the results for the baseline scheme and the DR methods in Table  III in Section VII-C. Table III shows the execution time given a number of UEs for one service for the three methods. We run our simulation on the system with configuration (RAM = 8 GB, CPU = Core i5, SSD Hard Disk). 
 As the number of UEs in the system increases, the execution time increases polynomially for all three algorithms.
Since the baseline scheme is a simpler algorithm, with random PRB allocation and O-RU association based on distance, the execution time is less than the two other algorithms. Power and PRB are allocated in the DR scheme, but O-RUs are associated based on distance. Therefore the execution time is less than the proposed algorithm.
\begin{table}[H]
 \caption*{Table III: Execution Time vs. Number of UEs} 
\begin{center}
\begin{tabular}{ |l|l|l|l| }
%\hline
%\multicolumn{3}{|c|}{Country List} \\
\hline
\multirow{2}{*}{Number of UEs } &\multicolumn{3}{|c|}{Execution Time (usec)} \\
\cline{2-4}
{} &Proposed method & DR scheme & Baseline scheme \\
\hline
5 & 12.156 &8.9546 &6.6436\\
10 & 19.156   & 12.3112& 8.7870\\
15 &29.140 & 15.4778 &9.5648 \\
20    &44.573 &  21.5342 &14.8334 \\
25 & 67.912  & 30.7926 &21.5510 \\
\hline
\end{tabular}
\end{center}
\end{table}

%\begin{table}%[H]
%\vspace*{-0em}
% \caption {Execution Time vs. Number of UEs} \label{table:2}
% \vspace*{-1em}
% \begin{center}
% \scalebox{0.9}{
%  \begin{tabular}{|l| l| l| l| l| l|}
%   \hline 
%  Number of UEs & 5 & 10 & 15 & 20 & 25 \\
%  \hline 
%  Execution Time(usec) &  &  &  &  &  \\
%  Proposed Method & 12.156 & 19.156 &  29.140 & 44.573 & 67.912  \\ [.3ex]
%
%  DR method &  9.9546 & 13.3112 &  15.4778 & 17.5342 & 30.7926 \\ [.3ex]
%  Baseline scheme &  6.6436 &   7.7870  &  8.5648  & 9.8334 &  14.5510 \\ [.3ex]
%   \hline 
% \end{tabular}}
% \end{center}
% \vspace*{-2.5em}
% \end{table}


\begin{longtable}{|p{0.975\textwidth}|}
\hline \hline
\RaggedRight
\cellcolor{gray!15}
\textbf{\noindent Comment4:} ``Moreover, it is not clear why the fact that a function increase justifies the convergence of the algorithm. ''\\
\hline
\end{longtable}
\vspace*{-1\baselineskip}
\noindent \textbf{Response:\\}
We thank the reviewer for reporting this ambiguity. We revised this vague statement and added it to section VI-C-2. Below we reported the revised sentences.


Due to limited system resources, we have limits on VNFs' power, UE or O-RU power, fronthaul capacity, etc. As a result, the objective function, which is the aggregate throughput, cannot exceed its optimal value and become infinite. Therefore, if the aggregate throughput is infinite and increases without limit, the resources must also be unlimited. Hence, the system has an optimal solution: its maximum aggregate throughput in the feasible region.
Consequently, we can guarantee the convergence of the iterative algorithm if the objective function is the ascending function concerning the number of iterations. Thus, it will converge to its optimum value if it is a strictly ascending function and to its local optimum if it is a non-monotonically ascending function.

\begin{longtable}{|p{0.975\textwidth}|}
\hline \hline
\RaggedRight
\cellcolor{gray!15}
\textbf{\noindent Comment5:} ``In addition, only the convergence to the local optimum that is closest to the initial values is achieved, and it is not clear how it can be extended to global convergence. ''\\
\hline
\end{longtable}
\vspace*{-1\baselineskip}
\noindent \textbf{Response:\\}
We agree with the reviewer that we only talked about the convergence of the proposed algorithm, and we do not talk about the global and local convergence. 

Moreover, Fig. 12 shows the aggregate throughput vs. the number of UEs for the optimal solution and proposed algorithm. Therefore, there is a low interference comparison between the proposed method and the optimal solution for the system. In this figure, the proposed solution is near the optimal solution. Thus, the simulation demonstrates that it almost reaches the global optimum. Hence, Fig. 12 numerically illustrates the global convergence of the algorithm.

In addition, to extend our solution to the global optimum, we must prove that the algorithm monotonically increases in the non-optimal set of solutions.

 Moreover, in the low interference system, in which the PRB and VNF assignment is obtained straightforward, and the whole problem of the first step is about power allocation, in the first step of our algorithm that the optimal power and PRB allocation are obtained for the fixed O-RU association; we have
$\mathcal{T}(\bold{P}^{i},\bold{E}^i,\bold{G}^{i-1}) > \mathcal{T}(\bold{P}^{i-1},\bold{E}^{i-1},\bold{G}^{i-1})$ (strictly increase).
 Since the PRB assignment is straightforward and we used the KKT condition for power allocation, the problem is a convex optimization. The KKT condition problem has a global convergence and achieves the optimal solution in convex optimization. Accordingly, the algorithm monotonically increases to reach its optimum solution. However, in the second step of the algorithm, we can prove that the algorithm is increased, but we can not talk about monotonically increases.
Hence we have
$\mathcal{T}(\bold{P}^{i},\bold{E}^i,\bold{G}^{i}) \geq \mathcal{T}(\bold{P}^{i},\bold{E}^{i},\bold{G}^{i-1})$.
Therefore, we have 
$\mathcal{T}(\bold{P}^{i},\bold{E}^i,\bold{G}^{i}) > \mathcal{T}(\bold{P}^{i-1},\bold{E}^{i-1},\bold{G}^{i-1})$.
Consequently, the algorithm converges to the 
global optimum solution in low interference.


Meanwhile, Zangwill's global convergence theory of iterative algorithms is a standard theory for providing the global convergence of an iterative algorithm. We need to use this theory to prove our algorithm in a standard way which is not straightforward in this paper.

\begin{longtable}{|p{0.975\textwidth}|}
\hline \hline
\RaggedRight
\cellcolor{gray!15}
\textbf{\noindent Comment6:} ``In general, the paper must clarify to "what" is the algorithm converging or the global maximum can be understood by the reader and it can be confusing. ''\\
\hline
\end{longtable}
\vspace*{-1\baselineskip}
\noindent \textbf{Response:\\}
We agree with the reviewer that we must clarify "what" the algorithm converges. As mentioned in the previous question, we prove that the algorithm can converge to a global convergence in low interference.
Moreover, in Fig. 12, we numerically demonstrate our algorithm's global convergence in low interference.
Nevertheless, although we didn't demonstrate the unconditional global convergence of our algorithm, we did verify its unconditional local convergence. 
\begin{longtable}{|p{0.975\textwidth}|}
\hline \hline
\RaggedRight
\cellcolor{gray!15}
\textbf{\noindent Comment7:} ``Regarding presentation, there are some grammar typos, such as "reformulated as follow" in Section VI.B, second paragraph.  ''\\
\hline
\end{longtable}
\vspace*{-1\baselineskip}
\noindent \textbf{Response:\\}
We are pleased to thank the reviewer for improving the grammar in our paper. We have read the paper one more time and tried to fix any grammar problems in the paper.
\begin{longtable}{|p{0.975\textwidth}|}
\hline \hline
\RaggedRight
\cellcolor{gray!15}
\textbf{\noindent Comment8:} ``Finally, according to IEEE style, tables and algorithms must be placed at top of the page, in addition to figures, which have already been correctly placed.  ''\\
\hline
\end{longtable}
\vspace*{-1\baselineskip}
\noindent \textbf{Response:\\}
It is our pleasure to thank the reviewer for this comment. We agree with the reviewer that the
tables and algorithms need to be moved. As a result, all these mistakes were corrected.

\clearpage
\noindent
\begin{longtable}{|p{.975\textwidth}|}
\hline \hline %\hline \hline \hline
\Centering
\cellcolor{gray!60}
\textbf{Reviewer 3} \\
\hline \hline %\hline \hline \hline
\RaggedRight
\cellcolor{violet!15}
\textbf{\noindent Comments to the Author} ``
The paper has addressed the reviewers' comments satisfactorily. ''\\
\hline
\end{longtable}
\vspace*{-1\baselineskip}
\noindent \textbf{Response:\\}
We appreciate the reviewer's time and attention in reading this manuscript. 

\begin{longtable}{|p{0.975\textwidth}|}
\hline \hline
\RaggedRight
\cellcolor{gray!15}
\textbf{\noindent Comment1:} ``One minor issue is in Eq. (3) where the plus sign should be in the position after the quantization noise term. ''\\
\hline
\end{longtable}
\vspace*{-1\baselineskip}
\noindent \textbf{Response:\\}
We would like to thank the reviewer for correcting our mistake. We modified the position of the plus sign, and the updated version is below.
\begin{align}\label{eqI}
&I_{r,u(s,i)}^{k} =\underbrace{  \sum_{j=1}^{{R}} {\sigma_q}^2 |\boldsymbol{h}_{r,{u(s,i)}}^k|^2 }_{\text{(quantization noise)}} + \nonumber\\
 &\underbrace{\sum_{\substack{l=1 \\ l\neq i}}^{{U}_{s}} e^{k}_{u(s,i)}e^{k}_{u(s,l)}  p_{u(s,l)}^{k}\sum_{\substack{r'=1 \\ r'\neq r}}^{R}|{\bold{h}_{r',u(s,i)}^{k \: H}} \bold{w}_{r',u(s,l)}^{k} g_{u(s,l)}^{r'}|^2}_{\text{(intra-slice interference)}},
\end{align}

\end{document}


