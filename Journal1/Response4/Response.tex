%\documentclass[12pt, draftclsnofoot, onecolumn,letterpaper]{IEEEtran}
\documentclass[12pt, letterpaper]{article}
%\documentclass[12pt, letterpaper]{IEEEtran}
\usepackage{ multirow }
\usepackage{longtable}
\usepackage{geometry}
\usepackage{ragged2e}
\usepackage[table]{xcolor}
\usepackage{booktabs}
\usepackage{graphicx}
\usepackage{caption}
\usepackage{subcaption}
\usepackage{lipsum}
\usepackage{makeidx}
\usepackage{enumerate}
\usepackage{color}
\usepackage{refstyle}
\usepackage{cite}
\usepackage{amsmath}
\usepackage{amssymb}
\usepackage{nomencl}
\usepackage{amsmath}
\usepackage{multirow}
\usepackage{graphicx}
\usepackage{multirow}
\usepackage{anysize}
\usepackage{float}
\usepackage{epstopdf}
\usepackage{threeparttable}
\usepackage{multicol}
\usepackage{amssymb}
\usepackage{adjustbox}
\usepackage{hyperref}
%\usepackage[none]{hyphenat}
%\usepackage{float}

%\usepackage{fixltx2e}
\usepackage{amsmath, amssymb, upgreek, amsthm}
\usepackage{graphicx}
\usepackage{tikz}
\geometry{letterpaper, left=20mm, right=20mm, top=20mm, bottom=20mm}
\usetikzlibrary{patterns} % LATEX and plain TEX when using Tik Z
\allowdisplaybreaks
\setlength{\textfloatsep}{2ex}
\usepackage{array}
\usepackage{enumitem}
\setlength{\parindent}{1 em}
\setlength{\parskip}{0.5 em}
\renewcommand{\baselinestretch}{1.25}
\def\dsd{d_\text{SD}}
\def\Rcoop{R_\text{Coop}}
\def\rhd{R_\text{HD}}
\def\rsd{R_\text{SD}}
\def\rsh{R_\text{SH}}
\def\Pcoops{\mathcal{P}^\text{Succ}_\text{Coop}}
\def\dsh{d_\text{SH}}
\def\dhd{d_\text{HD}}
\def\psibar{\overline{\mathcal{P}}^\text{Succ}_{i}}
\def\psbara{\overline{\mathcal{P}}^\text{Succ}_{1}}
\def\psbarb{\overline{\mathcal{P}}^\text{Succ}_{2}}
\def\psbarc{\overline{\mathcal{P}}^\text{Succ}_{3}}
\def\psbard{\overline{\mathcal{P}}^\text{Succ}_{4}}
\def\psbare{\overline{\mathcal{P}}^\text{Succ}_{5}}
\def\Ri{R_{i}}
\def\Ps{\mathcal{P}^\text{Succ}_\text{Direct}}
\def\frk{\mathrm{f}_{r_k}(r)}
\def\Rcoopj{R_\text{coop}^j}
\title{\bf \vspace*{-4ex} Statement of Responses to the Editor and the Reviewers of Paper-TNSM \\[-6ex]}
\date{}

\begin{document}
%\vspace*{-10ex}
%\sloppy
\maketitle
We would like to thank the editor and reviewers for their constructive comments on our manuscript. They have been beneficial in revising this paper, and we have improved the technical content and presentation quality through their assistance. We greatly appreciate their generous help. We also aim for a more concise manuscript.
We hope that the modifications we have made to the manuscript and the responses we have provided herein will alleviate the reviewers' concerns. Below, please find our detailed responses to the editor and reviewers' comments and suggestions.
\\ [-3.ex]
% % % % % % % % % % % % % % % Editor % % % % % % % % % % % % % % % % % % % %


\clearpage
\noindent
\begin{longtable}{|p{0.975\textwidth}|}
\hline \hline
\Centering
\cellcolor{gray!60}
\textbf{Editor} \\
\hline \hline %\hline \hline \hline
\RaggedRight
\cellcolor{violet!15}
\textbf{\noindent  Comments to the Author} ``I think that the paper has improved substantially. However, to make the paper suitable for publication, the comments of reviewer 2 should be taken into account.''\\
\hline
\end{longtable}

\vspace*{-1\baselineskip}
\noindent \textbf{Response:\\}
It is our pleasure to thank the Editor for his constructive comments on how to improve our paper. Thanks to the comments, we were able to make improvements to our paper and eliminate problems. In response to the reviewers and Editor's comments, we are confident we addressed their suggestions.
We have updated the manuscript and uploaded our point-by-point response to the reviewers' comments (below).

%\begin{longtable}{|p{0.975\textwidth}|}
%\hline \hline
%\RaggedRight
%\cellcolor{green!10}
%[1] F. Patolsky, B. P. Timko, G. Yu, Y. Fang, A. B. Greytak, G. Zheng, and C. M. Lieber, ``Detection, stimulation, and inhibition of neuronal signals with high-density nanowire transistor arrays,'' Science, vol. 313, no. 5790, pp. 1100-1104, 2006.
%\\
%\hline
%\end{longtable}




% % % % % % % % % % % % % % % Reviewer 1 % % % % % % % % % % % % % % % % % % % %
\clearpage
\noindent
\begin{longtable}{|p{.975\textwidth}|}
\hline \hline %\hline \hline \hline
\Centering
\cellcolor{gray!60}
\textbf{Reviewer 1} \\
\hline \hline %\hline \hline \hline
\RaggedRight
\cellcolor{violet!15}
\textbf{\noindent Comments to the Author} ``
The authors have taken into consideration all the comments and suggestions raised by the Reviewer and have provided a substantial work to improve the quality of the manuscript. In that regard the Reviewer recommends the acceptance of the paper''\\
\hline
\end{longtable}
\vspace*{-1\baselineskip}
\noindent \textbf{Response:\\}
It is a pleasure to have the reviewer read this manuscript; we appreciate his/her attention, which helps us enhance our paper. 



% % % % % % % % % % % % % % % Reviewer 2 % % % % % % % % % % % % % % % % % % % %
\clearpage
\noindent
\begin{longtable}{|p{.975\textwidth}|}
\hline \hline %\hline \hline \hline
\Centering
\cellcolor{gray!60}
\textbf{Reviewer 2} \\
\hline \hline %\hline \hline \hline
\RaggedRight
\cellcolor{violet!15}
\textbf{\noindent Comments to the Author} ``
The paper addresses most comments from previous revisions. However, its organization and presentation is still too weak. ''\\
\hline
\end{longtable}
\vspace*{-1\baselineskip}
\noindent \textbf{Response:\\}
Throughout this process, we thank the reviewer for taking the time to read the manuscript carefully, commenting thoughtfully, and offering suggestions that made the manuscript stronger and more valuable. 
In this response, we hope to alleviate the reviewer's concerns.

\begin{longtable}{|p{0.975\textwidth}|}
\hline \hline
\RaggedRight
\cellcolor{gray!15}
\textbf{\noindent Comment1:} ``First, the introduction is too long and does not synthesize the expectations of the rest of the paper. It must be changed to follow a the structure followed by other papers published in TNSM. ''\\
\hline
\end{longtable}
\vspace*{-1\baselineskip}
\noindent \textbf{Response:\\}
Based on this comment, we have read several TNSM papers, rewritten the introduction's paragraph, removed the additional parts, and tried to enhance it as much as possible. 

\begin{longtable}{|p{0.975\textwidth}|}
\hline \hline
\RaggedRight
\cellcolor{gray!15}
\textbf{\noindent Comment2:} ``Second, many paragraphs are too long, what makes the path of the paper difficult to follow. For instance, paragraphs in the introduction give too much detail. Such details must be moved to the appropriate section and place a summary in the introduction. ''\\
\hline
\end{longtable}
\vspace*{-1\baselineskip}
\noindent \textbf{Response:\\}
We have rewritten the introduction and literature review section, removed the additional part, and put it in the correct sections.
We have shortened long paragraphs and sentences in this paper in order to make it easier to read.


\begin{longtable}{|p{0.975\textwidth}|}
\hline \hline
\RaggedRight
\cellcolor{gray!15}
\textbf{\noindent Comment3:} ``Regarding the technical content, the performance evaluation (Table III) must also include the related work, so that the reader knows the position of the proposed scheme in relation to dynamic resource allocation and baseline schemes. ''\\
\hline
\end{longtable}
\vspace*{-1\baselineskip}
\noindent \textbf{Response:\\}

\begin{longtable}{|p{0.975\textwidth}|}
\hline \hline
\RaggedRight
\cellcolor{gray!15}
\textbf{\noindent Comment4:} ``Moreover, it is not clear why the fact that a function increase justifies the convergence of the algorithm. ''\\
\hline
\end{longtable}
\vspace*{-1\baselineskip}
\noindent \textbf{Response:\\}
We thank the reviewer for reporting this ambiguity. We revised this vague statement and added it to section VI-C-2. Below we reported the revised sentences.


Due to limited resources in power, the number of PRBs, and the number of activated VNFs and restrictions based on this limitation on power, energy, fronthaul capacity, etc., the objective function that is the summation of the achievable rates of UEs cannot exceed its optimal value and become infinite. Therefore, we can guarantee the convergence of the iterative algorithm if the objective function is the strictly ascending function concerning the number of iterations. Consequently, it converges to the optimum value.

\begin{longtable}{|p{0.975\textwidth}|}
\hline \hline
\RaggedRight
\cellcolor{gray!15}
\textbf{\noindent Comment5:} ``In addition, only the convergence to the local optimum that is closest to the initial values is achieved, and it is not clear how it can be extended to global convergence. ''\\
\hline
\end{longtable}
\vspace*{-1\baselineskip}
\noindent \textbf{Response:\\}


\clearpage
\noindent
\begin{longtable}{|p{.975\textwidth}|}
\hline \hline %\hline \hline \hline
\Centering
\cellcolor{gray!60}
\textbf{Reviewer 3} \\
\hline \hline %\hline \hline \hline
\RaggedRight
\cellcolor{violet!15}
\textbf{\noindent Comments to the Author} ``
The paper has addressed the reviewers' comments satisfactorily. ''\\
\hline
\end{longtable}
\vspace*{-1\baselineskip}
\noindent \textbf{Response:\\}
We appreciate the reviewer's time and attention in reading this manuscript. 

\begin{longtable}{|p{0.975\textwidth}|}
\hline \hline
\RaggedRight
\cellcolor{gray!15}
\textbf{\noindent Comment1:} ``One minor issue is in Eq. (3) where the plus sign should be in the position after the quantization noise term. ''\\
\hline
\end{longtable}
\vspace*{-1\baselineskip}
\noindent \textbf{Response:\\}
We would like to thank the reviewer for correcting our mistake. We modified the position of the plus sign, and the updated version is below.
\begin{align}\label{eqI}
&I_{r,u(s,i)}^{k} =\underbrace{  \sum_{j=1}^{{R}} {\sigma_q}^2 |\boldsymbol{h}_{r,{u(s,i)}}^k|^2 }_{\text{(quantization noise)}} + \nonumber\\
 &\underbrace{\sum_{\substack{l=1 \\ l\neq i}}^{{U}_{s}} e^{k}_{u(s,i)}e^{k}_{u(s,l)}  p_{u(s,l)}^{k}\sum_{\substack{r'=1 \\ r'\neq r}}^{R}|{\bold{h}_{r',u(s,i)}^{k \: H}} \bold{w}_{r',u(s,l)}^{k} g_{u(s,l)}^{r'}|^2}_{\text{(intra-slice interference)}},
\end{align}

\end{document}


